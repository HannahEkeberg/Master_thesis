\chapter{Discussion}



\section{Uncertainties}
While the uncertainties in intensity,  and in efficiency, ......, the uncertainties in the number of counts (and number of decays???) have an independent impact on each activity and thus cross section. 
The relative efficiency uncertainty is typically ca. 5\%. ......

For the activitities, the scipy curve fit uses chi squared to estimate the A0, and thus the error is weighted. If there is enough points with low uncertainty the uncertainty in A0 will also be low. That is important, for the beam current and the cross sections. 

\section{Beam current variance minimization}

The variance minimization (figure \ref{fig:varmin_beamcurrent}) led to an overall more constant weighted average beam current after the variance minimization, where Ni(d,x)$^{56}$Co and Cu(d,x)$^{62}$Zn in particular were affected positively. The other reactions have about the same values before and after variance minimization. With the exception of compartment 9, where the beam current is higher than before the variance minimization, the beam currents are more consistent (table \ref{tab:weighted_BC}). Figure \ref{fig:varmin_beamcurrent} clearly show an improvement of $^{56}$Co and $^{63}$Zn. The gamma-lines used to estimated the end of beam activities are listed in tables \ref{tab:Products_Fe}, \ref{tab:Products_Ni}, \ref{tab:Products_Cu} and \ref{tab:Products_Ir} for iron, nickel, copper and iridium respectively. \\ 

\noindent
The monitor reaction  $^\text{nat}$Fe(d,x)$^{56}$Co ($t_{1/2}$=77.236 d) decays by $\epsilon$ (100\%) to stable $^{56}$Fe \cite{Junde2011}. Figure \ref{fig:monitor_BC+CS}a shows that the three cross section measurements are in very well agreement with the recommended IAEA cross sections. This nucleus can be produced directly via $^{56}$Fe(d,2n) (Q=-7.6 MeV), $^{57}$Fe(d,3n) (Q=-14.2 MeV) and via $^{d,4n}$ (-25.3 MeV). There is no feeding for this decay channel and the reaction is independent $^{56}$Co. The activity of $^{56}$Co was thus estimated using onestep decay, with the gamma-lines listed in table \ref{tab:Products_Fe}. All the gamma-lines are not present in background spectra with the exception of 1238.288 keV (66.46\%), with a countrate of $10^{-2}$, which is very low in comparison to the countrate of the peak. In general, $^{56}$Co shares multiple lines with $^{56}$Mn. \textcolor{red}{should exclude or do background subtraction??}. On figure \ref{fig:varmin_beamcurrent}, the datapoints for this reaction are in well agreement with the weighted average beam energy, but has a slightly lower value. The variance minimization did not change the beam current very much for this reaction. \\

\noindent 
The monitor reaction $^\text{nat}$(d,x)$^{61}$Cu ($t_{1/2}$=3.339 h) decay by $\epsilon$ (100\%) to stable $^{61}$Ni \cite{Zuber2015}. Figure \ref{fig:monitor_BC+CS}b shows datapoints which are in accepted agreement with the recommended IAEA cross sections. However, in foil 1 and foil 2 the cross section is slightly overestimated, while foil 3-5 underestimates slightly. The cross section is difficult to measure precisely when the derivative of the excitation function is high, so the cross section of foil 10 does not fit the peak accurately, but all the measured cross section points are within the energy uncertainty. The activity was estimated using onestep decay, with the gamma-lines listed in table \ref{tab:Products_Ni}. On figure \ref{fig:varmin_beamcurrent}, it is clear that the measured beam current points "oscillates" between the weighted average beam current, but the points are in well compliance with the weighted average beam current. The variance minimization did not change the beam current very much for this reaction. \\

\noindent 
The monitor reaction $\text{nat}$(d,x)$^{56}$Co ($t_{1/2}$=77.236 d) decays by $\epsilon$ (100\%) to stable $^{56}$Fe \cite{Junde2011}.Figure \ref{fig:monitor_BC+CS}c shows the cumulative cross section with feeding from $^{56}$Ni ($\epsilon:100\%$). However, since the Q-value (table \ref{tab:Products_Ni}) for Ni(d,x)$^{56}$Ni is 24.6884 MeV, the activity of $^{56}$Co in foil 1, 2 and 3 which was exposed for deuterons energy above the threshold, were estimated using twostep decay, while the remaining activities were estimated using one-step decay. The gamma-lines used for the identification and estimation of activity are listed in table \ref{tab:Products_Ni}. The cross section reported from IAEA is cumulative, thus the independent cross sections were summed to get the cumulative cross section. The measured cross section points are slightly overestimated in comparison to the recommended IAEA cross sections, but are in well compliance with the experimental data, and are thus accepted results. On figure \ref{fig:varmin_beamcurrent}, the overall estimated beam current for this monitor reaction shows that it is clearly high in comparison to the weighted average beam energy. However, the variance minimization helped to "pull" the values down, especially in foils 1-3. \\

\noindent 
The monitor reaction $^\text{nat}$Ni(d,x)$^{58}$Co (figure \ref{fig:monitor_BC+CS}d) shows the cumulative cross section of the ground state ($t_{1/2}$=70.86 d) which decay by $\epsilon$ (100\%) with feeding from the isomer $^{58m}$Co ($t_{1/2}$=9.19 h, IT:100\%) \cite{Nesaraja2010}. Since $^{58m}$Co lack observable gamma-lines, the activity was estimated using twostep decay, with two optimizing parameters, inwhich were the activities for both of them independently. Since the IAEA cross section is cumulative, the cross section provided here is the summation of the two independent cross sections of the isomer and ground state. Within uncertainties the estimated cumulative cross section of $^{58}$Co is in well agreement with the recommended IAEA cross sections. On figure \ref{fig:varmin_beamcurrent}, the estimated beam current is also in very well agreement with the weighted average beam energy. The variance minimization did no change the values very much.    \\

\noindent
figure \ref{fig:monitor_BC+CS}e shows the monitor cross section for reaction $^\text{nat}$Cu(d,x)$^{62}$Zn ($t_{1/2}$=9.193 h) decays by $\epsilon$ (100\%) \cite{Nichols2012}. The activities for this product was estimated using onestep decay. The measured cross sections appear to be in well agreement with the IAEA recommended cross sections. This reaction has a Q-value (table \ref{tab:Products_Cu}) -15.490 MeV, thus the cross section in foil 7-10 is below threshold and therefor zero. In figure \ref{fig:varmin_beamcurrent}, the measured beam current in foils 1-3 is in perfect compliance with the weighted average beam current, and foils 4-6 are slightly above. The uncertainty in the measured beam current in foil 6 is overestimated, which is because it is close to the threshold. The variance minimization contributed to "pulling" the beam current down, and reduce the uncertainty in foil number 6. \\

\noindent 
Figure \ref{fig:monitor_BC+CS}f shows the monitor cross section for reaction $^\text{nat}$Cu(d,x)$^{63}$Zn ($t_{1/2}$=38.47 m) decay by $\epsilon$ (100\%) \cite{ERJUN2001}. The activities for this product was estimated using onestep decay. The Q-value (table \ref{tab:Products_Cu}) for this reaction is -6.3733 MeV, and the reaction cross section is below threshold for foil 10. The cross sections estimated in foils 7-9 are in well compliance with the recommended cross section data, but the cross section in foil 8 is slightly underestimated. The rest of the points are off, and are overestimated, but it follows the expected shape. The beam current on figure \ref{fig:varmin_beamcurrent} shows that the overall beam current for foils 1-6 is high, the beam current in foil 7 and 9 are in agreement with the weighted average beam current, and the beam current in foil 8 is underestimated. This particular cross section was tested for various beam currents with low values of $\chi^2$ in each compartment, but did not change the cross section in figure \ref{fig:monitor_BC+CS}f such that it was aligned within uncertainties of the IAEA cross section. This particular reaction thus contributes to the weighted average beam energy being pulled up slightly. \\

\newline 
Figure \ref{fig:monitor_BC+CS}g shows the monitor cross section for reaction $^\text{mat}$Cu(d,x)$^{65}$Zn ($t_{1/2}$=243.93 d) decays by $\epsilon$ (100\%) \cite{Browne2010}. The activities for this product was estimated using onestep decay. The measured cross sections in foil 1-5 is in excellent agreement with the IAEA recommended cross section data. The remaining points are within uncertainties agreeing with the IAEA data. The beamcurrent in figure \ref{fig:varmin_beamcurrent} shows that the beam current for this reaction is in well compliance wiht the weighted average beam energy, and likewise Fe(d,x)$^{56}$Co, the values are slightly lower contributing to a lower weighted average beam energy. The variance minimization did not affect the estimated beam current particularly much.  \\

\noindent 
To summarize, using the weighted average beam current (figure \ref{fig:monitor_BC+CS}) in general shows good compliance with the recommended IAEA cross sections that were used in the calculation of the beam current. In general, the measured cross section values are within uncertainty of the recommended cross section. The high energy measurements are in general more precise, and it is clear that where it is a rapid change in the excitation function, the energy bins seems to be slightly wrong. However, the measured cross sections are within uncertainty in both energy and cross section, with the exception being the high energy exponential tale for $^{63}$Zn. 

\section{Cross section products}

In general, the products that were expected were mostly seen, with a few exceptions which were mainly due to too short counting time and shared gamma-lines which were particularly evident in the iridium products. The Coulomb barrier for the compound nuclei resulting from deuterons were ca. 13.2 MeV for the compound platinum nuclei ($^\text{nat}$Ir(d,x)$^{193^*/195^*}$Pt), ca. 6.7 MeV for the compound copper nuclei ($^\text{nat}$Ni(d,x)$^{60^*/62^*/63^*/64^*/66^*}$Cu), ca. 6.9 MeV for the compound zink nuclei for protons... ($\text{nat}$Cu(d,x)$^{65^*/67^*}$Zn) and ca. 6.4 MeV for the compound cobalt nuclei for protons  ($^\text{nat}$Fe(d,x)$^{56^*/58^*/59^*/60^*}$Co). The observed products resulting from iridium were constrained by the high Coulomb barrier, and only radionuclides of iridium and platinum were observed. \\

In general, there was no clear evidence of decay of compound nuclei by emission of tritons, deuterons or $^{3}$He nuclei. As mentioned in the section \ref{....}, for compound nuclear reactions, decay channels with emission of single nucleons are more fed, and alpha-particles due to the large binding energy, but all energetically possible decay channels will be fed. The cross sections for those reaction channels were thus too low or overshadowed by other compound peaks to be observed. \\

Due to energy straggling and scattering the further back in the stack, the uncertainty in energy increeases with beam energy. 



%In general, we got most of the results that we expected to see, based upon energy threshold, decay channel of compound nucleus which would increase the cross section and based upon the Coulomb barrier. 

%Generally, there is no evidence of decay by triton, deuteron or $^{3}$He, allthough they are energetically possible. This is due to the higher spin, so that decay channels with lower spin is favoured instead. Alpha-decay happens for Iron, Nickel and Cupper, but as we can see in  table.... the Coulomb barrier (equation ...) is a lot higher. Thus we do not see products lower than one p down (platinum and iridium). 
%For Iron (Z=26) Nickel (Z=28) and Cupper (Z=29), the Coulomb barries is almost half the 

\subsection{Nickel products}
From nickel, there are 5 stable nuclei, $^{58}$Ni (68.077\%), $^{60}$Ni (26.223\%), $^{61}$Ni (1.1399\%), $^{62}$Ni (3.6346\%) and $^{64}$Ni (0.9255\%), resulting in many possible reactions. The observed reactions were the cumulative measurements of $^{52}$Mn, $^{59}$Fe, $^{55}$Co, $^{56}$Co, $^{5}$Co, $^{58}$Co, $^{60}$Co, $^{56}$Ni, $^{65}$Ni and $^{57}$Ni and the independent measurements of $^{54}$Mn, $^{56}$Co, $^{57}$Co, $^{58}$Co, $^{58m}$Co, $^{60}$Cu and $^{64}$Cu. In addition, the $^{59}$Fe (cumulative), $^{58}$Co (independent) cross sections are the first reported.\\





\subsubsection{\centering{\small{$^{52}$Mn (cumulative)}}}
The cumulative cross section for $^{52m}$Mn ($t_{1/2}$=21.1 m) and $^{52g}$Mn ($t_{1/2}$=5.591 d) \cite{Dong2015} is reported in table... and can be seen in figure ... The isomer decays by $\epsilon$ (98.25\%) to $^{52}$Cr and by IT (1.75\%), hence the feeding is relatively low. The isomer and ground state share multiple gamma-lines, and the first registered count of $^{52}$Mn are about 2-3 hours after end of beam, where the isomer had more or less decayed out. Therefor, it was difficult to extract independent measurements of the isomer and ground state. $^{52}$Mn takes place in the decay chain decaying via $\epsilon$ from $^{52}$Co (104 ms), $^{52m+g}$Fe (45.9 s/8.275 h) to stable $^{52}$Cr, but there was no evidence of the two former, thus this reaction is not subject to much feeding. The end of beam activities were estimated using onestep decay, with the gamma-lines listed in table \ref{tab:Products_Ni}. This product can be produced via  $^{58}$Ni(d,2$\alpha$) with reaction Q-value -1.2356 MeV, via $^{60}$Ni(d,2n2$\alpha$) with reaction Q-value -21.6226  MeV and via $^{61}$Ni(d,3n2$\alpha$) with reaction Q-value -29.4427 MeV. It is clear that the alpha-reaction channels are constrained by the Coulomb barrier, which is approximately 14.4 MeV for alpha-particles from the compound nuclei, as the first observed measurement was at 14.63 MeV. With the two latter reaction channels, the excitation function steadily increases, and combined becomes a large compound peak. \\
\noindent 
The measured datapoints agrees with previous experimental data \cite{Hermanne2013},\cite{Takacs2007},\cite{Usman2016},\cite{Amjed2013}. CoH overestimates the compound peak, but follows the shape well, but the maximum seems to be slightly shifted towards lower energies. This is also the case for TENDL and TALYS, but here the cross sections are underestimated. ALICE follows the experimental shape of the excitation function, and allthough it underestimates the cross section values, it is still a relatively good model. \textcolor{red}{EMPIRE----}. 

\subsubsection{\centering{\small{$^{54}$Mn (independent)}}}
The independent cross section for $^{54}$Mn ($t_{1/2}$=312.20 d) \cite{Dong2014}, are reported in table ... and in figure ..., and decays via $\epsilon:100\%$ to stable $^{54}$Cs. The end of beam activities were estimated using a onestep decay the gamma-line listed in table \ref{tab:Products_Ni}. $^{54}$Mn can be produced via $^{58}$Ni(d,2p$\alpha$) with reaction Q-value -8.5383 MeV, via $^{60}$Ni(d,2$\alpha$) with reaction Q-value -0.6296 MeV, via $^{61}$Ni(d,n2$\alpha$) with reaction Q-value -8.4497 MeV and via $^{62}$Ni(d,2n2$\alpha$) with reaction Q-value -19.0454 MeV. The alpha-reaction channel is here also constrained by the alpha-barrier, which was approximately 14.4 MeV. However, there is one measurement below this alpha barrier, at 11.41 MeV. \textcolor{red}{is this due to tunneling of the alpha-particle or or an error, most likely not any other decay channel because of the energy threshold? The background spectra from cave 4c (detector 1-6) shows a gamma-line which is less than 1 keV in difference from the gamma-line used in the estimation of the cross section?}. The datapoints before 14 MeV is most likely false. Should do background subtraction? The compound peak increases steadily as more reaction channels open. \\ 
\noindent 
For the two first points measured at 11.41 and 14.63 MeV, there was no previous experimental data: \cite{Hermanne2013, Takacs2007, Usman2016,  Amjed2013}. , and comparing to the reaction models, the suggested cross section is zero. However, the measured cross sections are very low. The point measured at 17.42 MeV is higher than the previous experimental data. \textcolor{red}{should not trust those 3 points? Threshold region is weird}. The measured points at 19.92 MeV and above are in agreement with the previous experimental data. For the measured points above 24.28 MeV, there is excellent compliance with the CoH reaction model and the previous measured datapoints, and the shape is also excellent. TALYS and TENDL follows the same shape, but underestimates the magnitude of the excitation function. ALICE suggests a higher threshold for the reaction, and that the compound peak is shifted towards higher energies. The magnitude also seems to be underestimated. \textcolor{red}{EMPIRE}...

\subsubsection{\centering{\small{$^{56}$Mn (cumulative)}}}
???? 

\subsubsection{\centering{\small{$^{59}$Fe (cumulative)}}}
The cumulative measurement of $^{59}$Fe ($t_{1/2}$=44.490 d) \cite{Basunia2018} in table ... and figure .... It is part of a decay chain in which $^{59}$Fe is the first observed element. It decay by $\beta^-$-decay (100\%) to stable $^{59}$Co. The end of beam activities were estimated using the gamma-line in table \ref{tab:Products_Ni}, and fitted to a onestep decay. $^{59}$Fe can be produced via $^{58}$Ni(d,3p) with a reaction Q-value -12.3596 MeV, $^{61}$Ni(d,2$\alpha$) with reaction Q-value -20.3596 MeV, $^{62}$Ni(d,p$\alpha$) with reaction Q-value  -2.6597 MeV and via $^{64}$Ni(d,2np$\alpha$) with reaction Q-value -19.1549 MeV. The threshold for the reaction is constrained by the Q-value of the two former reactions, and the alpha barrier for the third one. The first measured point is at 22.19 MeV, and the cross section is on the order of $10^{-1}$ mb. From the reaction models, the suggested threshold is at ca. 14 MeV, which is the energy of the alpha-barrier. The reason that this nucleus was not observed at lower energies is due to the low activity and the long half-life. There was no activity registered at 24.28 and 29.08 MeV, but the cross section values of 22.19, 26.74 and 31.29 MeV is in great agreement with the CoH reaction model. \\
\noindent 
Alice overstimates the the compound peak, and oscillates. TALYS and TENDL follows the threshold well, but overestimates the cross section for higher energies. \textcolor{red}{EMPIRE---}

\subsubsection{\centering{\small{$^{55}$Co (cumulative)}}}
The cumulative measurement of $^{55}$Co ($t_{1/2}$=17.53 h \cite{Junde2008})  can be seen in figure ... and table ... It is part of a decay chain and this is the first element observed. $^{55}$Co decays by $\epsilon$ (100\%) to long-lived $^{55}$Fe ($t_{1/2}$=2.744 y), which was not observed. The gamma-lines used in this identification are listed in table \ref{tab:Products_Ni}. The most intense gamma-lines for this radionuclide were 477.2 keV (20-2\%), 931.1 keV (75\%) and 1408.5 keV (16.9\%). However, due to background contamination in the former and latter, only the 931.1 gamma-line was used, along with other relatively weakly fed gamma-lines. The end of beam activity was estimated using one-step decay. $^{55}$Co can be produced via $^{58}$Ni(d,n$\alpha$) with reaction Q-value -3.6 MeV, $^{60}$Ni(d,3n$\alpha$) with reaction Q-value -23.9 MeV and via $^{61}$Ni(d,4n$\alpha$) with reaction Q-value 31.8 MeV. \textcolor{red}{The first observed value for $^{55}$Co is at 7.39 MeV (foil 10). Peaks are no background. However, the threshold for the experimental data are also below alpha barrier. Is the tunneling more probable for this reaction? Because other reactions are not possible..}\\
\noindent 
The data points from this work agrees very well with previous experimental data \cite{Ochiai2007, Avrigeanu2016, Usman2016, Hermanne2013, Takacs2007, Takacs1997, Zweit1991, Amjed2013}. CoH overestimates the cross section, but does a good job estimating the shape. ALICE is slighlt lower but does a pretty good job magnitude wise. TALYS and TENDL does not do a very good job, and does not estimate a distinct compound peak. 


\subsubsection{\centering{\small{$^{56}$Co (independent and cumulative)}}}
The cumulative cross section with feeding from $^{56}$Ni and independent cross section measurement of $^{56}$Co ($t_{1/2}$=77.236 d)n\cite{Junde2011} can be seen in figures ,,, respectively and in table .... decays by $\epsilon$ (100\%) to stable $^{56}$Fe. The gamma-lines used to estimate this radionuclide is listed in table \ref{tab:Products_Ni}, and the activity was estimated using a two-step decay, estimating the independent measurement of 56Co. Since the 56Ni reaction has a high threshold above 25 MeV, it was only feeding from the first 3 foils, so the three first was fitted using a two step decay and the latter were used single decay. To achieve the cumulative cross section, the 56Ni and 56Co cross sections were added. The cumulative and independent cross section looks very much the same, and looking at the cumulative, the feeding from 56Ni is in general very low due to the low cross section in comparison to 56Co (can be seen in table .... ). 

This nuclide can thus be produced from 56Ni feeding or directly via $^{58}$Ni(d,$\alpha$) (Q = 6.5 MeV), via $^{60}$Ni(d,2n$\alpha$) (Q=-13.9 MeV), via $^{61}$Ni(d,3n$\alpha$) (Q=-21.7 MeV) or via $^{62}$Ni(d,4n$\alpha$ (Q=-32.3 MeV). In addition, $^{58}$Ni(d,2n2p) is possible with reaction Q-value -21.8 MeV. The first measured point is at 7.39 MeV which is equal to foil 10. experimental data have been shown before. \textcolor{red}{WHY IS IT POSSIBLE? tunneling, other stuff, spin??? }. The first compound peak is due to the tunneling prob of the first reaction, and the opening of the second. The compound peak in this work suggests higher peak slightly shifted to higher energies. The second increase is due to other reaction channels opening up. There is a large coverage on experimental data for the cumulative reaction \cite{Ochiai2007, Avrigeanu2016, Takacs2007, Takacs1997, Zweit1991, Usman2016, Amjed2013, Hermanne2013} (some of them are stated cumulative, but some are also just assumed cumulative). ALICE does a terrible job, And the experimental data seems to be somewhere inbetween estimation done by TENDL and CoH. Both TALYS, TENDL and CoH does a good job for cross sections from 15-20 MeV, but CoH does the best job. \textcolor{red}{EMPIRE...}. For the independent reaction, there is no explicit experimental data. The shape of the reaction models follows the data and the cumulative shape from other plot well. ALICE does a good job here. TENDL does the best job for low energy points while CoH, does the best job. All models follows the shape very well, but ALICE underestimates. 



\subsubsection{\centering{\small{$^{57}$Co (independent and cumulative)}}}
The independent and cumulative cross section for $^{57}$Co ($t_{1/2}$=271.74 d) is reported in this work. $^{57}$Co decays by $\epsilon$ (100\%) to stable $^{57}$Fe, and is subject to feeding from $^{57}$Ni ($t_{1/2}$=35.60 h) \cite{Bhat1998}. Can be seen in figure ... and in table ... The two strong gamma-lines 122.06065 (85.60\%) and 136.47365 (10.68\%) were used in the analysis, and was not subject to feeding from other nuclei. 122.06065 was however present in some background spectra in cave 4C, but with count rates in the order of $10^{-3}$ it was ignored as it did not impact the result. With the feeding from $^{56}$Ni, the end of beam activity was estimated using twostep decay, with the $A_0$ of $^{57}$Co being the optimizing parameter. This nuclide can be produced via $^{58}$Ni(d,n2p) (Q=-10.4 MeV), $^{60}$Ni(d,n$\alpha$) (Q=-2.5 MeV), $^{61}$Ni(d,2n$\alpha$) (Q=-10.3 MeV) and via $^{62}$Ni(d,3n$\alpha$) (Q=-20.9 MeV). The shape of the independent excitation function predicts that the threshold opens about 7 MeV, which is where the first measured point in this work predicts, with cross section 1.29 mb, and the excitation function increases once the two reaction channels from $^{58}$Ni and $^{61}$Ni opens. The compound peak also increases once the reaction channel from $^{62}$Ni opens at ca. 20 MeV. There is no previous experimental data for the independent measurement. TENDL, and TALYS predicts the magnitude quite well, but the peak seems to be shifted slightly towards lower energies, also counts for ALICE, which is too low in magnitude. Measured datapoints inbetween ALICE and TALYS, but TALYS closest magnitude wise!\\

\noindent 
The cumulative cross section is in well agreement with previos experimental data for lower energies, at higher energies the previos experimental data scatters. The data from this work lays in the middle of the data \cite{Usman2016, Hermanne2013, Amjed2013, Avrigeanu2016, Takacs1997, Zweit1991, Takacs2007, Ochiai2007}. \textcolor{red}{comment on curves}



\subsubsection{\centering{\small{$^{58m+g}$Co (independent and cumulative)}}}
The independent cross section of $^{58m}$Co and $^{58g}$Co are reported, along with the cumulative cross section of both. The isomer ($t_{1/2}$=9.10 h) decay by IT (100\%) to the ground state ($t_{1/2}$=70.86 d), which decays to stable $^{58}$Fe \cite{Nesaraja2010a}. The cross sections can be seen in figures .. and table .. The gamma-lines used to identify the ground state are listed in table \ref{tab:Products_Ni}, with the most intense gamma-line at 810.7593 keV (99.450\%) not being subject to background of feeding from other nuclei. For the isomer, there was no observed gamma-lines (with the only and most intense one being 24.889 keV (0.0397\%)), so the activity for both of them was estimated using a two-step decay with the end of beam activities of the both serving as optimizing parameters. The groundstate can be produced via $^{58}$Ni(d,2n) (Q=-1.8 MeV), $^{60}$Ni(d,$\alpha$) (Q=6.1 MeV), $^{61}$Ni(d,n$\alpha$) (Q=-1.7 MeV), $^{62}$Ni(d,2n$\alpha$) (Q=-12.3 MeV) and via $^{64}$Ni(d,4n$\alpha$) (Q=28.8 MeV), and the isomer can be produced equally, with the Q-value being the energy level higher (can be seen in table \ref{tab:Products_Ni}). \\

\textcolor{red}{what about the threshold, ift barrier? }

\noindent
The independent measurement of the isomer (figure ..) does not quite agree with the previous experimental data from Avrigeanu et. al. (2016) \cite{Avrigeanu2016}. Avrigeanu et. al. used the same method, by minimization of the Bateman equtaion, along with a direct measurement of the 24.9 keV gamma-line, which was consistent within 10\% (\textcolor{red}{correctly written?}). One of the reason that it is hard to say is that a twostep decay with two optimizing parameters bring uncertainties to the second optimizing parameter, as they are correlated. The previous experimental data agrees well with TENDL and TALLYS, while this work agrees more with CoH, magnitude wise. The shape of the excitation function seems to be right for ALICE, TALYS and TENDL, but for CoH the compound peak seems to be shifted towards lower energies.  \\

\noindent 
The independent measurement of the ground state (figure ...), does not have any reported independent measurements (the work done by Avruígeanu et. al. (2016) posted independent of the isomer but cumulative ground state). TALYS and TENDL agrees very well with the measurements in this work. CoH is too high magnitude wise, but predicts the shape very well. ALICE is too low magnituse wise and the compound peak is shifted towards higher energies. \\

\noindent The cumulative cross section of the ground state can be seen in figure ..., qith the two cross sections summed. This data agrees well with the experimental data from Zweit (1991), Amjed (2013) and Avrigeanu (2016) \cite{Zweit1991, Avrigeanu2016, Amjed2013}. The work done by Takacs (1997) \cite{Takacs1997}, are systematically higher for higher energies, in which the explanation that the nuclear activity was measured before the complete decay of the isomer was represented in the article. The work done by Takacs et al. however matches well with TENDL and TALYS magnitudewise. CoH does quite a good job in predicting the shape and magnitude of the excitation function. ALICE and TENDL are too high magnitude wise and seem to follow the shape of the independent ground state measurements. ALICE does not do a good job. \textcolor{red}{EMPIRE?}


\subsubsection{\centering{\small{$^{60}$Co (cumulative)}}}
The cumulative measurement for $60m$Co ($t_{1/2}$=10.467 m) and $^{60g}$Co ($t_{1/2}$=1925.28 d) \cite{Browne2013} is reported in figure ... and table .... The isomer decays by IT (99.75\%) and by $\beta^-$ (0.25\%). $^{60}$Co is the first observed element in a decay chain, and decay by $\beta^-$ (100\%) to stable $^{60}$Ni. Due to the long half-life of the ground state, long counts was required to have a low statistical error. Due to the short half-life of the isomer, independent measurements of the isomer and ground state were not possible to extract. $^{60}$Co is present in the background, and it was necessary to use background subtraction.  
\textcolor{red}{perhaps should include 1173 keV even though it is fed by 55Co. Then only use counts registered after 170 hours?}. with  The activity was estimated using a one-step decay. $^{60}$Co can be produced via $^{60}$Ni(d,2p) with reaction Q-value -4.3 MeV, $^{61}$Ni(d,n2p) with reaction Q-value -12.1 MeV, via $^{61}$Ni(d,$\alpha$) with reaction Q-value 5.6 MeV, and via $^{64}$Ni(d,2n$\alpha$) with reaction Q-value -10.9 MeV. The first observed cross section was at 11.41 MeV, and within uncertainty, this can be when the $^{61}$Ni(d,n2p) reaction channel opens, combined with $^{60}$Ni(d,2p) reaction channel which opens at ca. 4 MeV. The excitation function steadily increases as the alpha-barrier is overcome, and all the reaction channels open. 

TENDL, TALYS and CoH suggests a compound peak with a maximum at ca. 21 MeV. The experimental data does no agree clearly with this, and this work suggests that the compound peak should be shifted towards higher energies. The previous experimental data \cite{Avrigeanu2016, Usman2016, Hermanne2013, Takacs2007} compared to this work is inconsistent. The work done by Takacs et. al. (2007) was based upon the expectation that the excitation function would increase above 25 MeV, and increase steadily up to 80 mb at 50 MeV. The  estimated cross sections (above 25 MeV?) were estimated using a spline fit, and thus the values are higher than expected??. The data in this work seems to agree well with the work done by Usman et. al. (2016) and Avrigeneau et. al. (2016). Due to the long half-life of this nuclide, it was difficult to have statistically enough number of counts.   

\subsubsection{\centering{\small{$^{56}$Ni (cumulative)}}}
The cumulative measurement of $^{56}$Ni (6.075 d) decays by $\epsilon$ (100\%) to $^{56}$Co ($t_{1/2}$=77.236 d). $^{56}$Ni is the first observed element of a decay chain with the parent being $^{56}$Cu ($t_{1/2}$=93 ms)\cite{Junde2011}, and is therefor reported as cumulative. The gamma-lines used in the estimation of end of beam activity is listed in table \ref{tab:Products_Ni}, with 158.38 keV (98.8\%) being the strongest gamma-line. All the gamma-lines were used with the exception of 269.50 (36.5\%) which was matched with background radiation in cave 4C. The end of beam activity was estimated using onestep decay. The nuclide can be produced via $^{58}$Ni(d,3np) (Q=-24.7 MeV). From figure ..., it is clear that the threshold is at ca. 25 MeV, with the first measured point in this data being at 26.74 MeV, with cross section 0.06 mb.  \\

\noindent 
The measured points in this work agrees well with other experimental datasets around threshold \cite{Hermanne2013, Amjed2013, Takacs2007}, for the three measured points in this work. CoH does a very good job with the prediction of the excitation function at this low cross section values. TENDL and ALICE also does a good job shapewise, but assumes slightly too high and lower threshold. ALICE predicts a  way to rapid increase in the excitation function, it is clear that it is having trouble estimating the excitation function at this low cross sections. \textcolor{red}{EMPIRE}

\subsubsection{\centering{\small{$^{57}$Ni (cumulative)}}}
The cumulative measurement of $^{57}$Ni ($t_{1/2}$=35.60 h), and decays by $\beta^+$ (100\%) to $^{57}$Co ($t_{1/2}$=271.74 d). $^{57}$Ni is the first observed element in a decay chain with $^{57}$Cu ($t_{1/2}$=196.3 ms)\cite{Bhat1998} and is thus reported as cumulative. The gammalines which were used are listed in table \ref{tab:Products_Ni}. The most intense gamma-line 1377.63 keV (81.7\%) was excluded since the peak was contaminated with background radiation in cave 4C (in the order of $10^{-2}$) which gave unrealistic measurements ca. 20 half-lives since end of beam. The other lines were sufficient to have a good measurement. This nuclide can be produced directly via $^{58}$Ni(d,2np) (Q=-14.4 MeV) and via $^{60}$Ni(d,3np) (Q=-34.5). In addition, the channel $^{58}$Ni(d,t) (Q=6.0 MeV) seems to be fed, as the reaction cross section threshold seems to be at around 10 MeV, if we trust previous experimental data. As the $^{58}$Ni(d,2np) reaction channel opens, the cross section increases, and reaches more when ... Cant explain shape here ... The feeding from $^{57}$Cu has reaction Q-value of 24.0 MeV, and this might contribute to the compound peak being larger.  \\ 

\noindent 
The data provided from this work is in well agreement with previous experimental data in the low energy region, but is too high for foil 1 and 2\cite{Hermanne2013, Takacs2007, Avrigeanu2016, Takacs1997, Zweit1991}. ALICE overestimates the data, while ALICE and TENDL does a very good job with the experimental data. The CoH excitation function does a good job but sugests a more narrow peak and decreases before experimental data does. 

%Background lines: 1377.63 (44K), 1730.44 (bc: 1729.6). 
%Shared? 1046.68 (52Mn, 1045.73 keV, 0.07\%), 1730.44 (61Cu, 1729.473, 0.054\%), 1919.52 (60Cu, 1919.7, 0.70\%). Question, remove? """

\subsubsection{\centering{\small{$^{65}$Ni (independent)}}}
The independent measurement of $^{65}$Ni ($t_{1/2}$=2.51719 h) is reported, which decay by $\beta^-$ (100\%) to stable $^{65}$Cu \cite{Browne2010a}. The gamma-lines used to estimate the end of beam activity are listed in table \ref{tab:Products_Ni} and are not subject to background radiation, nor feeding from other nuclei.  The activity was estimated using a onestep decay. This product can only be produced via $^{64}$Ni(d,p) (Q=3873.5 MeV). With the proton barrier for $^{66^*}$Cu being ca. 6.6 MeV, the threshold at 3 MeV makes sense? The reaction cross section is quite low, which is directly affected by the low abundance of $^{64}$Ni as natural nickel target. \\

\noindent Comparing to previous experimental data by Avrigeanu et. al. (2016) \cite{Avrigeanu2016}, the results obtained in this work is is in quite good agreement, especially around the maximum of the peak. However, uncertainties are a little hihger, and the high energy side of the tail seems to have moved a little to higher energies \textcolor{red}{why is this?}. From the reaction models, it is clear that TALYS \& TENDL and CoH \&ALICE uses different methods to estimate the peak. TENDL and TALYS get the shape best. \textcolor{red}{empire...}. 





\subsubsection{\centering{\small{$^{60}$Cu (independent)}}}
The independent measurement of $^{60}$Cu ($t_{1/2}$=23.7 m) decays by $\epsilon$ (100\%) \cite{Browne2013}. The gamma-lines listed in table \ref{tab:Products_Ni} are not subject to background feeding, nor feeding from other nuclei, with the exception of 1792.1 keV (0.082\%) from $^{55}$Co. Since the gamma-line is weak and the half-life is much longer, this was ignored. The end of beam activity was estimated using onestep decay. The ways inwhich this nuclide can be produced is via $^{60}Ni(d,2n)$ (Q=-9.1 MeV), $^{61}$Ni(d,3n) (Q=-17.0 MeV) or via $^{62}$Ni(d,4n) (Q=-27.6 MeV). It is clear that the threshold opens at ca. 10 MeV, and that the compound peak is fed by the other possible reaction channels as the energy increases.  

The datapoints in this work is in agreement with previous experimental data for Avrigeanu et. al. (2016), Ochiai et. al. (2006), and Takacs et. al. (1997) for low energies up to ca. 17 MeV \cite{Avrigeanu2016,Takacs1997,Ochiai2007}. A preposition for why these measurements are higher is not given in the article. ALICE seems to get the shape of excitation function good, while TENDL, TALYS and CoH does a good job magnitude wise. \textcolor{red}{EMPIRE...}

%1791 can be shared with 55Co, (1792.1, I=0.082, t=17.53 h)

\subsubsection{\centering{\small{$^{61}$Cu (independent)}}}
The independent measurement of the monitor reaction alone can be seen in figure ...., 

\subsubsection{\centering{\small{$^{64}$Cu (independent)}}}
The independent measurement of $^{64}$Cu ($t_{1/2}$=12.701 h) is reported, and  decays by $\epsilon$ (61.5\%) to stable $^{64}$Ni or by $\beta^-$ (38.5\%) to stable $^{64}$Zn \cite{Singh2007a}. The decay lacks gamma-lines, and from the gamma-decay of $^{64}$Ni, there is one gamma-line at 1345.77 MeV (0.475\%), which is not subject to background feeding, or feeding from other nuclei. This nuclide can only be produced via $^{64}$Ni(d,2n) (Q=-4.7 MeV), and the excitation function starts to increase around 5 MeV.\\

\noindent 
In comparison to previous experimental data \cite{Zweit1991,Avrigeanu2016,Takacs2007}, the uncertainties in the points measured in this work is much larger, but it follows a good shape. \textcolor{red}{why? is it because the counts were not counted for a sufficient time to have good statistcs, since the gamma-line is weak? }. The intensity is quite low, and the peak was only observed once for each measurement, and never in multiple foils. Should have been counted over longer time periods. The uncertainty is directly related to the uncertainty in the number of counts (foil 10: 47.8\%, foil 9: 13.1\%, foil 8: 48.1\%, foil 6: 27.9\%). So this is very clear. Even though particularly the measurement at 14.63 MeV is high, all points are still within uncertainties the same as previous data. ALICE follows the shape, of the experimental data, but all reaction models follows the data very well in both magnitude and shape. \textcolor{red}{EMPIRE...}  


\subsection{Copper products}
Regarding the products produced from copper, the number of products are less, since there is only two stable target nuclei $^{63}$Cu (69.15\%) and $^{65}$Cu (30.85\%), and the nickel products (d,xnp) have many stable isotopes. For this target, independent measurements of $^{65}$Ni and $^{64}$Cu, and cumulative measurements of $^{59}$Fe, $^{60}$Co, $^{61}$Co, $^{61}$Cu. In addition, the independent measurements of the monitor reactions $^{62,63,65}$Zn were made, which is described above. In addition, first measurement of $^{61}$Co.  \\
The Coulomb barrier for protons are ca. 6.8 MeV and for alpha ca. 14.4 MeV for the compound nucleus of $^{65^*/67^*}$Zn. \\

What did we expect?? 



\subsubsection{\centering{\small{$^{59}$Fe (cumulative)}}}
The cumulative measurement of $^{59}$Fe ($t_{1/2}$=44.490 d) can be seen in figure ... and in table.. The nucleus decays by $\beta^-$ (100\%) to stable $^{59}$Co \cite{Basunia2018}. The gamma-lines used in the analysis is listed in table \ref{tab:Products_Cu}, the two most intense gamma-lines 1099.245 (56.5\%) and 1291.590 (43.2\%) were used, which are not subject to background radiation, 1099.245 can be subject to feeding from $^{61}$Cu but with a large difference in half-life and intensity of the gamma-line, it is ignored.  1291.590 is not subject to feeding. The activity was estimated using a onestep decay. This nucleus is the first observed element in a decay chain and is therefor cumulative, with the possible feeding from $^{59}$Mn ($t_{1/2}$=4.59 s, which can contribute via reaction $^{65}$Cu(d,2pd$\alpha$) (Q=-28.7 MeV), as the cross section increases around 30 MeV. $^{59}$Fe can be produced directly via $^{63}$Zn(d,2p$\alpha$) (Q=-8.8 MeV) or via $^{65}$Cu(d,2$\alpha$) (Q=1.7 MeV). From figure ..., it is clear that the reaction channel opens at ca. 14 MeV, and the first measured point in this work is at 18 MeV (0.03 mb). 

In comparison to previous data \cite{Khandaker2014, Takacs2006a}, the data from this work matches well well with data from Khandaker et. al. (2014), and for Tacaks et. al. (2006), the values are lower, but mostly within uncertainty. There is one measured point from Tacaks (2016) at ca. 5 MeV, which is not commented in the paper. In comparison to reaction models, CoH does a very good job magnitude and shape wise, TENDL and TALYS does a good job shape wise, and ALICE is off, predicting a large compound peak at 25 MeV. 

\subsubsection{\centering{\small{$^{60}$Co (cumulative)}}}
The cumulative cross section of the isomer ($t_{1/2}$=10.467 m, IT: 99.75\%) and groundstate of $^{60}$Co ($t_{1/2}$=1925.28 d) can be seen in figure... and in table.... The ground state decays by $\beta^-$ to stable $^{60}$Ni \cite{Browne2013}. The gamma-lines used in this analysis is listed in table \ref{tab:Products_Cu}, with 1173.228 keV (99.85\%) and 1332.492 (99.9826\%). $^{60}$Co is present in the background, and it was necessary to use background subtraction. The cross section is reported as cumulative as there is possible feeding from the very long lived $^{60}$Fe ($t_{1/2}=2.62\cdot10^6$ y), but this was not seen. $^{60}$Co can  be produced via $^{63}$Cu(d,p$\alpha$) (Q=-0.5 MeV) and via $^{65}$Cu(d,2np$\alpha$) (Q=-18.3 MeV). On figure ... it is clear that the excitation function opens at ca. 10 MeV, and the first  measured point in this work is at 11.41 MeV. The cross section increases to a compound peak when the next reaction channel opens.  \\ 

In comparison to experimental data \cite{Takacs2006a, Khandaker2014}, this data aligns well with previous experimental data. As usual ALICE does a good job in predicting the compound peak shape wise. CoH is on the high side, but does better at higher energies, and TENDL does a very good job magnitude wise. TALYS is a little too low in magnitude. 

\subsubsection{\centering{\small{$^{61}$Co (cumulative)}}}
The cumulative cross section of $^{61}$Co ($t_{1/2}$=1.649 h) can be seen in figure.. and table .. It decays by $\beta^-$ (100\%) to $^{61}$Ni\cite{Zuber2015}. The gamma-line used in this analysis is listed in table \ref{tab:Products_Cu}, which is not subject to background feeding, however, the shared gamma-line with $^{61}$Cu (which also $\epsilon$-decay into $^{61}$Ni) with intensity 4.23\% can contribute here. All points measured are above $^{61}$Cu threshold, so there might be some feeding here??? This cross section is reported as cumulative since there is possible feeding from short-lived $^{61}$Fe ($t_{1/2}$=5.98 m), which can be produced via $^{63}$Cu(d,4p) (Q=-22.7 MeV) and via $^{65}$Cu(d,2n$\alpha$) (Q=-12-2 MeV). $^{61}$Co can also be produced directly via $^{63}$Cu(d,n3p) (Q=-19.5 MeV) or via $^{65}$Cu(d,np$\alpha$) (Q=9.0 MeV). Dont see any evidence from the alpha-channels from either, as the reaction threshold is at ca. 20 MeV. Cross section increases rapidly after, due to both feeding and measurement? \\ 

No previous experimental data! Reaction models predicts well. While CoH and TENDL does a good job shapewise, in this energy region TALYS is closest in magintude, but seem to increase to slowly. 

\subsubsection{\centering{\small{$^{65}$Ni (independent)}}}
The independent measurement of $^{65}$Ni ($t_{1/2}$=2.51719 h) can be seen in figure .... and in table .... The gamma-line used in the analysis can be seen in figure \ref{tab:Products_Cu}, and it not subject to background. $^{59}$Fe can feed into this (1481.70, 0.059\%) but since both cross section and intensity is low this was ignored. This product can only be produced via $^{65}$Cu(d,2p) (Q=-3.6 MeV). From the figure, the threshold is suggested at ca. 10 MeV, but the first measured point here is at 18.14 MeV. The cross section is very low though. The uncertainty in this data is mainly caused by large uncertainty in efficiency (ASK ANDREW!!) and in the number of counts with relative uncertainty 22\% and 26\%. But as clearly can be seen the values are way to high in comparison to both experimental data \cite{Takacs2006a} and in comparison to other reaction channel codes. These were also counted short time after end of beam. All reaction models and experimental data agrees quite well. 

\subsubsection{\centering{\small{$^{61}$Cu (cumulative)}}}
The cumulative cross section of $^{61}$Cu ($t_{1/2}$=3.339 h) \cite{Zuber2015} can be seen in figure ... and table .... It decays by $\epsilon$ (100\%) to stable $^{61}$Ni. Three gamma-lines were used and are listed in table \ref{tab:Products_Cu}, are not contaminated by background radiation, nor by other nuclei. The cross section is reported as cumulative with the possibility of feeding from $^{61}$Zn ($t_{1/2}$=89.1 s, $\epsilon$ (100\%)) inwhich can be produced via $^{63}$Zn(d,4n) (Q=28.4 MeV). $^{61}$Cu can be produced directly via $^{63}$Zn(d,3n) (Q=-15.5 MeV) or via $^{65}$Cu(d,5n) (Q=-33.3 MeV). The reaction channel opens at ca. 20 MeV, and it increases steadily, with the contribution from $^{61}$Zn feeding from ca. 30 MeV, and the latter direct reaction from ca. 34 MeV. 

The measured datapoints are right around threshold, and the first observed point is at 25.32 MeV (1.18 mb). It aligns well with previous experimental data \cite{Nakao2006,Diksic1979}, and CoH is doing an excellent job in predicting the excitation function. ALICE, TALYS and TENDL also do a very good job! 


\subsubsection{\centering{\small{$^{64}$Cu (independent)}}}
The independent measurement of $^{64}$Cu ($t_{1/2}$=12.701 h) can be seen in figure.... an table .... It decays by $\epsilon$ (61.5\%) to stable $^{64}$Ni or $\beta^-$ (38.5\%) to stable $^{64}$Zn \cite{Browne2010}. The gamma-line which was used is not subject to background radiation and is not shared with other nuclei. However, the intensity is weak, but the channel is heavily fed and the cross section is high which can be seen on figure.. This product can be produced via $^{63}$Zn(d,p) (Q=5.7 MeV) or via $^{65}$Zn(d,2np) (Q=-12.1 MeV). The reaction threshold is at ca. 2.4 MeV according to reaction modeling, and the first measured point in this work is at 3.94 MeV (73.54 mb). It is clear that the first reaction channel opens early,which contributes to the first compound peak, and then the other opens.. \\ 

The experimental data \cite{Simeckova2011, Bartell1950,Fulmer1970,Takacs2006a, Diksic1979,Khandaker2014,Bartell1950,Ochiai2007} aligns very well with this data. ALICE and CoH does a different approximation in comparison to TENDL and TALYS. TENDL and TALYS does a good job in predicting the compound peak, but should have been slightly broader. CoH and ALICE does not directly predict the compound peak, assumes no drop in the excitation function. 

\subsection{Iron products}
Alpha barrier ca. 13.5 MeV, and proton barrier ca. 6.4 MeV.


\subsubsection{\centering{\small{$^{48}$V (cumulative)}}}
The cumulative cross section of $^{48}$V ($t_{1/2}$=15.9735 d) can be seen in figure.... and in table.... $^{48}$V decays via electron capture/beta+ decay $\epsilon$ (100\%) to stable $^{48}$Ti. Because of the possible feeding from $^{48}$Cr ($t_{1/2}$=21.56 h, $\epsilon$ 100\%) the cross section is reported as cumulative. The reaction channel for $^{48}$Cr opens at 25.7 MeV (via $^{54}$Fe(d,nt$\alpha$) and since the chances of triton emission is low, the expected product is low. $^{48}$V has two strong gamma-lines 983.525 keV (99.98\%) and 1312.106 keV (98.2\%), but are both present in the background, but with a countrate in the order of $10^{-3}$. \textcolor{red}{the activity looks to high.. maybe background subtraction? too many false peaks i think:(}.
9

\subsubsection{\centering{\small{$^{51}$Cr (cumulative)}}}
\hspace{6cm}
\subsubsection{\centering{\small{$^{52}$Mn(cumulative)}}}
\hspace{6cm}
\subsubsection{\centering{\small{$^{54}$Mn (independent)}}}
\hspace{6cm}
\subsubsection{\centering{\small{$^{53}$Fe (cumulative)}}}
\hspace{6cm}
\subsubsection{\centering{\small{$^{59}$Fe (independent)}}}
\hspace{6cm}
\subsubsection{\centering{\small{$^{55}$Co (independent)}}}
\hspace{6cm}
\subsubsection{\centering{\small{$^{57}$Co (independent)}}}
\hspace{6cm}
\subsubsection{\centering{\small{$^{58}$Co (cumulative)}}}
can have isomer and groundstate \hspace{6cm}


\subsection{Iridium Products}
$\alpha$-barrier: ca 25 MeV, proton-barrier: ca 13.7 MeV

For a 33 MeV deuteron beam on the target stack, cross sections for $^{188}$Ir, $^{189}$Ir,  $^{190m1+g}$Ir, $^{190m2}$Ir, $^{192}$Ir, $^{194m1+g}$Ir, $^{194m2}$Ir, $^{188}$Pt, $^{189}$Pt, $^{191}$Pt and $^{193m}$Pt was measured, which described below. There was however no evidence that anything which would emit more than one proton in the decay of the compound nucleus, like Os, Re, W or Ta was produced. Identifying the Iridium products was a difficult task. Firstly, as the Coulomb-barrier \textcolor{red}{equation...} for these reactions is higher than for Ni, Cu and Fe as target nuclei, since the number of protons is higher. Thus we known that reactions like (d,$\alpha$), or (d,xp) would not be heavily fed. However, most of the energetic thresholds are less than 33 MeV, so even though they are weakly fed, they will still be fed if the channel is possible. This we did not see any evidence of. Secondly, many of the nuclei have gammas that are so close in energy that the germanium detectors would identify the peak as one. As the reaction routes producing Ir- and Pt-radionuclides is heavily favoured, the peaks with feeding from multiple nuclei would still mostly be caused by those nuclides. The question is then, was anything else than Ir and Pt radionuclides produced, or were they produced but could not be distinguished because of the peaks. As a summary, a few of the nuclides were excluded, but a few of the nuclides is questionable. It can also be mentioned that the compound nuclei $^{193}$Pt and $^{195}$Pt are nuclei with even Z and odd N, thus proton emission would be more likely, as discussed in \textcolor{red}{section ....}.  \\

\noindent 
Production routes via $\alpha$ emission is energetically possible.  $^{191}$Ir(d,$\alpha$)$^{189}$Os (stable) with reaction Q-value=12988.8 keV could not be observed since it is a stable product, and $^{193}$(d,$\alpha$)$^{191}$Os (ground state $t_{1/2}$=15.4 d, isomer $t_{1/2}$=13.10 h) with reaction Q-value 12569.8 keV, was not observed. $^{191}$Os was expected to be produced, as the cumulative cross sections within the deuteron energy region have been observed \textcolor{red}{cite tarkanyi 2019}. However measurements done in that work was small cross sections. The main hypothesis for $^{191}$Os not to be observed is that the only gamma-line of the ground state which was intense enough to be observed was 129.431 keV (26.50\%) which was too close to $^{191}$Pt ($t_{1/2}$=2.83 d) in energy. Thus the way to measure the $^{191}$Os cross section was to count long enough for $^{191}$Pt to decay completely, which would be for ten half lives which wold be approximately 1500 hours. The isomer also has one single gamma-line at 74.38 keV (0.0729\%), which was not observed due to low intensity, and multiple nuclei with similar gamma-line. $^{191}$Ir(d,2$\alpha$)$^{185}$W ($t_{1/2}$=75.1 d) with reaction Q-value 14964.9 keV was not observed, nor was $^{193}$Ir(d,2$\alpha$)$^{187}$W ($t_{1/2}$=24.0 h) with reaction Q-value 13653.6 keV. Thus it was concluded that W and below in mass number was produced at all. \\

\noindent 
Another reaction route which was expected to be observed was $^{193}$Ir(d,2p)$^{193}$Os ($t_{1/2}$=30.11 h) with reaction Q-value -2584.0 keV. Gamma-lines matched well with the spectra, and unique gamma-lines 321.616 keV (1.245\%) and 387.509 keV(1.226\%) (which was not in the background) was used to estimate the end of beam activity, the cross sections did not look reasonable. Re-radionuclides had similar gamma-lines to Pt and Ir-radionuclides, so we could not distinguish. No observation of Re-radionuclides were made, but a question-mark remain on this behalf, especially since $^{190}$Re and $^{192}$Re is even (odd-odd).  




%Compound nucleus $^{58}$Ni(d)$^{60*}$Cu is an even (odd-odd) nucleus with 29 protons and 31 neutrons. Proton and neutron emission is hence probable, for the nucleons to be paired and give a net spin as small as possible. $^{60}$Ni(d)$^{62*}$Cu is also an even (odd-odd) nucleus, with 29 protons and 33 neutrons, thus the same condition accounts for this one. $^{61}$Ni(d)$^{63*}$Cu is and odd nucleus with 29 protons and 34 neutrons. Hence, 

\subsubsection{\centering{\small{$^{188}$Ir (Cumulative and independent)}}}
$^{188}$Ir ($t_{1/2}$=41.5 h) is a radionuclide with one metastable isomer (with a half-life in the order of ms) along with the ground state, which decays by either isomeric transition or $\epsilon/\beta^+$, but the branching ratio is not stated. The reported cross sections for $^{188}$Ir, regardless of whether it is reported as cumulative or independent is the cumulative cross section of the ground state and isomer. Since an independent measurement of $^{188}$Pt ($t_{1/2}$=10.2 d), which feeds into $^{188}$Ir (100\%) was obtained, the independent measurement without feeding and the cumulative cross section with feeding is reported. The reaction threshold is $^{191}$Ir(d,2nt) with a reaction Q-value of -16231.0 keV. In this work, we did not see any evidence of this decay-route being fed. Thus, $^{188}$Ir can be produced via $^{191}$Ir(d,4np) with a reaction Q-value -24802.0 keV and an energy threshold of 25064.0 keV. From $^{193}$Ir as target nuclei, the threshold for 6np-particle emission is above energy threshold, so the only available route from this target nucleus is with emission of 4nt (Q-value is -30291.0 keV), which is not heavily fed either. It is clear that the excitation function first increases once the 4np-decay channel opens, but the measured points are at a low cross section value. The gamma-lines which were used are listed in table ..., The end of beam-activity was estimated using a single decay, where activity points ca. 200 hours after end of beam is slightly higher than the curve, because of the feeding. The end of beam activity however looks reasonable, and the cross section also looks reasonable. Comparing the measured cross section points to experimental data, F. Tarkanyi et. al. (2019), does not have any measured points below ca 35 MeV, but the points predicts that the excitation function increases in this region, where the measured cross sections in this work give reason to believe that the cross section increases from about 25 MeV. The independent measurement aligns well with TALYS and TENDL, but the large measurements 26 MeV is probably a little higher than then true value. \textcolor{red}{reaction models}. 

\subsubsection{\centering{\small{$^{189}$Ir (Cumulative and independent)}}}
\textcolor{red}{This is weird. Expectedly, the cumulative cross section of $^{189}$Ir should be larger than the independent measurement of $^{189}$Pt, due to $\epsilon$-decay (100\%). This is however not the case so the independent measurement of $^{189}$Ir is negative. This is also appearant in Tarkanyi et. al. (2019), so why is this?}
$^{189}$Ir ($t_{1/2}$=13.2 d) is a radionuclide with two metastable states with half-lives in order of microseconds. Thus the total cross section in general is the two isomers and the ground state. However, $^{189}$Pt ($t_{1/2}$=10.87 h) feeds into this radionuclide via $\beta^+/\epsilon$ (100\%), so both an independent measurement of the three states of $^{189}$Ir is reported along with the cumulative cross section with feeding from $^{189}$Pt. The gamma-lines which were used is listed in table ..., and the end of beam activity was estimated using single-step decay, although ideally this could have been fitted using a two-step decay with feeding from $^{189}$Pt, but due to \textcolor{red}{overestimation} in the end of beam activity, so it turned out large negative. Instead single-step decay was used where no peaks before 60 hours after end of beam was used, so that the $^{189}$Pt had decayed out.  
$^{189}$Ir can be produced via $^{191}$Ir(d,3np) with a reaction Q-value -16626.0 keV and energy threshold 16802.0 keV, or via $^{193}$Ir(d,5np) with reaction Q-value -30596.0 keV and energy threshold 30916.0 keV. This work provided four cross section measurements from 23-30 MeV, and it is clear that the cross section increases from ca. 20 MeV, comparing to experimental data from F. Tarkanyi et. al. (2006). The data provided by this work is in well compliance with the 2006 data, where the excitation function is at its highest between 30-35 MeV, when the other reaction channel has opened. However, 2019 data is shifted ca. 10 MeV higher in energy. Our data seems to agree with the 2006 version.  The 2019 data seems to be agreeing more with when the excitation function starts to increase looking at ALICE, TENLD and TALYS, where at 20-25 MeV, the function is basically 0. 



\subsubsection{\centering{\small{$^{190m1+g}$Ir (Cumulative and independent)}}}
$^{190}$Ir ($t_{1/2}$=11.78 d) decays by $\epsilon/\beta^+$-decay to stable $^{190}$Os (100\%). This radionuclide can be produced via $^{191}$Ir(d,2np) with a reaction Q-value -1769.3 keV and with an energy threshold 10359.2 keV or via $^{193}$Ir(d,4np) with a reaction Q-value -24221.2 and with an energy threshold 24474.0 keV. For this radionuclide, a total of eight cross section measurements were made in the energy region 13-30 MeV, and as expected, we did not see evidence that this radionuclide was produced for deuterons below 10 MeV. The xcitation function steadely increases from 13-20 MeV, and once the next reaction channel opens, the excitation function increases more steeply. Due to the independent measurement of $^{190m2}$Ir, an independent measurement of the ground state and the m1-isomer was made, which is given in figure ... The excitation function looks almost identical due to the weak branching from $^{190m2}$Ir (8.60\%), along with quite low measured cross sections of $^{190m2}$Ir (table...). Comment on reaction models. 

\subsubsection{\centering{\small{$^{190m2}$Ir (Independent)}}}
$^{190m2}$Ir ($t_{1/2}$=3.087 h) either decay by $\beta^+/\epsilon$-decay (91.4\%) to $^{190}$Os or isomer transition (8.6\%) to $^{190g}$Ir. \textcolor{red}{figure out Q-values and e thresholds}

\subsubsection{\centering{\small{$^{192g}$Ir (Cumulative)}}}
$^{192g}$Ir ($t_{1/2}$=78.829 d) either $\beta^-$ decays to stable $^{192}$Pt (95.24\%) or $\beta^+/\epsilon$-decays to stable $^{192}$Os (4.76\%). The radionuclide can be produced via $^{191}$Ir(d,p) with a reaction Q-value of 3973.55 keV, or via $^{193}$Ir(d,2np) with a reaction Q-value -9996.6 keV, and a energy threshold 10.1009 MeV \footnote{(https://www.nndc.bnl.gov/qcalc/qcalcr.jsp)}. Thus the reaction channel is energetically possible for 0 MeV deuterons. However, since there are two possible ways to produce this radionuclide, an increasing cross section would appear as the probability of proton emission as decay channel increases. With a proton Coulomb-barrier at approximately 13 MeV for both stable targets \footnote{(http://hyperphysics.phy-astr.gsu.edu/hbase/NucEne/coubar.html)}, the first peak in the excitation function should be somewhere around there, which we can clearly see on the figure, in the experimental data in this work. Then the cross section decreases up to about 20 MeV, and increases up to about 35 MeV, where the experimental data suggests that the cross section flattens. This measurement is a cumulative cross section with $^{192m1}$Ir ($t_{1/2}$=1.45 m) feeding in with isomer transition (99.98\%), and $^{192m2}$Ir ($t_{1/2}$=241 y) feeding in with isomer transition (100\%), where we did not make any independent measurements of either isomers due to too short and too long half life. We were able to produce and measure the cross section in each foil, and the cross section measurements done in this work is in well compliance with experimental data provided by Tarkanyi et. al. (2006, 2019), but the values are slightly higher for energies higher than 13 MeV. The gamma-lines which were used are listed in table \ref{tab:Products_information_IR}, and all the gamma-lines are neither in the background or feeding from other gammas. The end of beam activity was estimated using a single-step decay, as there is nothing beta-feeding in, and the isomers are in the cumulative cross section. \textcolor{red}{The theoretical predictions are off. Make more comments}. 

\subsubsection{\centering{\small{$^{194g}$Ir (Cumulative)}}}
The reaction Q-value for $^{193}$Ir(d,p)$^{194g}$Ir is 3842.22 keV, thus the energy threshold for the deuteron is 0 MeV. For this particular measurement, the cross section is reported as cumulative because of the $^{194m1}$Ir-isomer (E=147.0785 keV, $t_{1/2}$=31.85 ms) feeds into $^{194g}$Ir by isomer transition (100\%). $^{194g}$Ir decays by $\beta^-$ to stable $^{194}$Pt. The gamma-lines which were used to calculate the activity in the foils were 293.541 (2.5\%), 300.741 (0.35\%), 589.179 (0.140\%), 938.69 (0.60\%), 1150.75 (0.60\%), 1468.91 (0.19\%). Most of the gamma-intensities are less than 1\%, as more intense gamma-lines were either in the background or had gamma-lines which were overlapping with other nuclei. Thus the contribution to uncertainties can be of importance here. However, comparing the measured cross sections to other measured cross sections by Tarkanyi et. al. (2006, 2019), the uncertainty in cross section looks small. The end of beam activity was estimated using a single step decay, since the parent isomer $^{194m2}$Ir does not feed (decay by $\beta^-$ (100\%)), and there is no chance that we can measure the $^{194m1}$Ir isomer. The correspondence between the previous experimental data seems to be good. The measured points in this work seems to be a little on the high side on the tail, but yet within uncertainty. TENDL and TALYS does an ok job shaping the peak, but is both lower in magnitude and the maximum of the TENDL function shifted from about 13 to 10 MeV. The maximum of of the TALYS peak is in better compliance with the measured cross sections, and suggests a maximum at around 12 MeV. ALICE however is on \textcolor{red}{bærtur}

\subsubsection{\centering{\small{$^{194m2}$Ir (Independent)}}}
The reaction Q value for $^{193}$Ir(d,p)$^{194m2}$Ir ($t_{1/2}$=171 d) is 3652.22-X keV, due to the uncertain energy level of the isomer. Having a positive reaction Q-value, the deuteron energy threshold is 0 MeV. The cross section represented in this work is independent, as this isomer level is the highest one observed. $^{194m2}$Ir decays by $\beta^-$-decay to stable $^{194}$Pt. In the estimation of cross section, one single line was used, which was 390.8 keV (35\%). The more intense gamma-lines such as 482.6 keV (97\%), 328.5 (93\%) snf 600.5 (62\%)  were not used due to weak feeding from other nuclei, or being present in the background spectra. Thus, the total number of measured points are only five in the energy region 16-26 MeV, since stronger gamma-lines were excluded in the activity calculations. The end of beam activity was estimated using a one-step decay chain. The measurement of the reaction cross section for this nucleus is in well compliance with the results published by F. Tarkanyi et. al. (2006, 2019), magnitude wise (a little on the high side, but within uncertainty). There is however difficult to see if there is any clear curve shape of the excitation function, as especially the datapoints from F. Tarkanyi et. al. (2006) seems to have a high variation in the cross section values. The datapoints from F. Tarkanyi et. al. (2019) seems to be more systematic, along with smaller energy bins. The measured datapoints in this work suggests that the cross section increases with beam energy at least up to 26 MeV, which seems to be the general tendency for F. Tarkanyi et. al. (2019), without a few exceptions where the measured cross section decreases. Since this isomer only can be produced in one particular way, the expected shape of the excitation function is one global maximum, which should be shifted towards the low energy side due to the low energy threshold. There is no TALYS, TENDL or ALICE reaction code-data for this specific isomer (per april 2020), so it is difficult to determine the exact shape of the excitation function. With as long half-life as this isomer has, the foil should have been counted over a longer timeperiod after end of beam. Then multiple other gamma-lines could have been used to measure the end of beam activity, by excluding the timepoints where multiple nuclei feeds the same gamma-energy. 

\subsubsection{\centering{\small{$^{188}$Pt (Independent)}}}
$^{188}$Pt ($t_{1/2}$=10.2 d), $\epsilon/\beta^+$-decay to $^{188}$Ir (100\%). $^{191}$Ir(d,5n) with a reaction Q-value -26109.0 keV and an energy treshold 26384.0 keV. $^{193}$Ir(d,7n) is not possible in this deuteron energy range under 40 MeV. Thus as expected, the measured cross sections are three points in the energy range 26-30 MeV, and the points are in the threshold region, which is in compliance with both talys and tendl. The experimental data done by T. Tarkanyi et. al. (2006, 2019) have not measured any cross sections below 32 MeV, but the measured points supports that the threshold should be close to where we measured the first points, and then the excitation function increases for this specific reaction route. The gamma-lines that were used are listed in table ..., and the end of beam activity was found via single-step decay.  Around the threshold, ALICE, TALYS and TENDL agrees with the measuerd data. 

\subsubsection{\centering{\small{$^{189}$Pt (Independent)}}}
$^{189}$Pt ($t_{1/2}$=10.8 h) decays by $\epsilon/\beta^+$-decay to $^{189}$Ir (100\%). This radionuclide can be produced via $^{191}$Ir(d,4n) with a reaction Q-value -19389.0 keV and energy threshold 19593.0 keV, or via $^{193}$Ir(d,6n) with reaction Q-value -33359.0 keV and energy threshold 33707.0 keV. The measured cross sections suggest a energy threshold around 19 MeV, but the threshold of 19.5 MeV is within uncertainty in energy. Talys and Tendl suggests threshold 19.8 MeV. It seems like there is some disagreement on where the excitation function peak should be in the 30-35 MeV region, where in general, datapoints provided by this measurement suggests a higher cross sections shifted to slightly lower energies than experimental data by F. Tarkanyi et. al. (2006, 2019). It is also hard to say whether the next point would have been higher in cross section, if we could have measured that, but if not, TENDL and TALYS supports this curve shape. The 2006 data also suggests a lower but broader cross section peak, than the 2019 data. From TALYS and TENDL, it seems like the cross section is increasing from 37 MeV, where the 6n decay channel opens. The gamma-lines used are listed in table .., and to estimate the end of beam activity, the decay curve was estimated using a single-step decay. 

\subsubsection{\centering{\small{$^{191}$Pt (Independent)}}}
$^{191}$Pt ($t_{1/2}$=2.83 d) decays by $\epsilon/\beta^+$-decay to stable $^{191}$Ir (100\%). This radionuclide can be produced via $^{191}$Ir(d,2n) with a reaction Q-value -4017.0 keV and an energy threshold 4060.0 keV, or via $^{193}$Ir(d,4n) with a reaction Q-value -17988.0 keV and an energy threshold 18175.0 keV. In this work, ten independent measurements were made for this product nucleus, and the first one is right above energy-trehsold at ca. 5.5 MeV, and the two peaks in the cross section matches the different reaction channels which opens at ca 4 MeV and 18 MeV. The measured data in this work matches the experimental data from F. Tarkanyi et. al (2016, 2019). The end of beam activity was estimated using a single step decay. 
\textcolor{red}{REMEMBER that when above threshold, all reactions can happen. And even though for instance d,n is not favoured at higher energies, it can still happen, and thus the second peak may look higher.  }

\subsubsection{\centering{\small{$^{193m}$Pt (Independent)}}}
$^{193m}$Pt ($t_{1/2}$=4.33 d) decays by isomeric transition to long-lived $^{193}$Pt (100\%). The isomer can be produced via $^{193}$Ir(d,2n) with a reaction Q-value -3063.5 keV and energy threshold 3095.5 keV. Made 10 independent measurements for this reaction. Comparing to experimental data, the result look reasonable. Talys and tendl peak seems to be shifted toward higher energies, while ALICE seems reasonable position wise, suggesting a excitation function maximum at ca. 12 MeV, which matches the experimental data well. However, for higher energies ALICE looks strange. Due to the weak intensity of the gamma-line provided by the IT decay, we had to use X-rays to be sure that we got the right cross section. However due to many X-rays being fed only 66.831 keV was used. Since 66.831 keV is also an X-ray present in $^{192}$Ir ($t_{1/2}$=73.829 d) (intensity is 4.44\%), this X-ray was used to obtain the $^{193m}$Pt activities. Single-step decay was used to obtain end of beam activity. 

\subsubsection{\centering{\small{$^{52}$Mn (Cumulative)}}}
$^{52}$Mn ($t_{1/2}=5.591$ d) (figure \ref{fig:blablabla}) is reported as a cumulative cross section, because of the unidentified decay rate of $^{52}m$Mn ($t_{1/2}=21.1$ m). The isomer and ground state share multiple lines, and using the values after the isomer was decayed completely ($\simeq 5 hours =$ 10 half lives after end of beam), the activities were fitted to a single step decay. Hence the cross section is cumulative. From table \ref{tab:Products_information}, the Q-value\footnote{Calculating the Q value for various reactions: http://hyperphysics.phy-astr.gsu.edu/hbase/NucEne/coubar.html} in which $^{52}$Mn can be produced is -1235.6 keV, so the threshold is well within the energy range in which this work is operating in. As we can see in figure \textcolor{red}{figure}, the first observation is approximately at 14 MeV. The Coulomb barrier for the compound nucleus $^{58}$Ni(d)$^{58*}$Cu, the height is 12.7509 MeV \textcolor{(calculated from equation in theory)}, thus the chance of tunneling before that energy is relatively low. Once the reaction channel from $^{60}$Ni (with a Q value of -21622.6 keV), the increase in cross section is visible. The gammaslines which were used are strong, and as a consequence, the uncertainties in the cross section is relatively small. \\ 

$^{52}$Mn is an odd-odd nucleus with 25 protons and 27 neutrons. Both proton and a neutron is unpaired, so this decay channel is not necessarely favoured as a d,pn, d,3p3n etc. 

 
\subsubsection{\centering{\small{$^{54}$Mn (Independent)}}}
$^{54}$Mn ($t_{1/2}=312.20$ d) is an independent measurement of the cross section. There are multiple measurements at low cross section values.  The (d,2$\alpha$) reaction channel is available at most energies, but the Coulomb barrier is which is at 12.7509 MeV, suggests that the probability of tunneling is relatively low below 10 MeV. The cross section first starts rising at around 18-20 MeV, where multiple reaction channels opens. In general, the measured cross sections done in this thesis matches the earlier experimental data within uncertainties, allthough a little on the high side in the low energy region. 

\subsubsection{\centering{\small{$^{59}$Fe (Independent)}}}
There is no experimental data in this reaction, however the measured cross section data points agree well with talys. Tendl apparently has a harder time to find a good cross section model. This is not a strongly fed channel, first of all this is an even-odd nucleus with 26 protons and 33 neutrons, leaving one unpaired neutron. If we interpret the cross section correctly, the cross section starts to increase around when overcoming the coulomb barrier, and keeps increasing while other reaction channel opens. 

\subsubsection{\centering{\small{$^{55}$Co (Independent)}}}

Agrees very well with other experimental data. Z=27, N=28. Cs not super high. 

\subsubsection{\centering{\small{$^{60}$Co (Independent)}}}
Tendl looks off, however talys follows the experimental data pretty well. As we can see the experimental data is spread in the high energy region, and it is hard to say if actually have made good values. However we are well above energy threshold. 

\subsubsection{\centering{\small{$^{56}$Ni (Independent)}}}
The only way $^{56}$Ni can be produced is via $^{58}$Ni. The threshold is 



\newpage


\noindent \textcolor{blue}{The cross sections for the iridium products are listed in table \textcolor{red}{...}. Give a brief discussion on the varios products. And also, give a discussion of what we expected and what we did not see. THIS IS NOT A FINAL VERSION } In this experiment, $^{188}$Pt, $^{189}$Pt, $^{191}$Pt, $^{193m}$Pt, $^{188}$Ir, $^{189}$Ir, $^{190}$Ir, $^{192}$Ir, $^{194}$Ir, $^{194m2}$Ir, $^{188}$Re, $^{189}$Re and $^{190}$Re was observed. The products we did not observe was $^{183}$Ta (t$_{1/2}$=5.1 d), $^{186}$Re (t$_{1/2}$=3.7186 d), $^{186}$Ta (t$_{1/2}$=10.5 m), $^{18}$W (t$_{1/2}$=24.0 h). \\

\noindent $^{183}$Ta has a Q-value equal to 0.0 MeV for $^{191}$Ir(d,d$\alpha$)$^{183}$Ta and 4.020 MeV for $^{193}$Ir(d,nt2$\alpha$)$^{183}$Ta. The strongest gamma-line ($E_\gamma=59.318$ keV, $I_\gamma=42.1\%$) was not observed in any spectrum, and the second strongest gamma-line ($E_\gamma=246.059$ keV, $I_\gamma=27.2\%$) is within 1 keV from a gamma-line belonging to $^{189}$Ir ($E_\gamma=245.1$ keV, $I_\gamma=6.0\%$). Hence, we conclude that $^{183}$Ta was not observed. \\

\noindent $^{186}$Re has a Q-value equal to 0.0 MeV for $^{191}$Ir(d,t$\alpha$)$^{186}$Re and 13.1264 MeV for $^{193}$Ir(d,2nt$\alpha$)$^{183}$Ta. There is no data in the EXFOR database, and the reaction modelled cross sections provided by TALYS predicts that the cross section is \textcolor{red}{very low or zero} in this energy range. The two strongest gamma-lines were observed, ($E_\gamma=122.64$ keV, $I_\gamma=0.603\%$, $E_\gamma=137.157$ keV, $I_\gamma=9.47\%$). However, the activity which was estimated was not looking natural, hence the cross section curve is difficult to interpret \textcolor{red}{Might need some more work.} 

