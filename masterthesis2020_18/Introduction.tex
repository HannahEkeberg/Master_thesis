

......


Nuclear medicine is today an accepted field in medicine, where radionuclides are important in both therapautic and diagnostic purposes \footnote{https://sci-hub.tw/https://doi.org/10.1524/ract.1995.7071.special-issue.249}, and has especially  been of importance to diagnose and treat cancer. \\ 

Before a radionuclide can be used clinically, a well established excitation function must be known, to maximise the production and minimizing impurities. There already exists large amounts of information on neutron induced reactions. However the information on charged-particle induced reactions is not as strong, so we need more data on this behalf \foonote{(Syed M. Qaim. Nuclear data for production and medical application of radionuclides: Present status and future needs. Nuclear Medicine and Biology, 44:31–49, jan 2017.).} Along with neutron dryness, the search for alternative production methods using accelerators or charged particle beams are good!  
This work is a small contribution to nuclear reaction data for potentially useful platinum and iridium radionuclides, and a contribution to nuclear reaction data from deuteron induced reactions on copper, iridium, nickel and iron. 

Platinum and iridium radionuclides ! 

Nuclear cross section are usefull in many applications, such as ...., 


\subsection{A brief overview of the medical radiomedicine status today }

The knowledge of nuclear decay data, the ionizing radiation is crucial, the effect of low energy high intensity electrons emitted following EC and IT important (\cite{international2012iaea}, p. 3). Need to know the whole contribution to good and bad dose. Half-lives in the order 6h to 7 days dependent on uptake and retention in tumor. Suitable linear energy transfer. Ratio of non-penertrating to penetrating photon radiation should be high since it does not have the therapautic effect \cite{international2012iaea, p. 1}. Selective concentration and prolonged retention of the therapautic radionuclide in tumor and minimum uptake in normal tissue \cite{international2012iaea, p. 2}. 

Auger: ranges of about 10 $\mu$m, $\alpha$: 100 $\mu$m, beta: 1mm or more. Dependent on energies, beta will have therapautic effect if reach cell environment. In order to have the desired therapautic effect for alpha and auger, involves great skill in biochemistry and 

Therefor, thin target production are useful, because it gives nuclear production data 
In general, the database of decay characteristics nndc is good, but for instance auger spectra and knowledge lacks (\cite{international2012iaea}, p. 3)

Nuclear decay data very important when choosing radionuclide, but the nuclear reaction data is very important in the optimizing process, achieving max yield combined with minimal level of impurities ((\cite{international2012iaea}, p. 3)). This means that the cross section data is very very important! Via cyclotron, many processes can be used. Data requirements are very impotant since many other reactions occur. Proton and deuteron reactions are preffered since lead to higher yields. The ultimate choice depends on the availability of a suitable cycltron and the required target material (\cite{international2012iaea}, p. 4)





