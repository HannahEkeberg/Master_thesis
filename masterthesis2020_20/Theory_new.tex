\chapter{Targeted radionuclide therapy}


Today, multiple options for treatment of cancerous tissue are available, such as chemotherapy, surgery, immunotherapy, external beam therapy, bracytherapy and targeted radionuclide therapy. The latter three are treatment types utilizing ionizing particles to induce damage to the DNA. In external beam therapy X-rays, high-energetic gamma-rays, or accelerated particles like protons and heavier ions are focused externally towards the tumor, and in bracytherapy an unsealed radioactive source (usually a wire or pellet containing for instance a $\bets^-$-emitter), is placed in proximity to tumor (\cite{Vertes2011a}, p.2180). Targeted radionuclide therapy is an emerging alternative, which can deliver a cytotoxic level of dose to the cite of disease (\cite{Vertes2011a}, p. 2180). It offers a patient-specific treatment dependent on choice of radiopharmaceutical which targets a type of tumor or cell. A radiopharmaceutical consists of a radionuclide and a cell-targeting molecule called a tracer. Meanwhile brachytherapy and targeted radionuclide therapy are limited by the cancer location and the existence of metastasis, along with required knowledge of the tumor to maximise the dose over the tumor and minimizing the dose to healthy tissue (\cite{Vertes2011a}, p.2180), targeted radionuclide therapy utilizes radiopharmaceuticals which are typically injected intravenously and utilized the biochemical pathways in the body. thus with an appropriate tracer, targeted tissue with an high uptake of the radiopharmaceutical will receive a high dose, and healthy tissue can be spared \cite{Yeong2014a}.\\ 

A therapeutic agent need to have the two components of the radiopharmaceutical optimized for the radiation from the radionuclide to have a high probability of being deposited in the tumor, and ideally deliver a cytotoxic dose to all cancerous cells within a tumor while sparing all healthy cells. For instance, a high uptake-rate of the tracer suggests a shorter half-life of the radionuclide than slow uptake and long retention in tumor. The decay mode and radiation range need to be in coherence with the size and location, as well as the geometry of the tumor. Ranges from multi-cellular, cellular and sub-cellular are typically accomplished with $\beta^-$-particles, $\alpha$-particles and auger electrons, respectively (\cite{Vertes2011a} p. 2180-2182). Figure \ref{fig:cell_dimension} shows how the ranges of $\beta^-$, $\alpha$ and auger electrons differ on the cellular scale, where low energetic auger electrons have ranges on the scale of the cellular nucleus, while high energetic $\beta^-$-particles have ranges of up to several hundred cell diameters. \\ %\textcolor{red}{Geometrical factors of both the distribution and the tumor it self can have a degree of variations in the dose distribution due to differences in cross-fire dose and the fraction of the radiation bound to the cell that is deposited in the tumor. Particularly apparent for micrometastatic disease which presents as small cluster of tumor cells, magingying the impact of these factor. In addition, it is important to achiece a homogeneous dose deposition within the tumor, so that regrowth from an untreated subpopulation will be avoided. }. (all above in paragraph: handbook p. 2180-2182).


\begin{figure}
    \centering
    \includegraphics{Theory/cell_dimension.png}
    \caption{The figure illustrates the ranges of auger electrons, 5.3 MeV alpha particles and low and high energetic $\beta^-$ particles. Figure is from \cite{international2012iaea}, p. 2}
    \label{fig:cell_dimension}
\end{figure}


For the tracer, properties such as a rapid blood clearance and transport (\cite{Vallabhajosula2009a} p. 145) and high uptake and retention in the tumor \cite{international2012iaea} p. 2) are important characteristics, along with tissue-targeting \cite{Yeong2014a}. It can target the desired cells by for instance a specific receptor, enzyme, membrane, transporters or antigens (\cite{Vallabhajosula2009a}, p. 145). Radiometals are also used, which consists of a bifunctional chelator, which is a molecule containing molecules which can donate a lone pair of electrons, like nitrogen, oxygen or sulfur. If the radiometal has an oxidation state of $3^+$, it will be tighlty bound by the chelator, and can transported to the tumor\cite{Ekeberg2020}. Figure \ref{fig:therapy_chelator_peptide} shows an illustration of how an example of how a radionuclide can be transported into the desired cell attached to a chelator, via a specific peptide. \\
\begin{figure}
    \centering
    \includegraphics[width=12cm\textwidth]{Theory/therapy-peptide.jpg}
    \caption{A radionuclide is bound to a chelating agent, and with a peptide, the radiopharmaceutical targets the cancer cells. Figure is from \cite{Thundimadathil2012a}. }
    \label{fig:therapy_chelator_peptide}
\end{figure}

For the radionuclide, along with range and decay mode, the half-life, production method, chemistry and biological behavior are important characteristics (\cite{Vertes2011a}, p. 2181). %, there are physical properties concerning the radionuclide, such as physical half-life, decay-mode and decay product, radiation energy and in-tissue range, and biological properties concerning the tracer such as tissue targeting, biological half-life, retention in tumor and the uptake in healthy tissue (3).  Thus, the radiopharmaceutical requires two components in which complement each other to deposit the dose in the cancerous tissue. 

In nuclear medicine, the effective half-life of the radiopharmaceutical is important as it takes both the physical half-life and the time it takes for the radiopharmaceutical to be cleared or excreted from the body \cite{Yeong2014a}. The physical half-life must be long enough to permit radiosynthesis and qauality control (\cite{Vertes2011a}, p. 2185). However, in therapy, high radiation dose is desired, which is easier to achieve with shorter half life, so that should also be compensated for. The choice of radionuclide should match the uptake rate and the retention of the cell-targeting molecule, to avoid radioactive waste handling and dose to healthy tissue \cite{Yeong2014a}. Therapautic radionucldies typically have half-lives  in order of a few hours to several days (\cite{international2012iaea}, p. 1). In addition, knowledge about the decay product is important, for instance whether it is toxic or if it takes place in natural processes in the body, or if contributes to an additional undesired dose. The chemical-biological properties are also relevant, as it must be chemically possible to attach radionuclide to the targeting molecule, and the binding must stay stable in the body, over a time period which is stable as long as the physical half life (\cite{Vertes2011a}, p. 2185). \\

\noindent
Along with the ability to use radionuclides in therapy, radionuclides can also be used for diagnostic purposes with imaging, either with positron emitters (positron emission tomography) inwhich annihilates with atomic electrons close to the cite of decay, and sends out two observable 511 keV photons, or emission of a strongly fed gamma-energy which is detected (single photon emission tomography). The combination of a diagnostic and a therapautic agents with similar properties so that the biochemical uptake in the body is the same, is a new approach in which can give information of how the uptake is distributed in the body, and can image the state of decease after therapies \cite{Rosch2017a}. This is called theranostics. 

\section{Particle interaction in tissue}  \label{sec:particle_interactions}

Ionizing radiation are particles with sufficient energy to cause ionizations along the particle track, thus separating an atom and one or more electrons. The free electron(s) can ionize further, and the positive ion can cause undesired reactions. DNA is a large molecule with two strands bound in a double helix structure. Each strand is composed of sugar and phosphate groups, and nitrogenous bases which bind the two strands (\cite{Douglass2018} p. 11). These  bases are called adenine \& guanine amd cytosine \& thyamine (always bound pairwise), and are bound through weak hydrogen bonds which are exposed for strand breaks. The cell is equipped with an impressive repair mechanism, and unless both strands of the DNA are damaged, called a double stranded break (DSB), most damages are repaired. Radiation damages in the DNA can be caused directly by the ionizing particle or indirectly via free radicals, which are subject to other ionizations. Since the body contains large amounts of water, ionization of water molecules giving for instance H$^\bullet$ or OH$^\bullet$ are important damaging factors. Damages induced in the DNA can be lethal to the cell and either cause apoptosis or mutation in which can cause cancer. In therapy, the goal is to make malignant cells to undergo apoptosis, thus DNA is referred to as the target (\cite{Douglass2018}, p. 9). Choosing a particle with a high probability of inducing damage will induce multiple double stranded breaks if passing near by. 

Linear energy transfer (LET) describes the energy absorbed by the medium, and is defined as the average energy deposited per unit length of the density material (\cite{Douglass2018}, p. 101). 



\begin{equation}
    \text{LET}=\frac{dE}{dx}
\end{equation}

To maximise the chances of inducing damages in the DNA and minimizing exposure of healthy tissue, choosing a particle with a high linear energy transfer is important in targeted radionuclide therapy. Figure \ref{fig:DNA_let} illustrates how $\beta^-$-particles, alpha-particles and auger electrons deposit energy on the scale of DNA. \\

$\beta$-decay occur whenever there is an overweight in number of neutrons/protons, where the former transforms a neutron into an electron, proton and an antineutrino ($\beta^-$-decay): $$n\rightarrow p+e^-+ \overline{\nu_e}$$ The contrary $\beta^+$ decay transforms a proton into a positron, neutron and a neutrino ($\beta^+$-decay): $$p \rightarrow n+e^+ +\nu_e$$ Since the neutron mass is higher than proton mass with $2m_e$ MeV/$c^2$, there is an energy threshold for the reaction to occur. If the reaction is below the energy threshold, electron capture occurs instead, which is an electromagnetic interaction between an atomic electron (creating a vacancy in the atomic shells) and a nuclear proton that transform into a neutron and a neutrino $$p+e^-_\text{atomic}\rightarrow n+ \nu_e$$ For $\beta$-decay, the energy is distributed between three particles, thus the energy of the $\beta$-particle is not discrete.   \\
$\alpha$-decay occurs when the nuclear radius is so large that it is affected by the Coulomb barrier. Thus emission of an alpha particle lowers the binding energy as the alpha particle carries a large amount of binding energy. The $\alpha$-energy are discrete values which can be detected using spectroscopy. \\
Auger electrons are result from electron capture or internal conversion, which happens when a gamma-ray interacts electromagnetically with an atomic electron which is emitted. The vacancy in the atomic shell can lead to a cascade of X-rays and auger electrons with energies in the X-ray range. These energies are discrete, as the X-ray energy is discrete. From $\beta$ and sometimes $\alpha$ decay, the daughter nucleus is left in an excited state and decays by gamma-emission (all from special curriculum \cite{Ekeberg2020}) \cite{KraneKennethS.Halliday1987} (chapters 8,9,10).  \\

\noindent A medium consists of positively charged nuclei and negatively charged electrons. Charged particles have a short range in a medium compared to neutral particles, as the Coulomb force forces the particle to interact continuously along the path either by scattering inelastic with the atomic electrons or scattering elastic with the nuclei. Figure \ref{fig:particle_interaction} shows how various particles such as X-rays, gamma-rays, electrons and protons deposit energy (y-axis) as a function of depth. Elastic scattering is the less dominant process, where the energy loss is small, as long as the nuclei in the medium are larger than the incoming particle \cite{Leo1994}, p. 21.  \textcolor{red}{Inelastic collisions dominates where the atomic electrons are either excited or ionized (which citation???? Instrumentation book?)}. Under the assumption that the collision is elastic, the collision is head-on and the particle has high energy, the maximum energy transfer can be calculated using conservation of momentum and energy

\begin{equation} \label{eq:collision_E_transfer}
    Q_\text{max}=\frac{4m_eM}{m+M}E
\end{equation}

where $m_e$ is the mass of an atomic electron, M is the mass of the incoming charged particle and E is the kinetic energy of the incoming charged particle \footnote{To see calculation , see: https://ocw.mit.edu/courses/nuclear-engineering/22-55j-principles-of-radiation-interactions-fall-2004/lecture-notes/energydeposhcp.pdf}. While LET describes the energy transferred per unit length, the stopping power describes the energy loss of a charged particle per unit distance. The collision loss for heavy charged particles (protons and heavier ions) at high energies is therefor low. The stopping power for heavy charged particles is described by Bethe-Block: 

\begin{equation} \label{eq:betheblock}
    -\frac{dE}{dx} = 2\pi N_a r_e^2 m_e c^2\rho \frac{Z}{A}\frac{z^2}{\beta^2}\Big[\ln \Big( \frac{2m_e\gamma^2v^2W_\text{max}}{I^2} \Big)-2\beta^2 -\delta -2\frac{C}{Z} \Big]
\end{equation}
where
\begin{multicols}{2}
%$2\pi N_a re^2m_e c^2=0.1535 \text{ MeVcm$^2$/g} $\\ 
\noindent
$r_e: $ classical electron radius\\
$m_e: $ electron mass\\
$N_a: $ Avogadro's number \\
$I: $ mean excitation energy \\
$Z: $ atomic number of absorbing material\\
$A: $ atomic weight of absorbing material \\
$\rho: $ density of absorbing material \\
$z: $ charge of incident particle \\
$\delta: $ density correction \\
$C: $ shell correction\\ 
$W_\text{max}: $ maximum energy transfer in each collision\\
$\beta=v/c:$ incident velocity of the particle\\
$\gamma=\frac{1}{\sqrt{1-\beta^2}}: $ Lorentz factor

\end{multicols}

(\cite{Leo1994}, p. 24)

 As the particle slows down, the more energy per unit length will be deposited, as the charged particle picks up electrons. This is known as the Bragg peak, which can be seen as the red curve on figure \ref{fig:particle_interaction}. Most of the energy is deposited near the end stop. The stopping power of heavy charged particles are proportional to the charge of particle and the inverse velocity squared. Therefor, particles with a higher charge will have a higher more specific Bragg-peak and a shorter range in tissue, with the same initial energy. This behaviour of heavy charged particles is especially useful in external beam therapy and is utilized to have a very specific dose over tumor as the dose before is low and the dose after Bragg peak is approximately zero (\cite{Leo1994}, p. 27-28). In addition, to increase the width of the Bragg peak, a spread-out Bragg-peak is obtained by accelerating the particle in a spectrum of energies, which is the blue curve on figure \ref{fig:particle_interaction}.  \\

Electrons can experience energy loss either from  collisions, or via the electromagnetic radiation that arises when electrons are loosing energy (bremsstrahlung), due to the small mass. However, for energies up to a few MeV, the collision energy loss dominates (\cite{Leo1994}, p. 37). For electrons, the maximum energy transfer per collision is half of the initial energy, which means that electrons lose energy fast via collisions. Electrons scatters rapidly, and changes direction continuously due to the equal mass of the atomic electrons. The energy loss of electrons fluctuates much more than heavy charged particles which is due to much greater energy transfer per collision and to the emission of bremsstrahlung. To absorb major part of the electron's energy, is a few collisions, and results in greater range straggling.  (\cite{Leo1994} p. 42). From figure \ref{fig:particle_interaction}, the 22 MeV electrons are loses energy exponentially, \textcolor{red}{due to bremsstrahlung??}\\ 
 
%Beta-electrons have a continuous spectrum of energies and absorption of beta decay electrons exhibit behavour which is well approximated to an exponential form (instrumentation p. 42). 
%Low energetic electrons are small in mass to large angle deflection by scattering from nuclei (p. 48). 


\begin{comment}
\begin{equation}
    -\frac{dE}{dx}=2\pi N_a r_e m_e c^2\rho \frac{Z}{A}\frac{1}{\beta^2}\Big[ \ln \frac{\tau^2(\tau+2)}{2(I/m_ec^2)^2)} + F(\tau) - \delta - 2\frac{C}{Z} \Big]
\end{equation}

where $\tau$ is the kinetic energy of particle in $m_ec^2$ (instrumentation p. 37).  
\end{comment}

Photons and neutrons however are neutral particles and are not energy-degraded. Instead neutral particles are attenuated as a function of distance traversed x and the absorption coefficient $\mu$ of the material 

\begin{equation} \label{eq:photon_attenuation}
    I = I_0 e^{-\mu x}
\end{equation}

where I is the intensity as a function of distance and $I_0$ is the intensity at x=0 (\cite{Leo1994}, p. 53). X-rays produced from a X-ray tube and gamma-rays degrades exponentially, and are thus more penetrating than charged particles. As gamma-emitters are not directly used in targeted  radionuclide therapy, the dose from gamma-radiation following alpha or beta decay, or X-rays following electron capture or internal conversion needs to be taken into account.\\
For high energetic X-rays there is a build up effect, where the photons induce ionizations, and the free electrons contribute to a higher dose. This can be seen for the 22 MV X-rays in figure \ref{fig:particle_interaction}. This is utilized in external beam therapy, maximizing the dose over the tumor. For gamma-rays, the energy is not sufficiently high, so instead the curve follows an exponential form as the dotted line in figure \ref{fig:particle_interaction}.   \\


\begin{figure}
    \centering
    \includegraphics[width=10cm]{Theory/DNA_LET.jpg}
    \caption{The figure illustrates how $\beta^-$-particles, $\alpha$-particles and auger electrons deposit their energy on the scale of DNA. The figure is assembled from \cite{Alotiby2019}, where the upper figure is from \cite{Hillyar2015}, and the lower figure is from \cite{Pouget2011}. 
    %\footnote{https://openresearch-repository.anu.edu.au/bitstream/1885/164319/1/Thesis_Revised\%20copy_last\%20version.pdf}    \footnote{Accurately determining the number of Auger electrons per nuclear decay for medical isotopes - Scientific Figure on ResearchGate. Available from: https://www.researchgate.net/figure/Biophysical-properties-of-beta-alpha-and-Auger-electron-emitting-radionuclides-uper_fig1_334879307 [accessed 2 May, 2020]}}
    }
    \label{fig:DNA_let}
\end{figure}


\begin{figure}
    \centering
    \includegraphics[width=0.6\textwidth]{Theory/Bragg-peak-and-Spread-Out-Bragg-Peak-SOBP-for-a-proton-beam-in-comparison-with-photon.jpg}
    \caption{Medium depth along x-axis, energy deposition in tissue (or dose?) on y-axis. The figure is from \cite{Cianchetti2012a}. }
    \label{fig:particle_interaction}
\end{figure}


In general,  $\beta^-$-particles have relatively low LET and are thus suited for treating large tumors, but the dose to healthy tissue may be hard to avoid. $\alpha$-particles have short range in tissue, typically one to a few cells in diameter. Has a high LET-value, radiation with LET=100 keV/$\mu$m has the distance between ionizing events is nearly identical to that between DNA strands increasing the probability of creating highly cytotoxic double strand breaks \cite{Vertes2011a}, chapter "Targeted radionuclide therapy". One of the major problems with $\alpha$-emitters is the decay products, as a typical $\alpha$-decay chain results in multiple emission of $\alpha$ and $\beta$-particles. For low energetic electron emitters such as auger emitters, the range is so low that in order to deposit energy in the DNA, it must be incorporated into the cellular nucleus. Thus, it will only affect the cell targeted, and as we can see in figure \ref{fig:DNA_let} when incorporated into DNA, can induce many damages in the DNA.  \\





\section{Production of radionuclides}
%The ideal radiopharmaceutical delivers a lethal dose to all cells within a tumor, while sparing normal tissue. ()

%(p. 2183): Also important to keep in mind that even with a uniformly distributed radionuclide, the dose received by individual cells can vary because of the differences in cross fires dose and the fraction of the radiation bound to the cell that is deposited in the tumor. This is particularly apparent for micrometastastic desease which presents as small clusters of tumor cells magnifying the impact of these factors. (p.2183): the success of radionuclide therapy is also critically dependent upon achieving homogeneous dose deposition within the tumor so that regrowth from an untreated subpopulation will be avoided. 
%(p.2185):Half life: should be long enough to permit radio synthesis and quality control and in some cases distribution to locations distant from the production site of the radiopharmaceutical. More importantly the half life shold be compatible with the pharmacokinetics of localization in tumor and clearance from normal tissue. Finally its generally easier to achieve higher radiation dose with shorter half-life radionucldies. 
%Cell culture expermiments have showsn that increasing the dose rate for low linear energy transfer radiation can lead to a greater degree of tumor control. 
The radionuclide availability is an important factor in targeted radionuclide therapy. Reactors, cyclotrons and natural decay chains have traditionally been used as radionuclide sources \cite{Vertes2011a}, p. 2185. Proton rich nuclei are typically produced in accelerators/cyclotrons using positively charged particles, and neutron rich nuclei are typically been products of fission or produced in the neutron flux resulting from fission in a reactor. Thus therapeutic radionuclides producing $\beta^-$-emitters needs neutrons, which are the main source of reactors. With research reactors today aging \cite{Yeong2014a}, alternative production routes to produce critically medical radionuclides are important. \\

There are many different production routes available for a single radionuclide, dependent on choice of target, particle beam and beam energy. The production route has an associated reaction cross section which is dependent on the beam energy. The nuclear cross section data is very important in optimization of production processes, achieving the maximum yield of the desired radionuclide combined with the minimum level of radionuclidic impurities \cite{international2012iaea}. A high degree of radionuclidic purity is required for therapautic radiopharmaceuticals depending on the nature of the molecule that will be labelled, specific activity (GBq/mmol) may also be important consideration. It is impossible to chemically separate isotopes of the same element \cite{Qaim2017c}. We want to be sure that the what is injected into the patient does not have isotopic impurities which gives undesired dose to the tissue, nor will we have isotopes with no therapautic effect. This is to yield ffective treatment, but especially in cases where the body does not excrete the element from the body, the result can lead to toxicity effects. Carrier-free production which are molecules which exclusively contain the desired radionuclides is desired because it gives the highest specific activity. The only option to minimize impurities is to choose an appropriate energy window which minimizes the production of co-products. There already exists large amounts of information on neutron induced reactions. However the information on charged particle induced reactions is not as strong so we need more data on this behalf \cite{Qaim2017c}. Production of medical radionuclides should be cheap and available for everyday medical purposes. Accelerators can be small in size and handled easily by medical personnel. Many hospitals which performs nuclear medicine even have a cyclotron facility on site, which is advantageous as its practical to avoid travelling logistics and to have medical radionuclide supply in proximity of examination/treatment site \cite{Ekeberg2020}.  


%Selection of a radionuclide for targeted radionuclide radiotherapy must also take into account its chemical biological properties. 
%Chemical methods must be devised for attaching the radionucldie to the targeting molecule of interest without compromising the tumor localizing capacity of the molecule. In addition the bond between the radionuclide and the carrier molecule must be stable in the in vivo environment over a time period consistent with the physical half life of the radionuclide in order to minimize damage to normal organs. 


\begin{comment}

%\newpage
%Beta-emitting particles were the first ones which were used, long range (high energy) for large tumors and short range (low) energy  more than 2 mm range for long range particles

%Range of alpha-particles a few cell diameters, offering the prospect of matching the cell-specific nature of targeted molecular carriers with radiation having a similar range of action. Short range alpha particles also may be ideally suited for minimal residual disease settings applications in which radionuclide therapy has the greatest chance of making a meaningful clinical impact. These include treating micrometastasis, residual tumor margin left after surgival debulking, and tumors present in the circulation such as leukemia and lymphoma. Have much higher energies than beta paricles, and along with short range, have very high LET. Radiation at 100 keV  keV/$\mu$m has this quality because at this LET, the distance between ionizing events is nearly identical to that between DNA strands, increasing the probability of creating highly cytotoxic double DNA strand breaks. 

%Low energy electron emitters: deposit energy in subcellular dimensions. Can arise from two decay processes; internal conversion and electron capture. Both create inner atomic shell vacancies that are transferred by a series of rearrangments to outer atomic shells. the binding energy difference between the inner and outer subshells may then be transferred to other outer shell electrons, that are ejected from the atom. These electrons are called auger electrons, if they arise from higher shells and coster-kronig electrons if they originate from higher subshells. 

%\begin{figure}
%    \centering
%    \includegraphics[width=10cm]{Theory/DNA_LET.jpg}
%    \caption{The figure illustrates how $\beta^-$-particles, $\alpha$-particles and auger electrons deposit their energy on the scale of DNA. 
    
%    \footnote{Accurately determining the number of Auger electrons per nuclear decay for medical isotopes - Scientific Figure on ResearchGate. Available from: https://www.researchgate.net/figure/Biophysical-properties-of-beta-alpha-and-Auger-electron-emitting-radionuclides-uper_fig1_334879307 [accessed 2 May, 2020]}}
%    \label{fig:DNA_let}
%\end{figure}

\begin{figure}
    \centering
    \includegraphics[width=0.6\textwidth]{Theory/Bragg-peak-and-Spread-Out-Bragg-Peak-SOBP-for-a-proton-beam-in-comparison-with-photon.jpg}
    \caption{Medium depth along x-axis, energy deposition in tissue (or dose?) on y-axis. Find citation in special curriculum. }
    \label{fig:particle_interaction}
\end{figure}


- What is targeted radionuclide therapy
- Biological effect
- Characteristics: half-life, decay mode, 
- LET, stoppingpower



\textcolor{blue}{Whenever something is cited like (3), it means citation 3 in special curriculum Special curriculum p.4:}  many requirements before a radiopharmaceutical can be used clinically, there are physical proper-ties concerning the radionuclide, such as physical half-life, decay-mode and decay product, radiationenergy and in-tissue range, and biological properties concerning the tracer such as tissue targeting,biological half-life, retention in tumor and the uptake in healthy tissue (3).  Thus, the radiopharma-ceutical requires two components in which complement each other to deposit the dose in the canceroustissue.  For the tracer, a rapid blood clearance and transport (6, p.  145) and high uptake and reten-tion in the tumor (9.  p.  2) (special curriculum p.  4) are important characteristics.  It can targetthe desired cells by for instance a specific receptor, enzyme, membrane, transporters or antigens (6,p.  145).  Radiometals are also used,  which consists of a bifunctional chelator,  which is a moleculecontaining molecules which can donate a lone pair of electrons, like nitrogen, oxygen or sulfur.  If theradiometal has an oxidation state of $3^+$, it will be tighlty bound by the chelator, and can transportedto the tumor (special curriculum p.  4-5). 

Dependent on size and geometry of the tumor, along with localization, a radionuclide decay mode and radiation range in tissue are important characteristics. For instance, if tumor is small and in a critical region must be ...., also if up to brain must be smaller than BBB. Characteristics of emission, half-life, production method, chemistry and biological behaviour (handbook of nuclear chemistry, p. 2181). 

The ideal radiopharmaceutical delivers a lethal dose to all cells within a tumor, while sparing normal tissue. 

(p. 2183): Also important to keep in mind that even with a uniformly distributed radionuclide, the dose received by individual cells can vary because of the differences in cross fires dose and the fraction of the radiation bound to the cell that is deposited in the tumor. This is particularly apparent for micrometastastic desease which presents as small clusters of tumor cells magnifying the impact of these factors. (p.2183): the success of radionuclide therapy is also critically dependent upon achieving homogeneous dose deposition within the tumor so that regrowth from an untreated subpopulation will be avoided. 
(p.2185):Half life: should be long enough to permit radio synthesis and quality control and in some cases distribution to locations distant from the production site of the radiopharmaceutical. More importantly the half life shold be compatible with the pharmacokinetics of localization in tumor and clearance from normal tissue. Finally its generally easier to achieve higher radiation dose with shorter half-life radionucldies. 
Cell culture expermiments have showsn that increasing the dose rate for low linear energy transfer radiation can lead to a greater degree of tumor control. 
Radionuclide availability: reactors, cyclotrons, natural decay chains have all been utilized as sources of radionuclides for targetred radionuclide therapy. A high degree of radionuclidic purity is required for therapautic radiopharmaceuticals depending on the nature of the molecule that will be labelled, specific activity (GBq/mmol) may also be important consideration 
Selection of a radionuclide for targeted radionuclide radiotherapy must also take into account its chemical biological properties. CHemical methods must be devised for attaching the radionucldie to the targeting molecule of interest without compromising the tumor localizing capacity of the molecule. In addition the bond between the radionuclide and the carrier molecule must be stable in the in vivo environment over a time period consistent with the physical half life of the radionuclide in order to minimize damage to normal organs. 
\end{comment}


%\subsection{Cyclotron production using deuterons}
%The deuteron is  losely bound particle, with a break up energy blablabla. 
%In order simulate the detueron beam, anderson and ziegler stopping power was used. blablabla just writted down from head. 

%Deuteron stopping power: 
%Anderson\&Ziegler stopping power formalism.
%Is just the addition of the two corrections density effect ad shell correction??






\section{$\mathbf{^{193m}}$Pt as a potential therapautic agent}
$^{193m}$Pt ($t_{1/2}$=4.33 days) is an auger-emitting isomer which decays by isomeric transition (100\%) to the long-lived $^{193g}$Pt groundstate ($t_{1/2}$=50 years) \cite{ShamsuzzohaBasunia2017a}. Radionuclides produced from deuterons on natural iridium such as $^{191}$Pt, $^{193m}$Pt, $^{192}$Ir and $^{194}$Ir are believed to have potential in medicine, like chemotherapy, brachytherapy, radioimmunotherapy and imaging \cite{Tarkanyi2006a}. Platinum radionuclides are of special interest, as platinum is the main element in chemotherapautic agents such as cisplatin, which is a drug which is used clinically in treatment of testicular and ovarian cancer mainly, but also to treat esophagus, head and neck and bladder cancer \cite{Areberg1999}. Cisplatin  (cis-dichlorodiammine platinum(II)) is an inorganic molecule which contains one stable platinum atom surrounded by two clorine atoms and two ammonia molecules (NH$_3$). The cisplatin-molecule enters the cell nucleus, and binds to the DNA, example-wise shown in figure \ref{fig:cisplatin_DNA}, where the clorine-atoms are de-attached and the platinum-atom binds through covalent bonds to the DNA base guanine (and in some cases adenine, \textcolor{red}{is that correct?}), and breaks the bonds between the DNA nitrogeneous bases. Cisplatin thus targets the DNA. One of the major challenges with cisplatin is the chemical toxicity. However, when auger-emitters such as $^{193m}$Pt or $^{195m}$Pt replace the stable platinum atom, the local auger-damage effect increases the chemical damage of cisplatin, suggesting that a smaller amount of the drug is required, and the toxicity-limitations can be avoided \cite{HowellRogerW.SastryKandulaS.R.HillHeleneZ.Rao1985} p. 493. %\footnote{http://citeseerx.ist.psu.edu/viewdoc/download?doi=10.1.1.987.2577&rep=rep1&type=pdf#page=506, p. 493}.  \textcolor{red}{exactly how does it target the specific target cells, how is the uptake? how does the uptake of a healthy cell differ from cancer cells?}

%By replacing either of the stable nitrogen atoms with the PET-radionuclide $^{13}$N ($t_{1/2}$=9.965 minutes), or by a radionuclide platinum, where $^{191}$Pt ($t_{1/2}$=2.83 days, decay by electron capture (100\%) to $^{191}$Ir (stable)) , $^{193m}$Pt and $^{195m}$Pt ($t_{1/2}$=4.010 days, decay by isomer transition (100\%) to $^{195g}$Pt (stable)) is of special interest, cisplatin can be used for imaging or therapy\footnote{https://www.sciencedirect.com/science/article/pii/S0969804399000822?casa_token=ZLJ8YPQzGZMAAAAA:264QzKWpH8Kv6iHotiGMeoHTk8jKqmnoDgf709SrAD8BUWVwbRXriZbHgkYO_tHg-2qyX3Hvt9E}, but therapy is most common. \\ 

\noindent 
\textcolor{red}{As $^{191}$Pt is an electron-capture emitter, can be used in imaging, with for instance 129.4 keV (38.0\%) or 172.19 keV (43.2\%). Combining $^{191}$Pt with a therapautic agent might be possible for theranostic pair with either $^{193m}$Pt or $^{195m}$Pt? Can be combined with therapy as it releases auger electrons? }\\

%\subsection{Decay of $^{193m}$Pt}
Gamma-decay is a result of de-exitation of a nucleus with the release of a photon equal to the energy difference between the two states. The typical half-life of a populated excited state is less than $10^{-9}$ seconds, and states with longer half-lives are called isomeric states (\cite{KraneKennethS.Halliday1987} p. 175). This isomer decays by isomeric transition. In all decays, there are certain quantities which needs to be conserved; angular momentum, parity. A multipole of order $\ell$ transfers angular momentum $\ell \hbar$ per photon (\cite{KraneKennethS.Halliday1987} p.333). A nuclear state has a definite angular momentum $\ell$ \textcolor{red}{(ang mom + spin?)} and parity, and if a gamma transition is to happen between two states the photon must connect the two states by conserving angular momentum and parity. In order for the quantity $\ell$ to be conserved, the angular momentum can be integers between 
\begin{equation}
    |I_i - I_f| \leq \ell \neq I_i + I_f
\end{equation}
where i is initial spin and f is final spin. I is the total spin (angular momentum and spin). The parity decides whether  the radiation is electric multipole or magnetic multipole (equations \cite{KraneKennethS.Halliday1987} p.311)

\begin{equation}
\pi (ML)= (-1)^{\ell+1}, \quad \pi(EL) = (-1)^\ell    
\end{equation}

There are three populated states for this nuclide, the isomer state at 149.8 keV, with nuclear spin $13/2^+$ (4.33 d), a state at 14.3 keV with nuclear spin $5/2^-$ (2.52 ns), a state at 1.6 keV with nuclear spin $3/2^-$ (9.7 ns) and the ground state at 0.0 keV with nuclear spin $1/2^-$ (50 y)\cite{ShamsuzzohaBasunia2017a}. The decay scheme can be seen in figure \ref{fig:decayscheme_193mPt}. 

\begin{figure}
    \centering
    \includegraphics[width=0.8\textwidth]{Theory/decay_scheme_193mPt.png}
    \caption{The decay scheme of $^{193mPt}$. From (https://www.nndc.bnl.gov/nudat2/replotdec.jsp). Figure from Nudat 2.8 database \cite{VRAPCENJAKLidijaZERKIN2015}. }
    \label{fig:decayscheme_193mPt}
\end{figure}

For the decay of $^{193m}$Pt (E level=149.8 keV) to the excited state (E level=14.3 keV), the spin and parity changes from $13/2^+$ to $5/2^-$, which gives possible values for $\ell=4,5,6,7,8,9$. The electric decays have even parity when $\ell$=even, and magnetic has even when $\ell=$odd. If parity is unchanges in the decay ($\Delta\pi$=no), the electric multipoles are even and magnetic multipoles are odd. If the parity does change ($\Delta \pi$=yes) there would be odd electric and even magnetic multipoles. Hence for the possible transitions between $13/2^+$ to $5/2^-$ are whenever $\Delta \pi$=yes and $\ell=4,5,6,7,8,9$, which gives possible M4, E5, M6, E7, M8 and E9 transitions. \\

In general, the lowest possible multipole dominates, and the emission of a multipole of one order higher  ($\ell+1$ than $\ell$) is reduced by a factor ca. $10^{-5}$ (\cite{KraneKennethS.Halliday1987}, p. 335). Thus a multipole of order 4 or 5 has a low probability of occurring and thus the isomer has a long half-life. In comparison, the decay from $5/2^-$ to $3/2^-$ gives possible radiation $\ell=1,2,3,4$, $\Delta\pi$=no, gives possible M1, E2, M3, E4 and the same for decay from $3/2^-$ to $1/2^-$. \\

The gamma-lines emitted from $^{193m}$Pt are very weak (figure \ref{fig:decayscheme_193mPt}). Whenever gamma-decay is possible, another process called internal conversion is competing. It is an electromagnetic process where the nucleus electromagnetically with the atomic electrons, and an atomic electron is emitted instead of the photon (\cite{KraneKennethS.Halliday1987}, chapter 10, p. 341). The kinetic energy of the emitted electron is the transition energy minus the electron binding energy 

\begin{equation}
    T_e = \Delta E -B
\end{equation}

where B is the binding energy. The emitted electron is called a conversion electron, and the energy is comparable to the gamma-ray energy. The conversion electron varies with the atomic orbital (\cite{KraneKennethS.Halliday1987}, p.??), and the electrons following internal conversion are in a spectrum of different discrete energies. The transition energy must be higher than the electron binding energy, and as a consequence the electron is labelled with the shell it was emitted from \textcolor{red}{(remember that atomic shells are labelled with n: n=1=K, n=2=L, n=3=M, n=4=N, etc)}. \\

\noindent 
For $^{193m}$Pt, internal conversion is highly favoured before gamma-decay, thus the observed gamma in gamma-ray spectroscopy is weak. The total probability is the summed decay probability for gamma-decay and internal conversion 

\begin{equation}
    \lambda = \lambda_\gamma + \lambda_\text{IC}
\end{equation}

and the internal conversion coefficient $\alpha$ can be defined as 
\begin{equation}
    \alpha = \frac{\lambda_\text{IC}}{\lambda_\gamma}
\end{equation}

High values for $\alpha$ indicates high probability of internal conversion relative to the probability of gamma emission but the coefficient diverges towards infinity when $\lambda_\gamma$ reaches towards zero, which for instance is when he gamma transition is zero. In general, the coefficient increases $Z^3$, which will give a much greater coefficient for heavy nuclei than for lighter nuclei. In addition the coefficient decreases rapidly (ca. E$^{-2.5}$) with increasing transition energy. The multipole order also affects the coefficient, where a higher multipole order indicates a higher value. For higher atomic shells than the K shell (n=1) the coefficient decreases like $n^{-3}$ (\cite{KraneKennethS.Halliday1987} chapter 10, p. 346). \\

\noindent 
From a therapautic point of view, the most important process is the process which occurs after the release of the conversion electron. There is a vacancy is the shell following the emission of the atomic electron, and an electron from a higher shell or subshell fills  this vacancy. Radiative or nan-radiative processes can take place after to conserve energy \cite{Howell1992}. To conserve energy, an X-ray with the energy equal to the difference between the atomic states can be emitted, or that X-ray can interact electromagnetically with atomic electrons in same subshell, a higher subshell or shell \textcolor{red}{(remember shell: n=1,2.., subshell: spdf..)}. Dependent on where the ejected electron originated from, the electrons are called super Coster-Kronig, Coster-Kronig or auger electrons respectively. In practice the vacancy moves up to higher atomic shells and the result is a cascade of electrons and Auger electrons, until the reaction "fades out". Due to the low energies, they need to be located close to the cellular nucleus or incorporated into the DNA for induce damage (\cite{Vertes2011a}, p. 2203). When incorporated into DNA has they are equally almost effective as $\alpha$-emitters \cite{Howell1991} (\cite{Vertes2011a}, p. 2203). \\

The cellular nucleus is approximately 6 $\mu$m in diameter, while the thickness of DNA is approximately 2 nm. According to a simulation done by Howell (1992)\cite{Howell1992}, the ranges of the electrons from the decay are between 3.29 nm- 231 $\mu$m, which implies that the ranges are shorter than the cellular nucleus. According to this simulation, the isomer emits 26.4 Coster-Kronig and Auger electrons (energy released per decay is 10.353 keV) and 3 internal conversion electrons (energy released per decay is 126.738 keV). In addition, X-ray energy deposition of 12.345 keV  adds to the total energy deposition.  \textcolor{red}{The conversion electrons, contributes to dose?}. In this energy region, the energy loss is due to collisions and not bremsstrahlung. The electrons are deflected frequently due to the low mass, and the maximum energy loss is up to half of its kinetic energy per collision as described in equation \ref{eq:collision_E_transfer}. \\

There are multiple ways that this isomer can be produced, either in a neutron field in a reactor, or in a charged particle accelerator like a cyclotron, via the reactions $^{192}$Pt(n, $\gamma$) \cite{Qaim2017c} or $^{192}$Os($\alpha$, 3n) \cite{Hilgers2008}. One of the issues with production is that $^{193m}$Pt is difficult to produce with high specific activity \cite{Qaim2017c}, and the routes are not well investigated either. As described above, with reactors aging, there is a clear benefit of producing in a cyclotron. Otherwise, neutron induced reactions using for instance the UC Berkeley High Flux Generator \cite{Voyles2017c} can be investigated. To work with the highly toxic and difficult material Osmium is not desired, and in addition, the need of enriched target complicates the production \cite{Hilgers2009}. This work is a contribution to an investigation of a new production route, and the data obtained in this work is compared to previous experimental \cite{Tarkanyi2019, Tarkanyi2006a}. \\ 

\textcolor{red}{By itself, not useful for imaging.  191Pt and 195mPt can.  Can replace stable N with 13N, but thehalf life is so short that the radionuclide can not image the distribution it self, so not as a theranosticspair??  or does cisplatin distribute so fast within the body?}


\begin{comment}
\textcolor{red}{here write about gamma-decay and that the probability for M6 or whatever transition is improbable}. The populated isomer states decays from 149.8 keV to 14.3 keV releasing a 135.50 keV photon (0.1145475\%), from 14.3 keV to 1.6 keV releasing a 12.634 keV photon (0.70\%), and from 1.6 keV to the ground state releasing a 1.642 keV photon (0.0321). The photon abundance is thus low, and this isomer is not well suited for imaging. Due to the low intensity of the gamma-rays, it might be difficult to detect. There are X-rays too, but they overlap with other nuclei. Since the gamma-rays are weak, the IC probabilities are 99.89\%, 99.33\% and 99.99\% for each state respectively, calculated by subtracting 100 - gamma-intensity \cite{HowellRogerW.SastryKandulaS.R.HillHeleneZ.Rao1985}\footnote{http://citeseerx.ist.psu.edu/viewdoc/download?doi=10.1.1.987.2577&rep=rep1&type=pdf#page=506, p. 496}. This also indicates that the photon abundance is very low, as well high very high probability of low energy auger electrons. This also poses a challenge of observing this radionuclide using the emitted gamma-ray, which will later be shown.

In all decays, there are certain quantities in which needs to be conserved; angular momentum ($\ell$) and parity (maybe $\ell$ should be written as L instead??). Krane says that a multipole of order $\ell$ transfers angular momentum $\ell\hbar$ per photon (Krane, p. 333). A nuclear state has a definite angular momentum (angular momentum and spin) and parity, and if a gamma-transition is to happen between two states, the photon must connect the two states by conserving angular momentum and parity. In order for the quantity $\ell$ to be conserved, the angular momentum can be integer values between
\begin{equation}
|I_i-I_f| \leq \ell \leq I_i + I_f
\end{equation}
For the decay of $^{193m}$Pt (E level=149.8 keV) to the excited state (E level=14.3 keV), the spin and parity change is from $13/2^+$ to $5/2^-$, so $\ell=4,5,6,7,8,9$. The parity decides the wether the radiation is electric multipole or magnetic multipole (equations from Krane p. 331), 

\begin{equation}
    \pi(ML) = (-1)^{\ell+1}, \quad \pi(EL)=(-1)^{\ell}
\end{equation}
 
The electric decays have even parity when $\ell$=even, and magnetic has even when $\ell$ is odd. If parity is unchanged in the reaction ($\Delta \pi=$no), the electric multipoles are even and magnetic multipoles are odd. If the parity does change ($\Delta\pi=$yes), there would be odd electric and even magnetic multipoles. Hence the possible transition from $13/2^+$ to $5/2^-$ are whenever $\Delta \pi=$yes and $\ell=4,5,6,7,8,9$, which gives possible M4, E5, M6, E7, M8 or E9. 

In general, the lowest possible multipole dominates, and the emisssion of multipole of one order higher (L+1 than L), is reduced by a factor ca $10^{-5}$ (Krane p. 335, important!!). Thus, a multipole of order 4 or 5 has a low probability of occuring and thus the isomer has a long half life.  
In comparison to decay from isomer state, decay from $5/2^-$ to $3/2^-$ gives possible radiation, $\ell=1,2,3,4$, with no parity change, and $\Delta \pi=$no, gives possible M1, E2, M3, E4, and from $3/2^-$ to $1/2^-$ gives $\ell=1,2,3,4$, which also gives M1,E2, M3, E4. 

Half life: the decay rate constant is the sum of the decay rates of all the populates states  transitions, $\lambda=\lambda_{13/2^+} + \lambda_{5/2^-}+ ...$. 







\subsection{Gamma-decay and isomeric transition}
Gamma-decay is the lowering of the excitation energy by the release of a photon, with an energy $\Delta$E equal to the energydifference in the two states. The typical half lives of gamma-emission are less than $10^{-9}$ seconds, however, longer lived states of a nucleus which is not the ground state is called an isomer, and the gamma-decay of an isomer state is called isomeric transition (Krane, p. 175). Whenever gamma-decay is possible, another process called internal conversion is competing. It is an electromagnetic process, where the nucleus interacts electromagnetically with the atomic electrons, and an electron is emitted instead of the photon (Krane, chapter 10, p. 341). The kinetic energy of the emitted electron is the transition energy minus the electron binding energy

\begin{equation}
    T_e = \Delta E - B
\end{equation}

where B is a positive number (even though bound states are negative??). The electron is called a conversion electron, and this electron is high in energy and mathces the gamma-energy. 
The electron binding energy varies with the atomic orbital (Krane), and the electrons emitted following internal conversion are in a spectrum of different discrete energies. The transition energy must be higher than the electron binding energy, and as a consequence, the electron is labelled with the shell that it was emitted from. (remember, n=1=K, n=2=L, n=3=M, n=4=N) \\ 

In the case of the decay of $^{193m}$Pt, internal conversion is highly favoured instead of gamma-decay (the intensity of the gammas are very weak). The total decay probability is the summed decay probability for gamma-decay and internal conversion
\begin{equation}
\lambda = \lambda_\gamma + \lambda_{IC}    
\end{equation}

and the internal conversion coefficient $\alpha$ can be defined as
\begin{equation}
    \alpha = \frac{\lambda_{IC}}{\lambda_\gamma}
\end{equation}

High values of $\alpha$ indicates high probability of internal conversion, relative to probability of gamma-emisson, but the coefficient diverges towards infinity when $\lambda_\gamma$ reaches towards zero, which for instance is when the gamma-transition is zero. In general, the coefficient increases as $Z^3$, which will give a much greater coefficient for heavy nuclei than for lighter nuclei. In addition the coefficient decreases rapidly (ca. $E^{-2.5}$) with increasing transition energy. The multipole order also affects the coefficient, where a higher multipole order indicates a higher value. For higher atomic shells than the K shell (n=1), the coefficient decreases like $n^{-3}$ (Krane, chapter 10, p. 346). \\

\noindent 
In therapy, the most important process is the process which occurs after the release of the conversion electron. There is a vacancy in the shell where the conversion electron was emitted, and an electron from a higher shell drops down to this energylevel, with the release of an X-ray with an energy equal to the difference between the energy state of the two shells, $\Delta E$.\textcolor{red}{ If the transition is an electron from an L shell drops to K shell, and an electron from the L shell is ejected, the processes is called a KLL transition, and the energy of the auger electron is $E_{auger}=E_K - E_{LL}$ (Prasad A. Naik, in Encyclopedia of Spectroscopy and Spectrometry, 1999)\foonote{https://www.sciencedirect.com/topics/chemistry/auger-process}. If the vacancy is filled with an electron from the same shell (or subshell) but the ejected electron is from another shell, the electron is called a coster-kronig electron (like LLM, electron vacancy is moving from L to L and electron in M is emitted), and if the whole process occurs in the same shell, it is called a super coster-kronig process (MMM) }

The energy of the X-rays are lower in energy than the gamma-rays, typically. If one of the X-ray photon interacts within the atomic electrons (via photoelectric effect), the electron (which is called an auger electron) will be emitted with the energy of the X-ray minus the atomic binding energy  (Handbook of NUclear chemestry, p. 390)
\begin{equation} \label{eq:energy_auger}
    T_{a.e.}=\Delta E_{x-ray}-B
\end{equation}

From the vacancy from the auger electron, a new electron can take this place and release another X-ray. The auger electron can cause further ionizations in the atom, either by interaction it self, or from X-rays following the de-exication of another atomic electron by the vacancy. Thus it is possible to have a cascade of electrons and X-rays. The secondary electrons caused by the auger electron can lead to a cascade of new short-range electrons and X-rays, which are typically have ranges of nm in tissue (Handbook of nuclear chemistry p. 2203). Since the X-ray energy is in the low energy region, the auger electrons have low energies (from equation \ref{eq:energy_auger}).  

Since auger emitters are short range, they are very precise, and do only harm when bound to DNA or when incorporated into the cellular nucleus (handbook of nuclear chemistry, o. 2204), which means that no neightbooring cells will be affected. 

\textcolor{red}{After IC-process, vacancy is produced in an inner atomic shell (n) or subshell (like l=spdf). Vacancies in inner atomix orbitals are unstable, filled by electrons from higher energy levels. 4 processes, radiative X-ray transition, non-radiative transitions of auger, Coster-Kronig and super Coster-Kronig. move primary vacancies to higher shells or subshells. The non-radiative transitions involves multiplication of vacancies in the higher shells and subshels since two new vacancies are produced for each filled vacancy. Whenever energetically possible, super CS transitions dominate the other types. Thus the inner shell vacancies move upward to the valence and near valence shells of the atom, copious emission of electrons occur. Since the transition energies are very small for the higher shell transitions, the electrons ejected possess very small energies and is extremely short range (few nm) in biological matter, find a citation here, numb 8 in chapter. }
\end{comment}

\textcolor{red}{Energy loss of low E auger electrons. In this energy region, is due to collision loss, not bremsstrahlung. Deflects frequently due to low mass, and the max energy loss is $T_e/2$ per collision, as described in equation \ref{eq:collision_E_transfer}. }

%\newline
%General stuff $^{193m}$Pt: Cellular nucleus is approximately 6$\mu$m, while thickness of DNA is ca 2 nm (wikipedia). Range of the electrons from the decay is between 3.29nm-231$\mu$m, according to simulation done by Howell (1992) \cite{Howell1992}, so well within cellular nucleus. In its decay, it emits 26.4 coster-konig and auger electrons (energy realeased per decay: 10.353 keV) and internal 3 conversion electrons (energy released per decay: 126.738 keV). According to the simulation, an additive 12.345 keV is for X-ray energy deposition per decay. 

%Production: there are multiple ways that this isomer can be produced, either in a neutron field in a reactor, or in a charged particle accelerator like a cyclotron: $^{192}$Pt(n,$\gamma$) or via $^{192}$Os($\alpha$,3n). One of the issues with production is that 193mPt (and 195mPt) are difficult to produce with high specific activity \cite{Qaim2017c}, and are not well investigated. This study gives an examination of a new route. Many reasons, reactors are on their way out, and Osmium is a poisoneous and difficult target to work with, so using iridium as target is easy, (expensive though?) and the production of radionuclides below iridium is evidently in this work and in papers tarkanyi et al (2006,2019) low. 

%By itself, not useful for imaging. 191Pt and 195mPt can. Can replace stable N with 13N, but the half life is so short that the radionuclide can not image the distribution it self, so not as a theranostics pair?? or does cisplatin distribute so fast within the body? 



\begin{figure}
    \centering
    \includegraphics[width=0.5\textwidth]{Theory/cisplatin_DNA.png}
    \caption{The figure shows how cisplatin binds to DNA bases. \textcolor{red}{Try to find new figure, protein is not a part of this, but is a DNA repair protein which is a potentially molecular target for anticancer platinum drug cisplatin. }Figure is from \cite{Ratanaphan2011}. % https://www.researchgate.net/figure/Common-cisplatin-DNA-adducts-and-functions-For-instance-the-platination-of-human-serumfig2221919257[ }
    }
    \label{fig:cisplatin_DNA}
\end{figure}





%\section{General nuclear reaction theory}
\chapter{General nuclear reaction theory}

\begin{comment}
\textcolor{red}{paragraph based on special curriculum}
\textcolor{red}{need to rewrite this part as some of this is already mentioned above.. }
Medical radionuclides can be produced directly using charge particle (cyclotron) or neutron beams (reactors), or indirectly using radionuclide generators or fission (reactor). Medical radionuclides are typically produced in reactors, cyclotrons or by a longer lived-parent decaying into a short-lived daughter in a radionuclide generator system. In general, the production should be cheap, available. Today many radionuclides are only produced in reactors, which is the main source of neutrons, and with reactors aging (Chai Hong Yeong, Mu hua Cheng, and Kwan Hoong Ng. Therapeutic radionuclides in nuclear medicine: Current and future prospects. Journal of Zhejiang University: Science B,
15(10):845–863, 2014.), we need alternative routes to produce critical radionuclides. Cyclotrons have many benefits, like size so that it can be produced directly at the site of usage. One of the major disadvantages is that there is a need to enriched targets to get the desired reaction, and those can be very expensive. Along with high beam intensity the melting of the target can give challenges, so target cooling technqueis need to be there.    

In order to create isotopes, nuclear reactions need to occur. There are many different production routes available for a single radionuclide, which is dependent on multiple factors such as choice of target, incident particle-beam and beam energy. To each reaction route, there is an corresponding excitation function which tells us how probable the reaction channel is at various energies. The nuclear reaction data is very important for the optimization of the product, achieving minimal level of isotopic impurities and maximum yield (S M Qaim, R Capote, and F Tarkanyi. Nuclear Data for the Production of Therapeutic Radionuclides.
Trs 473, (473):395, 2011., p. 3). \\

Isotopic purity is important as it is impossible to separate isotopes of the same element \cite{Qaim2017c}. An undesired radionuclided can lead to undesired dose to healthy tissue, and a non-radioactive nuclide may lead to  poisoning (if large amounts injected), but it will not have any therapautic effect. This is especially important when working with poisoneos elements such as platinum. The only option to minimize isotopic impurities is to choose an appropritate energy window. 


Using charged particles instead of neutrons allows for measurement at multiple energies as the particle energy degrades in the foils. The neutron energy is not degraded in the same way, due to electric neutrality, thus can only give cross section at one single energy. 


\section{Radioactive decay law}
\textcolor{red}{where to place this? }

From here based on Krane chapter 6 \footnote{https://faculty.kfupm.edu.sa/phys/aanaqvi/Krane-Ch-6.pdf}

The activity of a nucleus is defined as the number of decayed nuclei per unit time of a radioactive product, which is equal to the radioactive decay rate 

\begin{equation} \label{eq:activity_decayrate}
   A =  \frac{dN}{dt}=-\lambda N
\end{equation}

where N is the number of nuclei, t is the time and $\lambda$ is the decay constant. Solving equation \ref{eq:activity_decayrate} gives number of decayed products at time t
\begin{equation} \label{eq:N(t)}
    N(t) = N_0 e^{-\lambda t}
\end{equation}

\noindent 
Since $N\propto$A, the relations $\frac{N_0}{A_0}=\frac{N(t)}{A(t)}$ are valid, and we can rewrite the equation \ref{eq:N(t)} to

\begin{equation} \label{eq:activity_decaylaw}
    A(t) = A_0 e^{-\lambda t}
\end{equation}


This accounts for single nucleus decaying into a daughter product, without anything first decaying into the parent nucleus. However it is common that a radioactive nucleus decays into another radioactive nucleus. Hence the daughter activity will increase due to feeding from the parent.
%The number of decayed nuclei N of nucleus i with a n-decay chain is then
%\begin{equation}
%    dN_i = \lambda_{i-n}N_{i-n}dt - ... -\lambda_{i-1}N_{i-1}dt - \lambda_iN_idt
%\end{equation}
For multiple decay, Bateman equation is used describing the activity in nucleus n of the decay chain \textcolor{red}{(Voyles2018, which article??)}

\begin{equation} \label{eq:ndecay_chains}
    A_n = \lambda_n \sum_{i=1}^n \Big[ \Big( A_{i,0}\prod^{n-1}_{j=i}\lambda_j \Big)\cdot \Big( \sum_{j=i}^n \frac{e^{-\lambda_j t}}{\prod_{i\neq j}^n (\lambda_i - \lambda_j)} \Big) \Big]
\end{equation}

where $A_n$ is the activity of nuclei n in the decay chain, with the corresponding decay constant $\lambda_n$. The equation sums over all nuclei in the decay chain. $A_{i,0}$ is the initial activity of nucleus i, and j is the nucleus which is feeding into nucleus i. 

\noindent 
If a target of stable nuclei is assumed, which is exposed to a particle beam which induces various nuclear reactions, the constant rate of production of a specific reaction is dependent on the number of target nuclei, the current of flux of the particle beam and the reaction cross section

\begin{equation}
    R = N_T \Phi \sigma
\end{equation}

\noindent 
where R is the production rate, $N_T$ is the number of target nuclei, $\Phi$ is the beam current or flux and $\sigma$ is the reaction cross section. In the assumption of the production rate being a constant value, the number of transformed target nuclei is small in comparison to the total number during the irradiation time. The number of produced nuclei from a specific reaction per unit time is thus thus the produced nuclei minus the decayed nuclei (activity)
\begin{equation}
    dN = Rdt - \lambda N dt
\end{equation}

which has the solution

\begin{equation}
    N(t) = \frac{R}{\lambda}(1-e^{-\lambda t})
\end{equation}

From equation \ref{eq:activity_decayrate}, the total activity produced during irradiation time t is thus 

\begin{equation} 
    A(t) = R(1-e^{-\lambda t}) = N_T \Phi \sigma (1-e^{-\lambda t})
\end{equation}

At the end of beam, the activity is denoted as $A_0$, and t is the irradiation time:
\begin{equation} \label{eq:activity_eob}
    A_0 = N_T \Phi \sigma (1-e^{-\lambda \Delta t_\text{irr}})
\end{equation}

\noindent 
When a target is irradiated, the activity of the product nucleus will increase until secular equilibrium is achieved, which is when the product rate and decay rate are constant. Hence it is not necessary to irradiate a target for more than 2-3 half lives.\\

\noindent 
If a spectrum is counted at a delay time $\Delta t_d$ after end of beam with a counting time $\Delta t_c$  the total number of decayed products are 

\begin{equation}
    N_D = \int_{\Delta t_d}^{\Delta t_d + \Delta t_c} A(t) dt
\end{equation}

Using equation \ref{eq:activity_decaylaw} for A(t), the solution to the above equation is 
\begin{equation} \label{eq:numb_of_decayed}
    N_D= \frac{A_0}{\lambda}e^{-\lambda \Delta t_d}(1-e^{-\lambda \Delta t_c})
\end{equation}

which again is equal to
\begin{equation}
    N_D = \frac{A(t)}{\lambda} (1-e^{-\lambda \Delta t_c})
\end{equation}

We can only know the number of decayed products which are detected. This is dependent on the efficiency of the detector, the intensity of the gamma-rays and the true number of decayed products

\begin{equation}\label{eq:Ngamma}
    N_C  = N_D \epsilon I_\gamma
\end{equation}

where $N_C$ is the number of observed/counted gamma-rays, $\epsilon$ is the efficiency of the detector and $I_\gamma$ is the gamma-ray intensity.\\ 

\noindent
Thus, we can obtain an expression for $A(t)$ after a delay time: 

\begin{equation} \label{eq:Final_Expression_A}
    A(t) = \frac{N_C \lambda}{\epsilon I_\gamma (1-e^{-\lambda \Delta t_c})}
\end{equation}

\noindent 
Again using \ref{eq:activity_decaylaw} for A(t), the above expression can be rewritten using $A_0$ and the delay time $\Delta t_d$

\begin{equation} \label{eq:Final_Expression_A0}
    A_0 = \frac{N_C \lambda }{\epsilon I_\gamma (1-e^{-\lambda \Delta t_c})e^{-\lambda \Delta t_d}}
\end{equation}
\end{comment}
\begin{comment}
Combining equation \ref{eq:activity_eob} and \ref{eq:numb_of_decayed}, total number of decayed products is

\begin{equation} \label{eq:finalN_D}
    N_D= \frac{N_T \Phi \sigma}{\lambda}(1-e^{-\lambda \Delta t_\text{irr}})\cdot e^{-\lambda \Delta t_d}\cdot (1-e^{-\lambda \Delta t_c})
\end{equation}

Combining equation \ref{eq:Ngamma} and \ref{eq:finalN_D}, we get the following expression 

\begin{equation}
    \frac{\epsilon I_\gamma}{N_C}=\frac{N_T \Phi \sigma}{\lambda}(1-e^{-\lambda \Delta t_\text{irr}})\cdot e^{-\lambda \Delta t_d}\cdot (1-e^{-\lambda \Delta t_c})
\end{equation}

From here, $A_0$ is the following equation \ref{eq:activity_eob}

\begin{equation}
    A_0 = \frac{\epsilon I_\gamma \lambda}{N_C}\cdot e^{-\lambda \Delta t_d}\cdot  (1-e^{-\lambda \Delta t_c})
\end{equation}





%\begin{equation}
%    \sigma = \frac{A_0 }{N_T \Phi (1-e^{-\lamda \Delta t_\text{irr}})}
%\end{equation}

\begin{equation}
 \Phi = \frac{A_0 }{N_T \sigma (1-e^{-\lamda \Delta t_\text{irr}})}    
\end{equation}


\end{comment}

%In a detector, we can only know the number of decayed products from which are registered. Whether the detector detects or not is dependent on efficiency of detector, the intensity of the gamma-rays and the true number of decayed products

%\begin{equation}\label{eq:Ngamma}
%    N_\gamma  = N_D \epsilon I_\gamma
%\end{equation}

%where $N_\gamma$ is the number of observed gamma-rays, $\epsilon$ is the efficiency of the detector and $I_\gamma$ is the gamma-ray intensity. Combining equation \ref{eq:finalN_D} and \ref{eq:Ngamma}, the get the following expression 

%\begin{equation}
%    \frac{\epsilon I_\gamma}{N_\gamma}=\frac{N_T \Phi \sigma}{\lambda}(1-e^{-\lambda \Delta t_\text{irr}})\cdot e^{-\lambda \Delta t_d}\cdot (1-e^{-\lambda \Delta t_c})
%\end{equation}


\section{Nuclear reactions and reaction cross sections}

A nuclear reaction occurs when a collision between two nuclei or a nucleus and a subatomic particle takes place. A nuclear reaction is denoted as
\begin{equation}
    X(a,b)Y
\end{equation}

\noindent where X is the target, a is the incoming projectile, b is the outgoing decay channel and Y is the product of the nuclear reaction (\cite{KraneKennethS.Halliday1987}, chapter 11.1). There are multiple processes which can occur, radiative capture is the process where a particle is captured and a $\gamma$-ray is emitted in a (x,$\gamma$) process. If the incoming and outgoing particle is the same, it is a scattering process, where elastic scattering leaves the target nucleus in the energy same state, and inelastic if the target nucleus is in an excited state. In addition, fission can occur, but not for the reactions in this work. In these types of experiments however, we are interested absorption of the incoming particle, and emission of particles to create products in which we can measure the reaction cross section.  \\

In a nuclear reaction, the total energy and linear momentum, proton and neutron number, angular momentum and parity are conserved quantities (assuming no meson formation) (\cite{KraneKennethS.Halliday1987}, p.380). A nuclear reaction can be a compound-reaction, pre-equilibrium-reaction or direct-reaction\cite{KoningA.J.Akkermans1999}. The compound nucleus model (Bohr, 1936) \cite{Bohr1936} is a model which describes the formation of a compound nucleus by absorption of an incoming projectile by a nucleus close enough to interact with the strong nuclear force, and the decay of the compound nucleus. The kinetic energy shared between the incoming projectile and the nucleon which was struck leads to multiple collisions with other nucleons and rapid exchange of energy. The energy is distributed throughout the nucleus, leaving the original nucleus in an highly excited state. There is a statistical probability that enough energy will be be concentrated on one single nucleon or group of nucleons such as protons, neutrons and alpha-particles, and escape the potential barrier of the nucleus (also refereed to as evaporation), which lowers the excitation energy \cite{KraneKennethS.Halliday1987}, chapter 11.10, p. 416). Compound nuclear reactions have an interaction time of ca. $10^{-18}$ seconds, achieving thermal equilibrium \cite{KoningA.J.Akkermans1999}. Since the nucleons collides rapidly, the information regarding the initial energy and the direction of the incoming particle is lost, and the outgoing decay channel depends only on conservation of the energy and angular momentum. As shown in figure \ref{fig:Nuclear_reactions}, where a $^{10}$B nucleus interacts with a deuteron ($^2$H), compound nuclear reactions can be divided into two stages; fusion of incoming particle and target nucleus, and evaporation of nucleons or groups of nucleons. This is decay of the compound nucleus, and this will lower the excitation energy. Compound nuclear reactions dominates in the low energy region (below 10 MeV) \textcolor{red}{cant find a citation on this.. Andrew: you said that direct reactions take place above 200 MeV, but \cite{KoningA.J.Akkermans1999} says that direct reactions dominates over 10 MeV and compound under 10 MeV}. \\ %where an incoming particle merges with a target nucleus by sharing the kinetic energy over all nucleons. This results in rapid collisions of the nucleons, achieving thermal equilibrium in the compound nucleus. 
The contrary, direct reactions have interaction time of ca, $10^{-22}$ seconds, and dominates at high energies, above 10 MeV. The reaction involves interaction with single nucleons, and can be elastic, transfer or break up, as shown in figure \ref{fig:Nuclear_reactions}. Since the incoming particle interacts with a single nucleon, the exit channel depends heavily on conservation of spin and parity, in addition to energy and mass. An intermediate reaction between compound and direct reactions are the pre-equilibrium reactions, where the incoming particle does not lead to a thermal equilibrium, but the information of initial energy and direction is lost. This is an important contribution for energies between 10 and more than 200 MeV \cite{KoningA.J.Akkermans1999}. In this work, compound nuclear formations and pre-equilibrium are observed, with a compound peak where one more reactions constributes to a peak, and the pre-equilibrium tale of the high energy side of the compound peak.   \\

\begin{figure}
    \centering
    \includegraphics[width=0.9\textwidth]{Theory/Direct-and-compound-nucleus-reaction-channels-taking-place-in-a-d-10-Be-reaction.png}
    \caption{The figure shows how direct and compound nuclear reactions differ. Figure is from \cite{Moro2018}. %https://www.researchgate.net/figure/Direct-and-compound-nucleus-reaction-channels-taking-place-in-a-d-10-Be-reaction_fig8_326366718 [accessed 13 May, 2020]}
    }
    \label{fig:Nuclear_reactions}
\end{figure}

A Nuclear reaction cross section represents the total probability that a compound nucleus will be formed and that it decays by a certain decay channel with particle emission. A way to visualize a nuclear reaction cross section is by imagining the incoming particle and the target nucleus as spheres, and if they overlap, the reaction will occur. The total reaction cross section is therefor proportional to the cross-section area of the two spheres
\begin{equation}
    \sigma_R = \pi r_0^2 (A_X^{1/3} + A_a^{1/3})^2
\end{equation}
where $r_0$ is a constant ($\approx$ 1.6 fm), $A_X$ and $A_a$ are the atomic mass number of target A and incoming particle a, and the radii of the nuclei are connected via $r_0A^{1/3}$. The unit of a cross section is therefor in a unit of area, typically in millibarn ($1\cdot 10^{-27}$cm$^2$) \cite{IAEA2009}, p. 8. The total reaction cross section between the incoming particle and the target nucleus is a more or less constant value, with the summation of the various nuclear reactions which can take place
\begin{equation}
    \sigma_\text{tot} = \sum_b \sigma(a,b)
\end{equation}
The outgoing decay channel can be inelastic and elastic scattering X(a,a)Y and X(a,a$^*$)Y, radiative capture X(a,$\gamma$)Y and absorption and compound nucleus formation X+a$\rightarrow C^* \rightarrow$Y +b.   

\textcolor{red}{(\cite{international2012iaea}, p. 23)}. The cross section for a nuclear absorption reaction to occur via the formation of the compound nucleus by the collision of a projectile with a target nucleus, and the decay of the compound nucleus via decay channel b can be expressed as
\begin{equation}
    \sigma(a,b)=\frac{\sigma_c P_b}{\sum_b P_b}
\end{equation}
where a is incoming projectile and b is decay channel, $\sigma_c$ is the formation of compoubd nucleus CS, and $P_b$ is the probability of compound nucleus decaying into channel b. In general, product produced via emission of neutrons and protons has the highest probability, but particles such as alpha, tritons, deuterons and $^3$He can occur before because of the lower binding energy per nucleon, which will be described in the next section. 


The cross section of a certain reaction (in a thin target) to take place can be estimated based on the production rate, the number of target nuclei, the beam flux or beam current (\cite{KraneKennethS.Halliday1987}, chapter 6)

\begin{equation}
    \sigma(E)=\frac{R}{N_T \cdot \Phi}
\end{equation}

Based on the induced activity in the target, the irradiation time and decay product nuclei during irradiation, the cross section can be find experimentally, with gamma-ray spectroscopy, which is used in this work: 
\begin{equation}  \label{eq:crossSection_simple}
    \sigma(E) = \frac{A_0 \cdot t_\text{irr}}{N_T \cdot \Phi(E)(1-e^{-\lambda t_\text{irr}})}
\end{equation}
\noindent where $A_0$ is the end of beam activity of the resulting product nucleus (Y), $t_\text{irr}$ is the irradiation time, $N_T$ is the number of target nuclei (X), $\Phi(E)$ is the projectile flux or current (a), and $\lambda$ is the decay constant of the product nucleus. \\ 

For each possible decay channel of the compound nucleus, there is an associated probability or cross section, which is dependent on the energy of the incoming projectile. A function which evaluates the various cross sections at different energies is called an excitation function. In figure \ref{fig:pt_reactionchannels}, the excitation function of the reactions channels for the platinum isotopes $^{188, 189, 191,193m}$Pt resulting from deuterons on natural iridium is plotted (using TENDL-2019 nuclear reaction code \textcolor{red}{cite}). Natural iridium consists of two stable isotopes, $^{191}$Ir (37.3\% abundance) and $^{193}$Ir (62.7\% abundance). $^{193m}$Pt can only be produced from $^{193}$Ir, ejecting 2 neutrons in the process, which can be denoted as $^{193}$Ir(d,2n)$^{193m}$Pt ($^{193}$Pt is the compound nucleus formation of deuteron on $^{191}$Ir, which has a low production cross section). The other platinum isotopes can be produced as $^{191}$Ir(d,2n)$^{191}$Pt or $^{193}$Ir(d,4n)$^{191}$Pt, $^{191}$Ir(d,4n)$^{189}$Pt or $^{193}$Ir(d,6n)$^{189}$Pt and $^{191}$Ir(d,5n)$^{188}$Pt or $^{193}$Ir(d,7n)$^{188}$Pt. For each reaction route possible, there is a resulting compound peak, hence, $^{193m}$Pt has only one peak, and the other platinum isotopes has two. In addition, the long exponential tale which can be seen for $^{193m}$Pt is the pre-equilibrium reaction. The desired particle emission is energy dependent, and the higher energy provided to the compound nucleus, the probability that more particles will be emitted is higher (\cite{KraneKennethS.Halliday1987}, chapter 11.10, p. 419). When a specific isotope is desired, the excitation function can tell us which energy window that maximizes the production and most importantly minimizes particularly other isotopes of the same element, due to the difficulty of separating same chemical elements. This is why it is important to find which window that maximizes the production. \\

\begin{figure}
    \centering
    \includegraphics{Theory/reactionchannels_pt.png}
    \caption{Reaction cross sections provided by Tendl for the reactions $^\text{nat}$Ir(d,x)$^{188,189,191,193m}$Pt}
    \label{fig:pt_reactionchannels}
\end{figure}

%\begin{comment}
\textcolor{red}{talys 1.9 in reader can contain important info!}




 
\begin{comment}

In the low energy-region in which isotope production typically takes place (\textcolor{red}{<80 MeV}?), compound nucleus reactions take place, where an incoming particle and target nucleus merges by sharing the kinetic energy on all nucleons, and particle emission takes place to reduce the excess energy. \textcolor{blue}{\footnote{blue text:https://web-docs.gsi.de/~wolle/TELEKOLLEG/KERN/LECTURE/Fraser/L24.pdf }Involves nucleon nucleon interactions, lead to a complete thermal equlibrium inside the CN. Releases energy by emission of neutrons, protons, alpha particles and gamma rays. A consequence of equilibrium is that the decay of CN should not depend on the way it was formed. "forgets" in all the collisions. Consequently, the decay of the compound nucleus depends only on the mass and atomic numbers, excitation energy and angular momentum.} The contrary are direct reactions, where an incoming particle interacts (over such a short time period) so that the incoming particle only interacts with one single nucleon, typically on the surface of the target nucleus (thus probably in high nucleor shells, with high spin). \textcolor{blue}{Angular distributions of direct reaction products are sensitive to the momentum transfer and parity change during the reactions. Thus based on the selection rules from angular momentum and parity conversion the angular distribution measurements in direct reactions yield spin and parities of states populated in the exit channel}. \textcolor{red}{Write abot feeding to the compound peak???. So in general; emission of protons and neutrons are more fed, because the probability of emitting one single nucleons is easier for the system. Since the reaction forgets the incoming projectile, and interacts with the whole nucleus, the prob of emission of t, 3He and d is lower, and the binding energy does not do that the channel is more fed, its only a lower Energy threshold. For alpha particles however, the binding energy which is about 28 MeV lowers the energy quite a lot, therefor favourable if Coulomb barrier is low enough?} \\

%A nuclear reaction in the energy ranges of isotope production via the Compound nucleus conserves quantities such as mass-energy, linear momentum, the total number of nucleons, angular momentum and parity \textcolor{red}{cite, Krane?}. 

\noindent 
The cross section for a reaction can be divided into the cross section of the formation of the compound nucleus via interaction with the incoming projectile a, and the probability that the compound nucleus decay by decay channel b. The total reaction cross section is thus the sum of all the different reaction channels (Handbook of nuclear chemistry, p. 157 (nuclear reactions)), 
\begin{equation}
    \sigma = \sum_b \sigma(a,b)
\end{equation}
where b can be multiple particles. 


\textcolor{red}{(\cite{international2012iaea}, p. 23): Compound nuclear model: A nuclear reaction can be viewed in two stages, the formation of compound nucleus by the collision of a projectile with a target nucleus, and the decay of the compound nucleus into pairs of reaction products. Can be expressed:
\begin{equation}
    \sigma(a,b)=\frac{\sigma_c P_b}{\sum_b P_b}
\end{equation}
where a is incoming projectile and b is decay channel, $\sigma_c$ is the formation of compoubd nucleus CS, and $P_b$ is the probability of compound nucleus decaying into channel b. }
\noindent The compound nucleus model (Bohr, 1936) is a model which describes the formation of a compound nucleus by absorption of an incoming projectile by a nucleus close enough to interact with the strong nuclear force, and the decay of the compound nucleus. The kinetic energy shared between the incoming projectile and the nucleon which was struck leads to multiple collisions with other nucleons and rapid exchange of energy. The energy is distributed throughout the nucleus, leaving the original nucleus in an highly excited state. The average energy per nucleon is not sufficient to overcome the binding energy of the nucleus, but due to the statistical distribution in energies there is a probability that one or more nucleons may get sufficient energy to escape the nuclear potential (Krane, chapter 11.10, p. 416). This is decay of the compound nucleus, and this will lower the excitation energy. We can include the formation of the compound nucleus in the nuclear reaction as \begin{equation}
    X + a \rightarrow C^* \rightarrow Y + b
\end{equation} where $C^*$ is the excited compound nucleus (Krane, chapter 11.10, p. 416)  \\

\noindent For each possible decay channel of the compound nucleus, there is an associated probability or cross section, which is dependent on the energy of the incoming projectile. A function which evaluates the various cross sections at different energies is called an excitation function. In figure \ref{fig:pt_reactionchannels}, the excitation function of the reactions channels for the platinum isotopes $^{188, 189, 191,193m}$Pt resulting from deuterons on natural iridium is plotted (using TENDL nuclear reaction code \textcolor{red}{cite}). Natural iridium consists of two stable isotopes, $^{191}$Ir (37.3\% abundance) and $^{193}$Ir (62.7\% abundance). $^{193m}$Pt can only be produced from $^{193}$Ir, ejecting 2 neutrons in the process, which can be denoted as $^{193}$Ir(d,2n)$^{193m}$Pt ($^{193}$Pt is the compound nucleus formation of deuteron on $^{191}$Ir, which has a low production cross section). The other platinum isotopes can be produced as $^{191}$Ir(d,2n)$^{191}$Pt or $^{193}$Ir(d,4n)$^{191}$Pt, $^{191}$Ir(d,4n)$^{189}$Pt or $^{193}$Ir(d,6n)$^{189}$Pt and $^{191}$Ir(d,5n)$^{188}$Pt or $^{193}$Ir(d,7n)$^{188}$Pt. For each reaction route possible, there is a resulting compound peak, hence, $^{193m}$Pt has only one peak, and the other platinum isotopes has two. The desired particle emission is energy dependent, and the higher energy provided to the compound nucleus, the probability that more particles will be emitted is higher (Krane, chapter 11.10, p. 419). When a specific isotope is desired, the excitation function can tell us which energy window that maximizes the production and most importantly minimizes particularly other isotopes of the same element, due to the difficulty of separating same chemical elements. \\

\begin{figure}
    \centering
    \includegraphics{Theory/reactionchannels_pt.png}
    \caption{Reaction cross sections provided by Tendl for the reactions $^\text{nat}$Ir(d,x)$^{188,189,191,193m}$Pt}
    \label{fig:pt_reactionchannels}
\end{figure}
\end{comment}
\begin{comment}
\begin{figure}
    \centering
    \includegraphics[width=12cm]{Ir(d)Pt.png}
    \caption{Nuclear chart for the platinum isotopes which can be produced from natural iridium }
    \label{fig:chart_irpt}
\end{figure}
\end{comment}


%\noindent A nuclear reaction
%The nucleus is built up on protons and neutrons, and these are bound through the strong nuclear force. 
\subsection{Constraints in nuclear reactions}
%\subsection{Energetic factors in nuclear reactions}

%\subsection{Dependence of }

%\noindent The binding energy depends on multiple parameters, which are based upon two models, the shell model and the liquid drop model (Krane, chapter 3.3, p. 68). The binding energy thus depends on the volume, which is constant throughout the nucleus, hence a volumeterm $a_V\cdot A$, a the surface of the nucleus which needs to be taken into account, since the nucleons on the surface is less tightly bound, $a_s\cdot A^{2/3}$. 
The potential energy of a nucleus is the sum of the attractive well from the strong nuclear force and the repulsive Coulomb barrier which acts repulsive between charged particles and the nucleus, acting long range (p. 152,\cite{Vertes2011a}). The radius of the potential well is up to a few femtometer. For a positively charged particle induced nuclear reaction, the energy of the particle should exceed the barrier, or there will be an elastic scatter. However, there is a chance of tunneling, which drops with a factor 1/r where r is the distance from the center of the nucleus (\cite{Vertes2011a}, chapter 3 - Nuclear Reactions, section, 3.2.3). The barrier also constraints the emission of particles for a decay channel of the compound nucleus, as the energy for an outgoing decay channel of positive particles must exceed the barrier. %There is also a centrifugal barrier, which is dependent on the orbital angular momentum of the the nucleus.  However, this barrier is more important in \textcolor{red}{direct reactions??}

The height of the Coulomb barrier is dependent on the radius and charge of the incoming or outgoing particle a and the target nucleus X.

\begin{equation} \label{eq:Coulomb_barrier}
    U_\text{Coulomb} = \frac{1}{4\pi \epsilon_0} \frac{e^2Z_X Z_a}{r_X + r_a}
\end{equation}

In addition, there is a centrifugal barrier, which can constraint some of the incoming particle energy in rotational energy, \textcolor{red}{which depends on the angular momentum of the incoming particle and and the nucleus???} (\cite{Vertes2011a} p. 155.). 

\begin{equation}
    U_\text{centrifugal} = \frac{\hbar \ell (\ell+1)}{r^2}
\end{equation}

The sum of the barriers are the total barrier but the Coulomb barrier is the most important. \\

%The orbital angular momentum is dependent on the mass and energy of the particle, but also the impact parameter of the reaction, which is the "distance" between the projectile and the nucleus

%\begin{equation}
%    \ell\hbar = (mv)b, \quad b\leq r_a + r_X
%\end{equation}

%\noindent where $r_a$ and $r_X$ are the radii of projectile and target nucleus. If $\ell=0$, the impact parameter is zero. If $\ell\neq 0$, then some of the energy of the projectile will go to rotational energy of the nucleus (Handbook of Nuclear Chemistry, chapter 3 - Nuclear Reactions, section, 3.2.4). The probability of this drops with $1/r^2$, and if the impact parameter is large, this will constraint some of the energy that would have gone to the nuclear reaction. Figure \ref{fig:sumOfPotentials} shows an overview of the various potentials; centrifugal barrier, Coulomb barrier and nuclear potential. The figure is for $^{58}$Zn, but the figure is only an example. \textcolor{red}{maybe make figure self?}. 


\begin{figure}
    \centering
    \includegraphics[width=8cm]{Theory/Coulomb_barr.png}
    \caption{The Coulomb barrier decreases exponentially, and contributes to a constraint in a nuclear reaction, since the energy of the particles must have a certain energy to interact. Figure is from \cite{Ongena2018a}. }
    \label{fig:Coulomb_barrier}
\end{figure}



\begin{comment}
\noindent The separation energy is the energy required to remove a particle from the nucleus. That is the difference in the binding energy for the two nuclei. The neutron separation energy is 
\begin{equation}
    S_n = B(^A_z X_n)-B(^{A-1}_z X_{n-1}) = c^2(m(X^{A-1}_zX_n)-()
\end{equation}\\ 
\end{comment}

In a nuclear reaction, the mass-energy is conserved, which is denoted as the Q-value. The reaction Q-value is the difference is masses between before and after the nuclear reaction occurred (\cite{KraneKennethS.Halliday1987}, chapter 11.2). It is defined as 

\begin{equation} \label{eq:Q_value}
    Q = (m_i - m_f)c^2 = (m_X + m_a - m_Y - m_b)c^2
\end{equation}

where $m_i$ is the initial mass, $m_f$ is the final mass and c is the speed of light. If $Q>0$, then the reaction is exoergic, which means that energy is released in the reaction. There is no threshold energy of the projectile required for the reaction to occur, if only the projectile is present the reaction can occur. If $Q<0$, then the reaction is endoergic, which means that the kinetic energy of the incoming projectile is converted into nuclear mass or binding energy. For endoergic reactions to occur, there is a minimum threshold energy of the projectile in order for the reaction to happen, which is defined as (\cite{KraneKennethS.Halliday1987}, 11.2, p. 382)

\begin{equation} \label{eq:reaction_threshold}
    E_\text{threshold} = (-Q) \cdot \frac{m_Y +m_b}{m_Y + m_b -m_a}
\end{equation}


The energy threshold thus depend on the Q-value, the Coulomb barrier can constrain charged particle-reactions, and the centrifugal barrier can constrain reactions if angular momentum $\ell\neq 0$. \textcolor{red}{The parity though depend, even numbers of $\ell$ mix with even, and odd with odd (\cite{Vertes2011a}, chapter 3Nuclear Reactions, section, 3.2.3)??}. This gives an indication on when a reaction can energetically occur, but does not tell us how probable the reaction is. \\%Equation \ref{eq:reaction_threshold} indicates that a higher mass for the particle in the outgoing channel $m_b$, will lower the energy threshold.\\ 

%The nuclear binding energy is a number which tells us how tightly bound the nucleus is, and how much energy is required to separate nucleons from the nucleus. The nuclear binding energy is related to the mass of the nucleus, through the famous equation   E=mc^2

The binding energy is the mass-difference between the nucleus as a whole, and the number of protons and neutrons added
\begin{equation} \label{eq:Binding_energy1}
    B = c^2(z\cdot m_p + n \cdot m_n - m_N)
\end{equation}
\noindent where z is the number of protons, n is the number of neutrons, $m_p$ is the proton mass, $m_n$ is the neutron mass, $M_N$ is the mass of of the nuclide, which is the number of nucleons A minus the number of electrons, $M_P = m_A - z\cdot m_e$ (the electronic binding energy per electron is excluded). From Krane's derivation of the nuclear binding energy (\cite{KraneKennethS.Halliday1987}, chapter 3.3, p. 65). \\

From equation \ref{eq:Q_value}, the larger the mass of the outgoing decay channel, the more negative the Q-value will be. Protons (q=+1) and neutrons (q=0) are the simplest decay channels of the compound nucleus, each carry a spin of $1/2$, with masses $m_p=938.28$ MeV/c$^2$, and $m_n=939.57$ MeV/c$^2$ respectively. Combinations like deuterons (d=1p+1n, q=+1) has a mass difference of $\Delta=$2.2 MeV/$c^2$ from realising 1 proton and 1 neutron separately, a triton (t=2n+1p, charge +1) with $\Delta=8.5$ MeV/$c^2$,  3-Helium ($^{3}$He=1n+2p, charge +2) with $\Delta=7.7$ MeV/$c^2$ and alpha-particle ($\alpha$=2n+2p, charge +2) with $\Delta=28.3$ MeV/c$^2$. Thus, Q-values are higher in value, the lighter the particle is. However, in this work, we can clearly see that protons, neutrons and alpha-particles are strongly fed decay channels, while the other decay channels does occur but are mostly not observed due to the low cross section. % The suggested reason for this is that due to \textcolor{red}{blablabla nuclear physics stuff, like shell structure}, protons and neutrons are favoured, but since the alpha-particle has such a large binding energy, this channel is also favoured.  

\begin{comment}
\subsection{Deuterons and stopping power}

The deuteron consists of a neutron and a proton, and is the simplest bound state of nucleons. Nucleons have an average binding energy per nucleon of 8 MeV. The detuteron with an observed mass value of 2.224 MeV (Krane, p. 81) is a weakly bound. Thus little energy required to break up the deuteron. \textcolor{red}{does this affect?}

The stopping power of a deuteron beam running through forms the Anderson \& Ziegler. \textcolor{red}{write about Anderson and Ziegler. And how does the stopping power give a flux??}


(Techinque nuclear and particle physics p. 30-31):  
For a particle beam the energy loss is not a continuous process, but collisions based on statistics. A measurement of identical particles will thus show a statistical distribution of ranges centered about the same mean value. This is called range straggeling \textcolor{red}{this part relevant for describing Ziegler flux?}. 

Energy straggling: the energy loss distribution: (instrumentation p. 49)
"For any given particle however, the energy lost will not be equal to this mean value because of statistical fluctuations which occur in the number of collisions suffered and in the energy transferred in each collision. An initially monoenergetic beam will therefor show a distribution of energy rather than a delta function peak shifted down by the mean energy loss given by the dE/dx formula after passing through a fixed thickness of material.. see if more necessary?"
\end{comment}
\begin{comment}
Range: How far will particles penetrate before they lose all there energy. Moreover, if assume that the energy loss is continous, this distance must be a well defined number, the same for alll identical particles with the same initial energy in the same type of material. This quality is called the range of the particle, and depends on the type of material, the particle and its energy. Experimentally the range can be determined by passing a beam of particles at the desired energy through different thicknesses of the material in question and measuring the ratio of transmitted to incident particles. For small thicknesses all the particles manage to pass through. As the range is approached this ratio drops. The surprising thing however is that the ratio does not drop immediately to the background level as expected of a well defined quantity. Instead the curve slopes down over a certain spread of thicknesses. This result is due to the fact that the energy loss is not continous, but statistical in nature. Indeed two identical particles with the same iniitial energy will not in general suffer the same number of collisions and hence the same energy loss. A measurement with an ensemble of identical particles therefor will show a statistical distribution of ranges cented about same mean value. This phenomenon is known as range straggling. In a first approximation this distribution is Gaussian in form. The mean value of the the distribution is known as the mean range and correspond to the midpoint of the corresponding slope. This is the thickness at which roughly half of the particles are absorbed. More commonly however what is desired is the thickness at which all the particles are absorbed, in  which case the in which case the point at which the curve drops to the background level should be taken.  This point is usually the tagent to the curve at the midpoint and extrapolating to the zero level. This value is known as the extrapolated or practical range
\end{comment}


\section{Nuclear reaction models}

The optical model (proton/neutron, and alpha/deuteron), gamma strength function. \textcolor{red}{talys 1.9 in reader can contain important info! p. 525}. Optical, gamma-ray strength, compound and preequilibrium. Also read the introduction. 

Theoretical nuclear reaction models are important, for both the general understanding of the physics in the data, and to estimate the required cross sections in cases where the data does not make sense or is not available(\cite{international2012iaea}, p. 23). In addition, can weight contradictive measurements. 

\subsubsection{EMPIRE 3.2.3}
\subsubsection{CoH 3.5.3}
\subsubsection{ALICE 2017}
\subsubsection{TALYS 1.9}
\subsubsection{TENDL 2019}



%, the equal number of protons and neutrons make up $z\cdot^1H$, thus we can rewrite equation \ref{eq:Binding_energy1} to 

%\begin{equation}
%    B = z\cdot m_{^{1}H} + n\cdot m_n - m_{A}
%\end{equation}

%\noindent 
%Separation energy of a particle is the energy required to be removed from the nucleus. This is defined as the difference in binding energy between the nucleus with the particle and the nucleus without the emitted particle. Since heaavier particles such as tritrium has a lower binding energy, more energetically easy to separate. n and p require more. 



\begin{comment}

\subsection{Nuclear cross sections}

The cross section for a reaction can be divided into the cross section of the formation of the compound nucleus via interaction with the incoming projectile a, and the probability that the compound nucleus decay by decay channel b. The total reaction cross section is thus the sum of all the different reaction channels, 
\begin{equation}
    \sigma = \sum_b \sigma(a,b)
\end{equation}
where b can be multiple particles. The general equation which is used to calculate cross sections in this experiment is the following equation, which is solved from equation \ref{eq:activity_eob} (referenced to in the later parts of the paper)

\begin{equation} \label{eq:cross_section_equation}
    \sigma(E)=\frac{A_0}{N_T\Phi(E)(1-e^{-\lambda \Delta t_\text{irr}})}
\end{equation}

not, need to check if this is the correct one.. 
\begin{equation}
    \sigma(E) = \frac{A_0 \cdot t_\text{irr}}{N_T \cdot \Phi(E)(1-e^{-\lambda t_\text{irr}})}
\end{equation}
\noindent where $A_0$ is the end of beam activity of the resulting product nucleus (Y), $t_\text{irr}$ is the irradiation time, $N_T$ is the number of target nuclei (X), $\Phi(E)$ is the projectile flux (a), and $\lambda$ is the decay constant of the product nucleus. 

\newpage

Why we are interested, why is there a need for nuclear production cross sections. 



The Q-value for nuclear reactions 

Compound nucleus, what happens. 
Q value, binding energy, particle emission, probabilities (why does n, p/alpha emission require higher energy, but is more favoured than T emission if both possible.)


\subsection{Expected products from deuterions on iridium, iron, cupper and nc}

Include figure of different nuclear charts, and write about Q values, and reaction channels, which makes it more probable that we observe it (eg. when neutron, proton/alpha channels opens).





\section{From book}

In a nuclear reaction, the energy, linear momentum, the total number of protons and neutrons, angular momentum and conservation of parity is conserved. \\

The compound nucleus; the collisions are described by a model called Multistep Compound Model (Feshback, Kerman and Koonin, 1980), which is the model that reaction modeling code EMPIRE and TALYS rely on. \textcolor{red}{find out what ALICE does ???}

Products of a nuclear reaction can be anything than conserve the various variables. Mass energy conservation is the Q-value, which is the mass differene in the reactant nuclei (projectile and target in this case) and product nuclei (decay channel particles and product). Binding energy (nuclear mass energy) often denoted as $\Delta=(M_0 - Au)c^2$, where A is atomic mass number, u is atomic mass unit. Q-value can be rewritten: 
\begin{equation}
    Q = \Delta_\text{projectile} + \Delta_\text{target} - \sum \Delta_\text{products}
\end{equation}

Charged particle reactions with heavy nuclei which is the case here, can have positive or negative Q-value. If the Q-value is negative, energy is required to induce the reaction. The reaction is negative if the sum of products is less than the sum of reactants. The excitation energy of the product nucleus is the extra energy which is deposited in the product nucleus. The decay of this is often times $\gamma$-rays (?). If the excitation energy is lower than the binding energy of the nucleus?  Threshold is the minimum energy required for the incoming projectile to form a product nucleus, which is then in its ground state. 

\begin{equation}
    \text{mass-energy}: \quad \Delta_t+ \Delta_p + E_p 
\end{equation}




\section{The decay of the compound nucleus}
At the energy ranges in which this work is operating in, the nuclear reaction is caused by a fusion between the incoming projectile to create an highly excited compound nucleus. \\ 

\noindent 
The decay favoured decay channel is dependent on multiple properties such as the charge and angular momentum. The coulomb potential is defined as 

\begin{equation}
    U_{c} = K\frac{z_1 z_ 2 e^2}{A_1^{1/3} + A_2^{1/3}}
\end{equation}

where z is the the number of charges, e is the electrical charge and $A^{1/3}$ is the nuclear radius which is equal to $r=r_0 A^{1/3}$. K is a constant value. 

The centrifugal barrier (which is the amount of energy which is conserved in rotational energy) depends on the angular momentum 

\begin{equation}
    U_{s} = \frac{\hbar \ell(\ell+1)}{2\mu r^2}
\end{equation}

where $\mu$ is the reduced mass $u\cdot \frac{A_1 A_2}{A_1 + A_2}$. 


\textcolor{red}{Table \ref{tab:decaychannel_particles} might be wrong in the sense $J^\pi$. Might just be the total spin which is of interest? But since $J=l+s$, the spin will give a higher intrinsic angular momentum. Since $|I_i - I_f| \leq \ell \neq I_i+I_f$, $\ell$ can be multiple values, and the parity is thus dependent on the value here. (Krane, chapter 8.5, p. 257). The centrifugal potential is thus} 

\begin{equation}
   U_{0,s} =  \frac{\ell (\ell +1)\hbar}{2mr^2}
\end{equation}

and the Coulomb potential is

\begin{equation}
    U_{0,c} = K\frac{z_1 z_2 e^2}{A_1{1/3}+ A_2^{1/3}}
\end{equation}

The 

\begin{table}[]
    \centering
    \caption{The table shows the most common decay channels for the decay routes available in this energy region. $\Delta$ is the binding energy of the alpha particle, which can be calculated using equation \ref{eq:Binding_energy1}, where the mass of the proton is $m_p = 938.28$ MeV/c$^2$ and $m_n=939.57$ MeV/c$^2$. $\Delta$ is thus the difference in in energy-mass between the particle composed of protons and or neutrons, and a equal number of protons and neutrons independent. All parities are positive due to angular momentum $\ell=0$.  }
    \begin{tabular}{|c|c|c|c|c|}
         \hline 
         Particle & \Delta (MeV) &J^\pi & Charge \\
         \hline
         p & - & $\frac{1}{2}^+$ & +1 \\
         n & - & $\frac{1}{2}^+$ & 0 \\
         d & 2.2 & $1^+$ & +1 \\
         t & 8.5 & $\frac{1}{2}^+$ & +1 \\
         ^{3}He & 7.7 & $\frac{1}{2}^+$ & +2 \\
        \alpha & 28.3 & 0^+ & +2 \\
        \hline
    \end{tabular}
    \label{tab:decaychannel_particles}
\end{table}


\begin{table}[]
    \centering
    \begin{tabular}{|c|c|c|c|c|c|c|}
        \hline
        Target & Proton & neutron & deuterion & triton & ^{3}H & \alpha  \\
        \hline
        Fe (Z=26) & & & & & &  \\
        Ni (Z=28) & & & & & &  \\
        Cu (Z=29) & & & & & &  \\
        Ir (Z=30) & & & & & &  \\
        
    \end{tabular}
    \caption{The Coulomb barrier in MeV for each of the targets. Can be used to explain why we see what we see. }
    \label{tab:my_label}
\end{table}

The total angular momentum J is the sum of the orbital and intrinsic angular momentum (spin), thus 
\begin{equation}
    \mathbf{J} = \mathbf{\ell} + \mathbf{s} 
\end{equation}

The spin is the sum of unpaired paired protons and neutrons. The total nuclear spin is thus $I = \sum_i J_i$.  The value of the angular momentum is
\begin{equation}
    |I_i - I_f| \leq \ell \geq |I_i + I_f|
\end{equation}

Thus, 





\end{comment}


\begin{comment}
%\section{Radioactive decay law and Gamma-ray spectroscopy}
\section{Detection and identification of radionuclides}
Gamma-ray spectroscopy is a method to identify and obtain information about radioactive nuclei present in a detector. As beta and alpha decay can result in an excited daughter product, the spectrum in fact shows the de-excitation of the daugher product. Since we know that these gamma-lines are transitions which happens right after a beta or alpha decay (or isomer transition), we identify the parent with gamma-ray spectroscopy. A detector has channels in which counts are regiserted. These channels are ... similar to the gamma-ray energy. Thus a spectrum has channels (which increases in energy) along the x-axis and and counts along the y-axis. If a detector registers many counts, it means that the state is highly populated, and the intensity of the gamma is strong (Krane, p. 351). 

%\subsection{Obtaining a gamma-rey spectrum}
Ideally, for all gamma-rays with the same energy, should be detected in the same channel giving a step function. However, realistically, the resolution of a detector is not that good, and instead of seeing a delta peak, the peak is typically gaussiam shape with a finite width. The full width half maximum $\Delta E$ of the peak tells us how well the relative resolution at gamma-energy E,
\begin{equation}
    \text{resolution} = \frac{\Delta E}{E}
\end{equation}

The energy resolution is important, as it tells us how well it can distinguish two close lying peaks from each other (\cite{Leo1994}, p. 117). The resolution of a germanium detector very good (0.1\% for a 1 MeV gamma-ray) in comparison to for instance NaI detector (8-9\% for a 1 MeV gamma-ray) (\cite{Leo1994}, p. 117). \textcolor{red}{explain why, prob in semiconductor chapter!} \\


The peak it self is not directly gaussian. Ionizing radiation statistics is based upon Poisson statistics, where  and the probability of observing N events is a discrete value

\begin{equation}
    P(N) = \frac{\mu^N e^{-\mu}}{N!}
\end{equation}

where $\mu$ is the mean value. This distribution counts when the probability is a small (eg decay prob?) value and that the total number of trials are large (number of decays) (\cite{Leo1994}, p. 85).  For poisson distribution, the average is equal to the variance; $\sigma^2=\mu$. From there, the standard deviation ($\sigma$) is thus equal to the squareroot of the average. 

The distribution is not symmetric, but as $\mu$ increases in value, the peak approxes a gaussian shape. The total number of counts is the area of the peak. The total peak is a Guassian assumption but with an exponential skew towards kiw E caused by incomplete charge collection, abd a step function for taking compton backgroun into account. 

In calculation of the peak area, there are two uncertainties of relevance, the relative statistical uncertainty in the counting from the Poisson statistics, 
\begin{equation}
\sigma N_i = \sqrt{N_i}
\end{equation}

If numb of counts $N_i =10000$, the relative uncertainty ($\frac{\sigma N_i }{N_i}= \frac{1}{{sqrt{N_i}}}$=1\%). Therefor we say that a good number of counts is 10000 or more to reduce the statistical uncertainty. The other is systematic in the detector, and can for instance be due to a process called annealing, which is heat damage to the detector. Can fix by taking a blanket of resistor wrap crystal in, rise to high temp, let it sit and slowly deheat to room temp, traps will defuse and detector is repaired (this is notes from Andrew).

Also write about deadtime! 
\subsection{Gamma-ray spectrum}

Spectrum: consists of photopeaks, a compton continuum, compton edge, backscatter peak, single exscape double escape. In cases where positrons exist, chances of having a broad fat 511 keV peak. 

Germanium detecors, highest resolution for gamma-rays, from frew keV to 10 MeV. The peak to Compton ratio is much greater due to the higher photoelectric cross section of Germnaium . The largets challenges are with signal to noise ratio, it is important to shield very well to minimze background radiation (Techniques for Nuclear and particle..... (\cite{Leo1994}, p. 241) \\

\noindent 
here from another citation: "Practical Gamma-ray Spectroscopy". Gordon R. Gilmore. Nuclear Training Services Ltd Warrington UK \cite{Gilmore2008} . \textcolor{red}{(can be find under articles in masterthesis). This book can also be used in particle interaction in matter check!!}
In a detector, the particles interacts as the photons described in particle interaction, via photoelectric, compton scattering and pair production. Photoelectric absorption where the photon is completely absorbed by atomic electron is desired because all of the energy is deposited within the detector. For a compton scattering event, if the resulting photon's energy is also deposited in the detector (for a large detector), then the total energy would add up. Same for pair production. The photon must interact in the detector volume, and the resulting electron and positron energy is deposited in the detector volume. However when the positron slows down, it annihilates with one atomic electron, releasing two 511 keV photons. If both annihilation photons's energy is deposited in the detector volume this will also contribute to a full width peak. If one 511 photon escape and the other is deposited, there will be a peak at $E_\gamma-511$ keV, and if both peaks escape, there will a double escape peak at $E_\gamma-1022$ keV. The "degree of incomplete absorption" depends upon the size of the detector and the gamma-ray energy. As previously discussed photoelectic effect dominates at low energies, and the less compton scattering and of course pair production (for E gamma higher than the threshold.). The detector size also matters because the larger the more room for the photon to scatter in and lose energy before escaping. (p. 32) \\

The total spectrum can be seen on p. 33 in the book. Pile-up is done because of random summing, determined by the statistical probability of two gamma-rays being detected at the same time and therefor on the sample count rate. 

Interaction with detector shielding: Photoelectric effect can be followed by emission of characteristic X-ray of the absorbing medium. X ray can escape the shielding and be deteced by the detector. Compton scattering: most gamma rays are scattered through the a large angle by the shielding, BACKSCATTERED. Whatever the initial enervy  was (if scattered by more than 120 degrees) are within 200-300 keV. Peak appears as broad. Pair production: annihilation peak (511 peak) caused by the escape of one of the 511 keV photons from the shielding following annihilation of the pair production positron. Analogous to the single and double escape mechanisms within the detector but only on 511 keV photons can ever be detected since they are emitted in the opposit direction. So in order to have a 511 peak, energy of gamma ray must be more than 1022 keV. (p. 34-35).

The 511 peak can also be expected when postron emitters are present since beta + particle interacts with electron.  

Since Compton scattering can be in a spectrum of energies, the gives rise to a Compton continuum, before the gamma-ray escapes the detector. 


The shape of the peak: The peak is a histogram that approximate a Gauss curve (p. 186). Peak searching (SAMPO) using first and second order derivatives to search for peaks (p.185)
Due to incomplete charge collection (that electron or holes are not collected) no matter how caused moves counts from the centre of the Gaussian distribution to lower channels, creating a low energy tale to the peak (p.135).  

\textcolor{red}{Include a picture of peak shape and gamma-ray spectrum!! from the same book}


\end{comment}