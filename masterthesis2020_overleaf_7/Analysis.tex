\chapter{Analysis}




The analysis of estimating production cross sections consisted of multiple steps. To obtain the end of beam activities the peak areas (number of counts) were found with gamma-ray spectroscopy using FitzPeaks\footnote{https://www.jimfitz.co.uk/fitzpeak.htm}. The efficiency calibration as a function of gammaray energy was done using $^{137}$Cs, $^{133}$Ba and $^{152}$Eu point sources. The energy degradation in the foils were simulated using NPAT\footnote{https://pypi.org/project/npat/}, giving the deuteron flux as a function of energy. Along with the simulation and IAEA recommened cross sections for the monitor reactions\footnote{https://www-nds.iaea.org/medical/monitor_reactions.html}, the weighted average beam current was estimated in each foil. 

\section{Gamma-ray spectroscopy}
The spectra were analyzed in FitzPeakz\footnote{https://www.jimfitz.co.uk/fitzpeak.htm}. The mathematic algorithm in which FitzPeakz is based upon is SAMPO80 \footnote{https://sci-hub.tw/https://doi.org/10.1016/0029-554X(81)90209-3}. \textcolor{blue}{The peaks are assumed Gaussian with an exponential tale on both sides of the peak. The exponential tale and Gaussian function are joined so function and first derivative are continous. The algorithm searches for peaks by using the smooth second difference (derivative?) Particularly good for detecting small peaks on a high or low background. The peak areas are calculated by fitting the precalibrated modified Gaussian to the data with a weighted least squares formula using a parabolic background. Fitting intervals are determined automatically by the program. Peaks separarated by less than 4 times the average fwhm are fitted together. }

For each spectra,  a report file containing peak energy, centre channel, full width half maximum, significance, goodness of fit, peak area, uncertainty in peak area, gammas per second, uncertainty in gammas per second and a background estimation for each peak was provided. The most important parameters were the energy, the peak area $N_C$ and uncertainty in peak area. Peak area was needed for the activity calculation in equation \ref{eq:Final_Expression_A0} which is an important parameter in the calculation of the cross section (equation \ref{eq:experimental_CS}), and in the calculation of the efficiency for the calibration sources (equation \ref{eq:efficiency_analytical}). Gammas per second (also called countrate) was used to get an indication if the rate of gammas, which were used as a critical tool to evaluate background contamination in a peak for instance. 

Figure \ref{fig:gammarayspectrum_example} shows an example of a gamma ray spectrum for one of the iridium spectra (Ir05) approximately 35 hours after end of beam. Figure \ref{fig:193mPt_spectra} shows the X-ray region and gamma region of $^{193m}$Pt. 

The nuclei were identified on behalf of thei're gamma-ray energy. The feeding of multiple nuclei into one gamma-ray peak was avoided by using other gamma-lines. 

\begin{figure}
    \centering
    \includegraphics{Analysis/gammaray_spec_Ir05.png}
    \caption{Gammaray spectrum for Ir05 taken approximately 35 hours after end of beam. Nuclei does not necessarily represent what is present in the spectrum, but where the peak would have been. Hard to include all since there are different decay times. }
    \label{fig:gammarayspectrum_example}
\end{figure}


\begin{figure}%
    \centering
    \subfloat[X ray plot of 193mPt. ]{{\includegraphics[width=10cm]{Analysis/193mPt_xray_plot.pdf} }}%
    \quad
    \subfloat[Gammaplot of 193mPt]{{\includegraphics[width=10cm]{Analysis/193mPt_gamma_plot.pdf} }}%
    \caption{\textcolor{red}{Which spectra are these??  }}%
    \label{fig:193mPt_spectra}%
\end{figure}


\subsection{Energy and peak-shape calibration} \label{subsec:fitz_calibration}
Used calibration sources $^{137}$Cs ($t_{1/2}=30.08$ years\cite{Browne2007}), $^{133}$Ba ($t_{1/2}=10.551$ years\cite{Khazov2011}) and $^{152}$Eu ($t_{1/2}=13.517$ years\cite{Martin2013}), which can be seen on figure \ref{fig:calsources}.

\textcolor{blue}{The calculated peak locations and areas are finally corrected with energy and efficiency calibration data to yield peak energies and intensities. For the energy calibration, linear interpolation on a linear scale and for the efficiency calibration linear interpolation on a log-log scale are used in this code. Calibration errors are added to the peak location and intensity errors to give the final result (p.94). \\ The peak shape calibration uses 7 parameters; two background peaks, peak height and location, peak width, distance from peak centroid to the starting point of exponential on either side. The minimization of the least-squares expression to solve for the peak parameters is done by a subroutine package  with an iterative gradient algorithm utilizing the variable metric method. Minimization is terminated when all all components in the next step change by less than $10^{-8}$, if four succeeding values of $\chi^2$ are the same or if 100 iterations have been completed. The performed shape calibration can be checked with a few parameters, goodness of fit, $\chi^2$ per degree of freedom, sigma and error correlation. Sigma below 5 and error correlation between -1 and 1 are acceptable values.  (p.90)} \\


From the webpage\foonote{jim-fitz.com/calib.html}:
Each detector was calibrated with peak shape and energy for the calibration sources. Fitzpeakz takes in energy (.enc) and peak shape (.shp) calibration source files, containing the energies listed in table\ref{table:calibration_gammas}. For the peak shape, the program determines the parameters of width and the amount of low energy tailing. The energy calibration and peak shape calibration was estimated to a 1st order function. 




\section{Efficiency calibration} \label{sec:efficiency_calibration}
The efficiency calibration is an important factor in the calculation of the cross section in equation \ref{eq:experimental_CS}. The detector efficiency is the number of events registered divided by the events emitted by the source. The absolute efficiency can be divided into intrinsic and geometrical efficiency, where the intrinsic efficiency is the number of events registered divided by the number of events hitting the detector. The
intrinsic efficiency thus depends on the interaction cross section between incident particle and detector material. For neutral particles, the size of the detector affects the intrinsic efficiency, the larger crystal the larger the probability of interaction is. The geometrical efficiency is the radiation emitted by the source which hits the detector. (Techniques for Nuclear and Particle Physics Experiments. William R. Leo. Second Revised Edition. Springer.Verlag Berkling Heidelberg GmbH, New York (1994). p. 121-122 ) \\

\noindent 
The efficiency was measured using calibration point sources $^{137}$Cs ($t_{1/2}=30.08$ years\cite{Browne2007}), $^{133}$Ba ($t_{1/2}=10.551$ years\cite{Khazov2011}) and $^{152}$Eu ($t_{1/2}=13.517$ years\cite{Martin2013}). Figure \ref{fig:calsources} shows the calibration points sources ($^{22}$Na was excluded during the data-analysis since it only contains a single gamma-line and gave poorer results). On each calibration source, a reference date is given with an activity, which here is referenced to as $A_0$ of the calibration sources.\\

\noindent Solving Equation \ref{eq:Final_Expression_A0} for effiency, $\epsilon$, the analytical effiency as a function of gamma-ray energy and intensity is 
\begin{equation} \label{eq:efficiency_analytical}
    \epsilon(E_\gamma)= \frac{N_C \lambda}{A_0 I_\gamma (1-e^{-\lambda \Delta t_c})e^{-\lambda\Delta t_d}}
\end{equation}

where $\lambda$ is the decay constant and $N_C$ is the number of counts in the measured spectra, and $\Delta t_d$ is the delay time since the reference date. The analytical efficiency gives one single value for the efficiency at energy $E_\gamma$, but we want a continuous function which gives the effciency at any gamma-energy. A model based upon Gallagher, W. J., Cipolla, S.J. (1974) was applied which takes the probability of penetration through the deadlayer of the detector and the probability of interaction in the detector volume into account

\begin{equation} \label{eq:efficiency_estimated}
\epsilon(E_\gamma) =  B_0 + \underbrace{(e^{-B_1 E_\gamma^{B_2}})}_\text{dead layer}  \underbrace{(1-e^{-B_3 E_\gamma^{B_4}}))}_\text{interacting with volume} 
\end{equation}
\noindent 
where $B_i$ is optimum parameters minimizing the $\chi^2$ in the scipy optimizing curve fit function\footnote{https://docs.scipy.org/doc/scipy/reference/generated/scipy.optimize.curve_fit.html}). Figure \ref{fig:efficiency_curve} shows an example of an efficiency curve for a detector at a specific distance from the detector. The uncertainty of the efficiency was estimated using equation \ref{eq:variance_full} numerically. For each source, the gamma-lines with the intensities which were used to calculate the efficiency points for each source is listed in table \ref{table:calibration_gammas}. \\


\begin{figure}
    \centering
    \includegraphics[width=0.6\textwidth]{Experiment/room131_15.png}
    \caption{An example of an efficiency curve with exact points calculated from equation \ref{eq:efficiency_analytical} and a curve fit from equation \ref{eq:efficiency_estimated}. }
    \label{fig:efficiency_curve}
\end{figure}

\section{End of beam activities}

The end of beam activities were estimated by extrapolating backwards in time, with the measured activities at various timepoints after the end of beam. The activities as a function of time sine EOB was calculated using equation \ref{eq:Final_Expression_A}, along with a self-attenuation correction: 

\begin{equation}
    A(\Delta t_d) = \frac{N_C\lambda}{\epsilon\I_\gamma (1-e^{-\lambda \Delta t_d})e^{-\mu\rho\Delta r/2}}
\end{equation}

where $\mu$ is the photoon attenuation coefficients from the XCOM photon cross section database+\footnote{https://www.nist.gov/pml/xcom-photon-cross-sections-database}, and $\rho\Delta r$ is the mass density of the foil. The gammas which were used are listed in tables \ref{tab:Products_Fe}, \ref{tab:Products_Ni}, \ref{tab:Products_Cu} and \ref{tab:Products_Ir} for iron, nickel, copper and iridium respectively. The gamma-ray self-attenuation (which is typically less than 0.2 \% (Iron paper, Andrew)) correction is based on the assumption that all activity that is made is located midway in the foil thicknesses. In reality however, the activity profile will follow the same shape as the excitation function over the energy range that expands over the foil, \textcolor{red}{if we assume that the stopping power dE/dx=0 which is a good estimation for thin foils less than 100 mg/cm$^2$??} (since activity and cross section are proportional). We do not know the excitation function ahead of time, and the excitation function does not change much either, since the foil thicknesses are so thin. So instead, this simplification is done, assuming that the average attenuation is through half of the foil thickness. \\ 

\noindent 
The equation describing the shape of the decay curve is given in equation \ref{eq:activity_decaylaw} for single decay or \ref{eq:ndecay_chains} for multiple decay. Decay chains of single and two-step decay (n=1,2) was sufficient in this analysis; 
\begin{equation} \label{eq:onestep_activity}
    A = A_0 e^{-\lambda \Delta t_d},\quad \text{ single step decay}
\end{equation}

and

\begin{equation} \label{eq:twostep_activity}
    A_2(t) = \lambda_n \Big[ A_{1,0}\lambda_1 \frac{(e^{-\lambda_1 } + e^{-\lambda_2})}{\lambda_1 - \lambda _2} + A_{2,0}e^{-\lambda_2 t} \Big],\quad \text{two step decay}
\end{equation}

where subnumber 1 is the parent nucleus, and subnumber 2 is the daughter nucleus. Parent activity is calculated from single step decay. The uncertainty was treated as \textcolor{red}{covarianced variables?} \\

The way in which the extrapolation was done was the scipy optimize curve fit function, where the $A_0$ of the daughter was the optimizing parameter. Since there is only one optimized parameter, there was no covariance and the uncertainty was calculated using equation \ref{eq:uncertainty_simplification}.  In the cases where neither parent or daughter activity were known, which were the case for the monitor reaction $^{58}Co$ with $^{58m}$Co decaying into the ground state by internal conversion, both parent and daughter activity were optimizing parameters which are very correlated and thus the uncertainty in end of beam activity was calculated \ref{eq:variance_full}. Figure \ref{fig:activity_curves} shows two examples of the two different activity curves; one step decay for $^{193m}$Pt ($t_{1/2}$=4.33 days) and two step decay for the monitor product $^{58}$Co ($t_{1/2}$=70.86 days) with feeding from the isomer $^{58m}$Co ($t_{1/2}$=9.10 hours).  


\begin{figure}%
    \centering
    \subfloat[Activity of $^{193m}$Pt ($t_{1/2}$=4.33 d) produced from iridium. The end of beam activity was estimated using a one step decay (equation \ref{eq:onestep_activity})]{{\includegraphics[width=11cm]{Analysis/activity_Ir_193mPt_f6.png} }}\hfill
    \subfloat[Activity of $^{58}$Co ($t_{1/2}$=70.86 d) produced from nickel. The end of beam activity is estimated using a two step decay (equation \ref{eq:twostep_activity}. The feeding is from $^{58m}$Co ($t_{1/2}$=9.10 h.) ]{{\includegraphics[width=11cm]{Analysis/activity_Ni_56Co_f7.png} }}%
    \caption{Two examples of activity curves. The uncertainty in activity decreases with increasing time since end of beam which is due to longer counts decreases the uncertainty. (from theory, counting statistics)  }%
    \label{fig:activity_curves}%
    
\end{figure}



\section{Estimation of the beam current}
The beamintegrator measured a current of 128.5 nA in front of the beam stack. However in order to have precise cross section measurements, the beamcurrent in each foil was estimated. The IAEA recommended monitor reactions (2017) $^\text{nat}$Ni(d,x)$^{61}$Cu,$56,58$Co, $^\text{nat}$Cu(d,x)$^{62,63,65}$Zn and $^\text{nat}$Fe(d,x)$^{56}$Co were used to obtain a weighted average beam current in each foil solving equation \ref{eq:Final_Expression_A0} for beam current $\Phi$:

\begin{equation} \label{eq:BC_simple}
    \Phi(E) = \frac{A_0}{N_T \sigma(E)_\text{mon}(1-e^{-\lambda \Delta t_\text{irr}})}
\end{equation}

Equation \ref{eq:BC_simple} builds upon the thin target assumption, which implies that the energy degradation dE/dx=0. However, we know that there is an energy distribution, which was estimated using NPAT's (Nuclear Physics Analysis Tool) Ziegler simulation. The ziegler code simulates the deuteron transport based upon the Anderson \& Ziegler stoppingpower formalism, using Monte Carlo simulations \textcolor{red}{write a few sentences in theory..}. The code provides the full deuteron energy and flux degradation in each foil, $d\phi/dE$, which can be visualized for the iridium foils in figure \ref{fig:ir_energyflux}.  Can be seen that as the deuteron energy is degraded, the mean value is shifted towards the low energy side, and the the peak width increases. As stoppingpower is inversely proportional to the charge particle energy ($-\frac{dE}{dx}\propto \frac{1}{\beta^2}$, bethe block), and along with scattering taking place towards the end of stack, the low energy tail is more degraded, and we see a skew towards the low energy, creating a broader energy-flux profile and a shift of the mean value (centroid). This shift leads to an increasing uncertainty in energy. The (normalized) flux-weighted average energy for each foil was calculated, \textcolor{blue}{ironpaper: which takes into account the slowing down of of deuterons, and reports effective energy centroid of each foil}, using the energy distributions $d\phi/dE$ provided by the Ziegler code:

\begin{equation} \label{eq:flux_weighted_average_energy}
    \langle E \rangle = \frac{\int E \frac{d\phi}{dE}dE}{\int \frac{d\phi}{dE}dE}
\end{equation}

The uncertainty in beam energy is divided into low energy and high energy tale, with the FWHM split by the centroid (figure \ref{fig:ir_energyflux}). 

Likewise, the energydependent monitor IAEA cross sections need to be flux-weighted over each foil. In order to do this, a spline interpolation over the energy array over each foil provided by the Ziegler simulation was spline interpolated with the IAEA recommended cross section data. Thus, the monitor cross section in equation \ref{eq:BC_simple} is modified to 

\begin{equation}
    \sigma (\langle E\rangle) = \frac{\int \sigma_\text{mon} \frac{d\phi}{dE}dE}{\int \frac{d\phi}{dE}dE}
\end{equation}


\begin{figure}
    \centering
    \includegraphics{Analysis/Ir_flux_distribution_B_+2_D_+4,25.png}
    \caption{Iridium energy flux distribution over the 10 foils. As the energy degrades, scewed and larger full width half max. The vertical line in each peak is the mean value. This indicates that at lower energies, the right uncertainty is greater than the left uncertainty in the peak.}
    \label{fig:ir_energyflux}
\end{figure}

With the end of beam activities for the monitor reactions, number of target nuclei and the flux- weighted IAEA cross sections, the beam current as a function of the flux-weighted average beam energy was estimated for each reaction in each foil. 

\subsection{Variance minimization of the deuteron transport calculations}
In theory, the estimated beam current of a charge particle beam should be constant, until completely stopped, since the majority of the incident particles does not interact in nuclear reactions, but only lose energy via elastic and inelastic scattering. However, non-consistant values of the beam current, especially in the back of the stack can be due to energy bins being assigned wrongly in the energy distribution simulation done in Ziegler or a systematic error in the areal density which gets progressively worse further back in the stack (Niobium paper, Andrew). A way to work around these errors was to perform a variance minimization varying the beam energy and the areal density of the foils with 20\% increase and decrease systematically, and estimate the reduced $\chi^2$ (equation \ref{eq:chisq_DOF}) over compartment 3,6 and 9. Variance minimization (Andrew's Niobium and iron paper + \foonote{https://sci-hub.tw/https://doi.org/10.1016/j.nimb.2016.09.018}). \\

 For compartment 3 ($E_d$=25 MeV) all seven monitor reactions were above threshold,  thus 6 degrees of freedom.  However,  early in the target stack,  the scattering was low, and the $\chi^2$ does not tell how well the energy bin assignment work further back in the stack.  For compartment 6 ($E_d$=18 MeV), all the six possible monitor reactions (for nickel and copper) were above threshold, and it gave a good estimate of how the beam current was developing throughout the stack.  In compartment 9 ($E_d$=10 MeV), five monitor reactions are above threshold (except for $^{62}$Zn).  At the very end it is possible to see the full effect of the scattering.\\

With the assumption that the beamcurrent loss is zero over one compartment, a linear fit-model (using the scipy optimize curvefit function) with a slope equal to zero was used to estimate the beam current in each compartment, and with the estimated $\chiˆ2$.  

Figure \ref{fig:BC_comp6} shows the uncertainty weighted linear fit over compartment 6. 

\begin{figure}
    \centering
    \includegraphics{Analysis/Compartment_6.png}
    \caption{The estimated (uncertainty weighted) beamcurrent over compartment 6. }
    \label{fig:BC_comp6}
\end{figure}

Figure \ref{fig:varmin_beamcurrent} shows the beam current before and after variance minimization, and the weighted average beam currents are listed in table \ref{tab:weighted_BC} estimated before and after the variance minimization. After variance minimization, the beam current estimated in each compartment (stabled lines) were similar, and meanwhile the weighted $\chi^2$ was about the same in compartment 6, it has improved in compartment 3 and very visible in compartment 9. In general the points are also more aligned. 

\begin{figure}%
    \centering
    \subfloat[]{{\includegraphics[width=11cm]{Analysis/beamcurrent_B_0_D_0.png} }}\hfill
    \subfloat[A 2\% increase in beam current and a 4.25\% increase in areal density gave the overall most consistent beam current, with reasonable values for the weighted . ]{{\includegraphics[width=11cm]{Analysis/beamcurrent_B_+2_D_+4,25.png} }}%
    \caption{Beam current before and after variance minimization.  }%
    \label{fig:varmin_beamcurrent}%
\end{figure}

\begin{table}[]
    \centering
        \caption{The weighted average beam current before and after variance minimization in each compartment. The beam current on the 88-Inch Cyclotron beam integrator was 128.5 nA.}
    %\footnotesize
    \begin{tabular}{c |c c}
        \hline 
       Compartment & \makecell{Before} & \makecell{After}\\ 
        \hline 
        \makecell{01} & \makecell{131.56 \pm 3.64} & \makecell{134.08 \pm 3.70}  \\ 
        \makecell{02} & \makecell{132.23 \pm 3.74} & \makecell{136.42 \pm 3.83} \\ 
        \makecell{03} & \makecell{133.81 \pm 3.64 } & \makecell{138.02 \pm 3.75} \\ 
        \makecell{04} & \makecell{134.89 \pm 4.21 } &  \makecell{138.88 \pm 4.31}\\ 
        \makecell{05} &\makecell{136.85 \pm 4.21} & \makecell{139.67 \pm 4.29} \\
        \makecell{06} &\makecell{137.40 \pm 4.53} & \makecell{138.85 \pm 4.58} \\
        \makecell{07} &\makecell{139.55 \pm 4.37} & \makecell{139.77 \pm 4.37} \\
        \makecell{08} &\makecell{133.60 \pm 4.27} & \makecell{134.96 \pm 4.32}\\
        \makecell{09} &\makecell{133.16 \pm 5.04} & \makecell{143.59 \pm 5.67} \\
        \makecell{10} &\makecell{108.49 \pm 5.80} & \makecell{121.75 \pm 6.65} \\
    
         %\makecell{Before} & \makecell{131.56 \pm 3.64} & \makecell{132.23 \pm 3.74} & \makecell{133.81\pm3.64 } & \makecell{134.89\pm 4.21 } & \makecell{136.85_{\pm 4.21}} & \makecell{137.40 \pm 4.53} & \makecell{139.55 \pm 4.37} & \makecell{133.60 \pm 4.27} & \makecell{133.16 \pm 5.04} & \makecell{108.49 \pm 5.80}  \\
         %\makecell{After} & \\
    \end{tabular}
    \label{tab:weighted_BC}
\end{table}

\begin{comment}
\section{Energy and Beam current}



For the equation for cross section (equation \ref{eq:cross_section_equation}), the beam current $\Phi(E)$ must be known. The beam integrator measured 128.5 nA, which is the current entering the stack. However, due to large energy degradation in the energy stack, there will be a certain spread of the beam, following scattering. In addition, there have not been an experiment with deuterium on a target stack before, so we also needed to see how much deuteron break up affected the current throughout the stack. Monitor reactions are reactions with well-known cross sections\footnote{https://www-nds.iaea.org/medical/monitor_reaction_article.pdf}. The IAEA recommended cross sections for $^\text{nat}$Fe(d,x)$^{56}$Co, $^\text{nat}$Ni(d,x)$^{61}$Cu, $^\text{nat}$Ni(d,x)$^{56,58}$Co and $^\text{nat}$Cu(d,x)$^{62,63,65}$Zn (\textcolor{red}{write about Q value, half life}) were used to estimate a more sensitive deuteron beam current throughout the stack. By solving \ref{eq:cross_section_equation} for beam current, the beamcurrent throughout the stack can be estimated

\begin{equation}
    \Phi(E) = \frac{A_0}{N_T (1-e^{-\lambda \Delta t_\text{irr}})\sigma(E)_\text{mon}}
\end{equation}

In cross section experiments using thin targets\footnote{Special curriculum p. 14}, the suggested value is a flux average cross section, which implies that the cross section is dependent on the flux-weighted average beam energy. One single foil thus provides one cross section measurement, with the uncertainty in energy only being dependent on the energy distribution in each foil. For thin targets, a stopping power $dE/dx=0$ is assumed which is a very good approximation for targets which are less than 50 mg/cm$^2$ \textcolor{red}{cite?}.   

Normalized differential beam current  (\textcolor{red}{need some help understanding this in detail})
\begin{equation}
    \sigma(E)_\text{mon}=\frac{\int \sigma(E)\frac{d\phi}{dE}dE}{\int \frac{d\phi}{dE}dE}
\end{equation}

\textcolor{blue}{From niobium paper: Since stoppingpower is inversely proportional for to cp energy, the low energy tail of the energy distribution is degraded more in each stack element than the high energy tail. This effect compounds to towards the rear of the stack, creating a significantly broadened low energy tail, and a progressevely larger net shift of the centroid to a lower energy. To account for this increasing energy uncertainty, a suitably representative energy must be established for each foil of the target stack. The flux weighted average deuteron energy in each foil $\langle E \rangle$ represents the energy centroid for deuterions in a target stack using the energy distributions from $d\phi/dE$ from the Ziegler stopping power deuterion transport }
\begin{equation}
    \langle E \rangle = \frac{\int E \frac{d\phi}{dE}dE}{\int \frac{d\phi}{dE}dE}
\end{equation}

where $\sigma(E)$ is the IAEA recommended cross section, $\frac{d\phi}{dE}$ is the energy dependent deuteron flux through each foil. The deuteron flux (or energy degradation) was estimated using a code called NPAT's (nuclear physics analysis tool) Ziegler simulation\footnote{https://pypi.org/project/npat/}. NPAT uses the Anderson \& Ziegler formalism for calculating charged-particle stopping powers in matter in a stack with targets. \textcolor{red}{write a few sentences about this in the theory!}. The code thus simulates the deuteron flux as a function of energy in the beam stack (assigned to a foil).

Figure \ref{fig:ir_energyflux} shows an overview of the flux-energy distribution in each foil for Iridium. The other monitor foils have the same functions.  \textcolor{red}{Write about recommended data, the spline function etc etc.. }

\begin{figure}
    \centering
    \includegraphics{Analysis/Ir_flux_distribution_B_+2_D_+4,25.png}
    \caption{Iridium energy flux distribution over the 10 foils. As the energy degrades, scewed and larger full width half max. The vertical line in each peak is the mean value. This indicates that at lower energies, the right uncertainty is greater than the left uncertainty in the peak.}
    \label{fig:ir_energyflux}
\end{figure}

\textcolor{red}{need some info from andrew here}

\subsection{Variance minimization}
\textcolor{blue}{From Niobium paper p. 57: In theory, the beam current should be more or less constant through the stack, even though the deuterons lose energy. Variance minimization is a technique to reduce the uncertainty in the deuteron beam energies. Non-consistent values  for the beam current further back in the stack can be wrong energy bin assignments in the modeled energy distribution (ziegler), or a systematic error in the areal density, which gets progressively worse further back in the stack, "due to the compound effect of systematic uncertainties in stack areal densities"}. The areal density and the beam energy was varied with 20\% increase and decrease, and the reduced $\chi^2$ (equation \ref{eq:chisq_DOF}) was estimated over compartment 3, 6 and 9. For compartment 3 ($E_d\simeq$25 MeV) all seven monitor reactions were above threshold, thus 6 degrees of freedom. However, early in the target stack, the scattering was low, and the $\chi^2$ does not tell how well the energy bin assignment work further back in the stack. For compartment 6 ($E_d\simeq$18 MeV), all the six possible monitor reactions (from nickel and copper) were above threshold, and it gave a good estimate of how the beam current was developing throughout the stack. In compartment 9 ($E_d\simeq$10 MeV), five monitor reactions are above threshold (except for $^{62}$Zn). At the very end it is possible to see the full effect of the scattering. \\

\noindent 
The beamcurrent loss is assumed zero in one compartment, so a linear beam current fit would have a slope equal to zero. The estimated beam current in each compartment was estimated using the scipy curve fit function, with a straight line as model. Figure \ref{fig:BC_comp6} shows the uncertainty weighted linear fit over compartment 6. 

\begin{figure}
    \centering
    \includegraphics{Analysis/Compartment_6.png}
    \caption{The estimated (uncertainty weighted) beamcurrent over compartment 6. }
    \label{fig:BC_comp6}
\end{figure}

Figure \ref{fig:varmin_beamcurrent} shows the beam current before and after variance minimization, and the weighted average beam currents are listed in table \ref{tab:weighted_BC} estimated before and after the variance minimization. After variance minimization, the beam current estimated in each compartment (stabled lines) were similar, and meanwhile the weighted $\chi^2$ was about the same in compartment 6, it has improved in compartment 3 and very visible in compartment 9. In general the points are also more aligned. 

\begin{figure}%
    \centering
    \subfloat[]{{\includegraphics[width=11cm]{Analysis/beamcurrent_B_0_D_0.png} }}\hfill
    \subfloat[A 2\% increase in beam current and a 4.25\% increase in areal density gave the overall most consistent beam current, with reasonable values for the weighted . ]{{\includegraphics[width=11cm]{Analysis/beamcurrent_B_+2_D_+4,25.png} }}%
    \caption{Beam current before and after variance minimization.  }%
    \label{fig:varmin_beamcurrent}%
\end{figure}


\begin{table}[]
    \centering
        \caption{The weighted average beam current before and after variance minimization in each compartment. The beam current on the 88-Inch Cyclotron beam integrator was 128.5 nA. }
    %\footnotesize
    \begin{tabular}{c|c c c c c c c c c c}
        & \makecell{c_{10}} & \makecell{c_{09}} & \makecell{c_{08}} & \makecell{c_{07}} & \makecell{c_{06}} & \makecell{c_{05}} & \makecell{c_{04}} & \makecell{c_{03}} & \makecell{c_{02}} & \makecell{c_{01}} \\ 
        \hline
         \makecell{Before} & \makecell{131.56_{\pm 3.64}} & \makecell{132.23_{\pm 3.74}} & \makecell{133.81_{\pm3.64 }} & \makecell{134.89_{\pm 4.21 }} & \makecell{136.85_{\pm 4.21}} & \makecell{137.40 _{\pm 4.53}} & \makecell{139.55_{\pm 4.37}} & \makecell{133.60_{\pm 4.27}} & \makecell{133.16_{\pm 5.04}} & \makecell{108.49_{\pm 5.80}}  \\        
         %\makecell{Before} & \makecell{131.56 \pm 3.64} & \makecell{132.23 \pm 3.74} & \makecell{133.81\pm3.64 } & \makecell{134.89\pm 4.21 } & \makecell{136.85_{\pm 4.21}} & \makecell{137.40 \pm 4.53} & \makecell{139.55 \pm 4.37} & \makecell{133.60 \pm 4.27} & \makecell{133.16 \pm 5.04} & \makecell{108.49 \pm 5.80}  \\
         \makecell{After} & \\
    \end{tabular}
    \label{tab:my_label}
\end{table}


\begin{table}[]
    \centering
        \caption{The weighted average beam current before and after variance minimization in each compartment. The beam current on the 88-Inch Cyclotron beam integrator was 128.5 nA.}
    %\footnotesize
    \begin{tabular}{c |c c}
        \hline 
       Compartment & \makecell{Before} & \makecell{After}\\ 
        \hline 
        \makecell{01} & \makecell{131.56 \pm 3.64} & \makecell{134.08 \pm 3.70}  \\ 
        \makecell{02} & \makecell{132.23 \pm 3.74} & \makecell{136.42 \pm 3.83} \\ 
        \makecell{03} & \makecell{133.81 \pm 3.64 } & \makecell{138.02 \pm 3.75} \\ 
        \makecell{04} & \makecell{134.89 \pm 4.21 } &  \makecell{138.88 \pm 4.31}\\ 
        \makecell{05} &\makecell{136.85 \pm 4.21} & \makecell{139.67 \pm 4.29} \\
        \makecell{06} &\makecell{137.40 \pm 4.53} & \makecell{138.85 \pm 4.58} \\
        \makecell{07} &\makecell{139.55 \pm 4.37} & \makecell{139.77 \pm 4.37} \\
        \makecell{08} &\makecell{133.60 \pm 4.27} & \makecell{134.96 \pm 4.32}\\
        \makecell{09} &\makecell{133.16 \pm 5.04} & \makecell{143.59 \pm 5.67} \\
        \makecell{10} &\makecell{108.49 \pm 5.80} & \makecell{121.75 \pm 6.65} \\
    
         %\makecell{Before} & \makecell{131.56 \pm 3.64} & \makecell{132.23 \pm 3.74} & \makecell{133.81\pm3.64 } & \makecell{134.89\pm 4.21 } & \makecell{136.85_{\pm 4.21}} & \makecell{137.40 \pm 4.53} & \makecell{139.55 \pm 4.37} & \makecell{133.60 \pm 4.27} & \makecell{133.16 \pm 5.04} & \makecell{108.49 \pm 5.80}  \\
         %\makecell{After} & \\
    \end{tabular}
    \label{tab:weighted_BC}
\end{table}
\end{comment}
For cross section calculations, equation \ref{eq:CrossSection_general} is used, with the estimated weighted average beam current. Figure \ref{fig:monitor_BC+CS} shows the estimated cross sections for the monitor reactions, using the weighted average beam current over all monitor foils. The recommended monitor cross section data for the monitor reactions are also plotted, which was used in the cross section calculation.  



\begin{figure}%
    \centering
    \subfloat[]{{\includegraphics[width=7.5cm]{Analysis/Fe_56Co.png} }}%
    \quad
    \subfloat[]{{\includegraphics[width=7.5cm]{Analysis/Ni_61Cu.png} }}%
    \subfloat[]{{\includegraphics[width=7.5cm]{Analysis/Ni_56Co.png} }}%
    \quad
    \subfloat[]{{\includegraphics[width=7.5cm]{Analysis/Ni_58Co.png} }}%
    \quad
    \subfloat[caption]{{\includegraphics[width=7.5cm]{Analysis/Cu_62Zn.png} }}%
    \quad
    \subfloat[]{{\includegraphics[width=7.5cm]{Analysis/Cu_63Zn.png} }}%
    \quad
    \subfloat[]{{\includegraphics[width=7.5cm]{Analysis/Cu_65Zn.png} }}%
    \quad
    \caption{Figure shows the estimation of monitor cross section using the estimated weighted average beam current for each reaction (not the total). It is compared along with the recommended (IAEA) monitor data, and other experimental data  }%
    \label{fig:monitor_BC+CS}%
\end{figure}

\section{Cross section measurements} 
With all variables for cross sections, cross sections can be calculated using equation \ref{eq:CrossSection_generall}. Since the energy was a flux-weighted average beam energy, the value that is provided as cross section is a flux-averaged cross section. An accurate measure of the cross section as a function of deuteron energy was possible, as the thin foils provides smaller average beam energy intervals, and it makes it possible to have more measurements if thick foils are replaced with several thinner (one single foil represents a single measurement). \textcolor{red}{in theory: Thin foils also produce minimal amounts of radioactivity, thus the deadtime of the detector and the dose to humans is low.}% For charged particles, the stopping power is inversely proportional to their energy\footnote{Andrew S. Voyles, Lee A. Bernstein, Eva R. Birnbaum, Jonathan W. Engle, Stephen A.
%Graves, Toshihiko Kawano, Amanda M. Lewis, and Francois M. Nortier. Excitation functions
%for (p,x) reactions of niobium in the energy range of Ep = 40–90 MeV. Nuclear Instruments
%and Methods in Physics Research Section B: Beam Interactions with Materials and Atoms,
%429:53–74, aug 2018.}, and therefor the energy degradation in thicker foils will be large. For thin targets, we can however assume that the stopping power $dE/dx\simeq 0$, and the cross section can be replaced with a differential (normalized) cross section 
%
%\begin{equation}
%    \sigma(E) = \frac{\int \sigma(E) \frac{d\phi}{dE}dE}{\int \frac{d\phi}{dE}dE}
%\end{equation}

%The energy-limits in the integral can be minimized and the error in the energy for charged particles will be small. \\

\textcolor{blue}{Thin foils decreases the energy width, making a more precise measurement dependent on energy. However the reported cross sections are flux-averaged over the energy distribution subtended by each foil. } 

The cross sections are reported as independent if there is nothing decaying into it (beta feeding or from isomer transition), which means that the first observed element in a decay chain is reported as cumulative unless it is the first possible element (which are the nuclei with one more proton more than the target nuclei, which for this experiment is Platinum (from Ir), Zink (from Cu), Copper (from Ni) and Cobalt (from Fe). If there is feeding, and the half life is much shorter or much longer than the specific nuclei, can choose appropirate timewindow and report as independent, when we know that the feeding has either died out or is very low! 

If possible with feeding: added together to get cumulative or subtracted giving independent. Multiplying by branching ratio. 

The measured cross sections in this work was compared to previous experimental data, along with reaction modelling codes TALYS\foonote{https://sci-hub.tw/https://doi.org/10.1016/j.nds.2012.11.002}, TENDL, ALICE20 and CoH. \\

\noindent 
\textcolor{blue}{from talys cite, p. 2844: Software for the analysis and prediction of nuclear reactions that involve neutrons, photons, protons, deuterons, tritons, $^3$He and alphaparticles, in the 1keV-200 MeV energy range and for target radionuclides of mass 5 and heavier. A suite of nuclear reaction models has been implemented in one single code system. This enables to evaluate nuclear reactions from the unresolved resonance range up to intermediate energies. Can have many different outputs we are interested in exclusive channnel cross sections, energy and double differential spectra p. 2845. 

Optical model calculations performed first, }

Talys takes in projectile, target element (specific isotope or all stable target isotopes), energy array with desired spacing and upper limit energy. 

\textcolor{blue}{from Andrew mail: Each code was ran on default parameters (no changes to nuclear structure or the modeling of the reaction physics). Adjusting allows to tune the code to match experimental data, but this is of limited value as the parameters you settle on to match one product nucleus causes most of the others to get even worse - i.e., its a local optimization, not improving the code globally.  We run our codes for papers on defaults, as 1) that's how 99\% of users do so, and 2) it reveals how good each code's predictive power is. 

For COH: To get both 191Ir and 193Ir to run, we had to adjust the parameter "tweakSD", which adjusts the effective single-particle state density for a particular particle emission channel.  In the end, we ran with tweakSD=0.25  for both outgoing alphas and neutrons  (protons were unaffected).  In other words, we set the single-particle state density for outgoing alphas and neutrons [(d,xa) and (d,xn) reactions]  to be 25\% of what they normally are, which is a HUGE change.} 


\textcolor{blue}{from talys cite p. 2843-2844: TENDL is developed from talys (TALYS evaluated Nuclear data Library).  This
library consists of a complete set of nuclear reaction data
for incident neutrons, photons, protons, deuterons, tritons, Helium-3 and alpha particles, from 10$^−5$ eV up to 200 MeV, for all 2430 isotopes from 6Li to 281Ds that are either stable of have a half-life longer than 1 second.
All data are completely and consistently evaluated using a software system consisting of the TALYS nuclear reaction code, and other software to handle resonance data, experimental data, data from existing evaluations,
and to provide the final ENDF-6 formatting, including
covariance information. The result is a nuclear data library with mutually consistent reaction information for
all isotopes and a quality that increases with yearly updates.  To produce this library, TALYS input parameters are
adjusted for many nuclides so that calculated cross sections agree with experimental data, while for important
nuclides experimental or evaluated data are directly included. Also feedback from integral measurements is
processed into the data libraries. For nuclides for which
(almost) no experimental data exists, default TALYS
calculations based on global models and parameters are
used.}


Dont understand this part..... 