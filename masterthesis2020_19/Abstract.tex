

\chapter{Abstract}
In this thesis work, a stack of ten natural irridium (99.9\%) foils were irradiated with a 33 MeV incident deuteron beam at the Lawrence Berkeley National Laboratory's 88-Inch Cyclotron, yielding ten cross section measurements from ca. 5 to 30 MeV in the activated foils as the energy decreases as the deuterons traverses through the foils. The motivation behind this experiment was to measure the $^\text{nat}$Ir(d,x) reactions which have lack of data..., with a special emphasis on finding the energywindow inwhich optimizes the production of the auger-emitting radionuclide $^{193m}$Pt which can have potential in targeted radionuclide therapy. In addition, ten monitor foils of nickel, copper and iron were placed within each compartment of irridium, to measure the deuteron current in each compartment, via the well-characterized monitor reactions $^\text{nat}$Ni(d,x)$
^{61}$Cu,$^{56,58}$Co, $^text{nat}$Cu(d,x)$^{62,63,65}$Zn and $^\text{nat}$Fe(d,x)$^{56}$Co.\\ 

The cross sections were estimated using the analogy to the activation equation for thin targets, $A_0 = N_T\cdot \Phi(E_d)\sigma(E_d)(1-e^{-\lambda t_\text{irr}})$, where the energy of the deuteron beam was weighted-averaged over each foil. Prior to radiation, the number of target nuclei of each foil was characterized measuring the length across each foil and the mass. The deuteron-induced activity in each foil as a function of time since end of beam was obtained using gamma-ray spectroscopy on pre-calibrated high purity germanium detectors. The measured activities were fitted to decay curves to find the activity at the end of beam. Along with the weighted averaged beam current estimated from the monitor reactions, the flux weighted cross sections were estimated. This work along with previous experimental data suggests an energywindow from ca. 11 to 15 MeV which maximises the production of $^{193m}$Pt. The highest measured cross section in this work was 148.98 $\pm$ 5.54 mb at 13.51 MeV. 
 For irridium, the independent measurements of $^{188,190m2, 194m2}$Ir and $^{188, 189, 191, 193m}$Pt, and the cumulative cross sections $^{188, 189, 190,192,194}$Ir are reported.  From the monitor foils, the first observed measurements of cumulative cross sections of $^{61}$Co from copper, cumulative cross sections of $^{48}$V, $^{51}$Cr, $^{53}$Fe from iron, the cumualative cross sections of $^{59}$Fe and the independent cross sections of $^{56,57,57}$Co are also reported in this work, along with the other products produced. The results are compared to experimental data from the EXFOR database, along with the nuclear reaction modelling codes TALYS, TENDL, CoH, ALICE and \textcolor{red}{EMPIRE}, using the default parameters to aid improvements of the codes, \textcolor{red}{which did not match the experimental data very well in general, in particular in magnitude}.  \\




